\chapter{MIPS Assembly}
\label{c-mips}

\noindent
\begin{tabular}{p{1.5in}|p{.5in}|p{.5in}|p{.5in}|p{.5in}|p{.5in}|p{.5in}|}
  \cline{2-7}
  R-Format            & op & rs   & rt   & rd   & shamt & funct \\ \cline{2-7}
  Bits                & 6  & 5    & 5    & 5    & 5     & 6     \\ \cline{2-7}
  add \$r1,\$r2,\$r3  & 0  & \$r2 & \$r3 & \$r1 & 0     & 32    \\ \cline{2-7}
  addu \$r1,\$r2,\$r3 & 0  & \$r2 & \$r3 & \$r1 & 0     & 33    \\ \cline{2-7}
  sub \$r1,\$r2,\$r3  & 0  & \$r2 & \$r3 & \$r1 & 0     & 34    \\ \cline{2-7}
  subu \$r1,\$r2,\$r3 & 0  & \$r2 & \$r3 & \$r1 & 0     & 35    \\ \cline{2-7}
\end{tabular}

\vspace{.1in}\noindent
\begin{tabular}{p{1.5in}|p{.5in}|p{.5in}|p{.5in}|p{1.5in}|}
  \cline{2-5}
  I-Format          & op & rs   & rt   & address \\ \cline{2-5}
  Bits              & 6  & 5    & 5    & 16      \\ \cline{2-5}
  lw \$r1,off(\$r2) & 35 & \$r2 & \$r1 & off \\ \cline{2-5}
  sw \$r1,off(\$r2) & 43 & \$r2 & \$r1 & off \\ \cline{2-5}
\end{tabular}

\section{Registers}

\begin{tabular}{|r|l|p{4in}|}
  \hline
  Number & Name   & Use \\ \hline
  0      & \$zero & 0    \\ \hline
  1      & \$at   & assembler use             \\ \hline
  2      & \$v0   & return value (value)      \\ \hline
  3      & \$v1   & return value (value)      \\ \hline
  4      & \$a1   & parameters (arguments)    \\ \hline
  5      & \$a2   & parameters (arguments)    \\ \hline
  6      & \$a3   & parameters (arguments)    \\ \hline
  7      & \$a4   & parameters (arguments)    \\ \hline
  8      & \$t0   & temp (not saved)    \\ \hline
  9      & \$t1   & temp (not saved)    \\ \hline
  10     & \$t2   & temp (not saved)    \\ \hline
  11     & \$t3   & temp (not saved)    \\ \hline
  12     & \$t4   & temp (not saved)    \\ \hline
  13     & \$t5   & temp (not saved)    \\ \hline
  14     & \$t6   & temp (not saved)    \\ \hline
  15     & \$t7   & temp (not saved)    \\ \hline
  16     & \$s0   & saved temp    \\ \hline
  17     & \$s1   & saved temp    \\ \hline
  18     & \$s2   & saved temp    \\ \hline
  19     & \$s3   & saved temp    \\ \hline
  20     & \$s4   & saved temp    \\ \hline
  21     & \$s5   & saved temp    \\ \hline
  22     & \$s6   & saved temp    \\ \hline
  23     & \$s7   & saved temp    \\ \hline
  24     & \$t8   & temp (not saved)    \\ \hline
  25     & \$t9   & temp (not saved)    \\ \hline
  26     & \$k0   & OS    \\ \hline
  27     & \$k1   & OS    \\ \hline
  28     & \$gp   & global pointer (0x10008000) points to middle of 64k block    \\ \hline
  29     & \$sp   & stack pointer    \\ \hline
  30     & \$fp   & frame pointer    \\ \hline
  31     & \$ra   & return address    \\ \hline
\end{tabular}

\section{Keeping Your Ends Straight}

Big (LR) and little (RL) endian

Consistent (same for bits)

Sparc is inconsistent big-endian.

\begin{tabular}{ccc}
  % after \\: \hline or \cline{col1-col2} \cline{col3-col4} ...
  Endian & Consistent & Inconsistent \\
  \vspace{6pt}Big\vspace{6pt}    & \bigendianCon & \bigendianInc \\
  \vspace{6pt}Little\vspace{6pt} & \littleendianCon & \littleendianInc \\
\end{tabular}


\section{Data Structures}

Implement the following data structure in assembly then write a MIPS function to calculate $mykey.block=mykey.p\times mykey.q$.

\begin{verbatim}
     struct keys{
        int p;
        int q;
        int public;
        int private;
        int block;
     };
\end{verbatim}

{\color{ans}

\begin{verbatim}
     .data
     mykey:
     mykey_p: .word 0
     mykey_q: .word 0
     mykey_public: .word 0
     mykey_private: .word 0
     mykey_block: .word 0
     .set mykey_off_p=mykey_p - mykey
     .set mykey_off_q=mykey_q - mykey
     .set mykey_off_public=mykey_public - mykey
     .set mykey_off_private=mykey_private - mykey
     .set mykey_off_block=mykey_block - mykey

     .text
     ! Since this operates on data we know the location of,
     !  we don't need to pass anything
     la $t1, mykey
     lw $t2, mykey_off_p($t1)
     lw $t0, mykey_off_q($t1)
     mul $t0,$t2
     mflo $t0
     sw $t0,mykey_off_block($t1)
\end{verbatim}


}

\section{Register Passing}

\subsection{Exponentiation by Multiplication}

Write code to calculate $n^m$ for $n$ a non-zero finite integer and $m$ a non-negative integer.

\begin{verbatim}
  # n^m by loop
  # n !=0 finite in a0
  # m >=0 finite in a1
  # n^m          in v0
  # 0 in
  pow_by_loop:
    # ensure arguments are ok
    mov  $v0,$zero
    beqz $a0,pow_done
    bltz $a1,pow_done
    # m=0 and setup
    addi $v0,$v0,1
    beqz $a1,pow_done
    # m>0, loop
  pow_loop:
    mul  $v0,$a0
    mflo $v0
    subi $a1,$a1,1
    bgtz $a1,pow_loop
  pow_done:
    jr $ra
\end{verbatim}


Now how do we call it?  Assume that $n$ is in \$s0 and $m$ is at address "int\_m" and we want the result in \$s1.

\begin{verbatim}
  mov $a0,$s0
  la  $t1,int_m  # note I use $t1 for address scrap space
  lw  $a1,0($t1)
  jal pow_by_loop:
  mv $s1, $v0
\end{verbatim}




\subsection{Polynomial Evaluation}

Write the MIPS assembly code for the following function.  Assume the array a has been defined as size n+1.  You do not need to write the code to call the function but you need to state where you assume the parameters and return address will be.

\begin{verbatim}
  int poly_eval(int* a, int n, int x){
    y=a[n];
    for(i=n-1;i>=0;i--){
      y=y*x+a[i];
    }
    return y;
  }
\end{verbatim}

{\color{ans}

\begin{verbatim}
###########################################
# poly_eval
# leaf procedure to evaluate polynomials
# parameters:
# a1 : pointer to array of coefficients
# a2 : largest index in array
# a3 : point to evaluate polynomial
# return value:
# v0 : value of polynomial
# temporary values:
# t0 : offset in array
# t1 : address in array
poly_eval:  add  $t0, $a2, $a2         # four bytes per integer
            add  $t0, $t0, $t0
            add  $t1, $t0, $a1         # address of element to get
            lw   $v0,0($t1)            # initialize the answer
            beq  $t0,$zero, poly_done  # if only one element then done
poly_do:    mul  $v0, $v0, $a3         # y=y*x
            subi $t0, $t0, 4           # next coefficient is four bytes down
            add  $t1, $t0, $a1         # next coefficient's address
            lw   $t2, 0($t1)           # next coefficient
            add  $v0, $v0, $t2         # add next coefficient
            bne  $t0,$zero, poly_do    # more coefficients left
poly_done:  jr   $ra                   # return
\end{verbatim}

}


\subsection{Xor Encryption}

Consider the problem of xor encryption.  The $i^{\hbox{th}}$ cipher text character, $C_i$ is given by
\beqn
C_i = P_i \oplus K_i
\eeqn
where $P_i$ is the $i^{\hbox{th}}$ plain text character and $K_i$ is the $i^{\hbox{th}}$ key character.  The decryption is then given by
\beqn
P_i = C_i \oplus K_i.
\eeqn
This encryption method is thus symmetric.

\begin{verbatim}
#
#  xor
#
# $a0 contains plaintext
# $a1 contains key
# $a2 contains ciphertext
xor:
  mov $t3,$a1
  lb $t0,0($a0)
  lb $t1,0($a1)
xor_loop:
  xor $t2,$t0,$t1
  sb  $t2,0($a2)
  addi $a0,$a0,1
  addi $a1,$a1,1
  addi $a2,$a2,1
  lb $t0,0($a0)
  beqz $t0, xor_done
xor_load:
  lb $t1,0($a1)
  bgtz $t1, xor_loop
  mov $a1,$t3
  j xor_load
xor_done:
  jr $ra
\end{verbatim}

\subsection{Bubble Sort}


\begin{verbatim}
procedure bubbleSort( A : list of sortable items )
    n = length(A)
    repeat
       swapped = false
       for i = 1 to n-1 inclusive do
          if A[i-1] < A[i] then
             swap(A[i-1], A[i])
             swapped = true
          end if
       end for
       n = n - 1
    until not swapped
end procedure
\end{verbatim}

\begin{verbatim}
#
#  Bubble Sort
#
# $a0 points to start of array
# $a1 points to last element in array
         move $t0, $a0
         move $t1, $a1
outter:  move $t4, $0           # swapped this round is false
         lw   $t2, 0($t0)       # get the left compare value
inner:   lw   $t3, 4($t0)       # get the right compare value
         addi $t0, $t0, 4       # increment the left pointer
         ble  $t2, $t3, no_swap # if right>left  swap, else don't
swap:    sw   $t2, 0($t0)       # place left value on right in array
         sw   $t3, -4($t0)      # place right value on left in array
         ori  $t4, $0, 1        # set swapped true
         blt  $t0, $t1, inner   # if not at end then keep going
         subi $t1, $t1, 4       # if at end then shorten the list
         move $t0, $a0          # reset the first element
         b    outter            # start another major loop
no_swap: move $t2, $t3          # no swap, so right element is new left
         blt  $t0, $t1, inner   # if not at end then keep going
         subi $t1, $t1, 4       # if at end then shorten the list
         move $t0, $a0          # reset the first element
         bnez $t4, outter       # start another major loop if swapped
\end{verbatim}



\section{Block Passing}

Let us reconsider affine encryption as outlined in Section~\ref{s-stack-affine}

We will be passed a pointer to a string of plaintext, *P, and the length of the string, len.  Additionally we need the affine parameters a, b, and n.  Five parameters cannot be passed in registers, as we only have four, so we will use a block.  Modulus is handled nicely by div in mips so we have no problems there.  To be really careful I will use divu (unsigned division).

If an error is detected I will use {\tt break \$zero} to halt execution.  You could also write your own error handler but that did not seem reasonable given the length of the code already (3 pages).  I have tried to exhibit good commenting techniques.  They greatly simplify others reading and editing.
\clearpage

\begin{verbatim}
##############################################
# _affine_encrypt
#
#
# Author: Keith Schubert
# Date  : Nov 4, 2005
# Desc  : Affine encryption of a string
# Method: calculate then modulus.
# BlkPtr: _affine_encrypt_block_pointer
#         var  contents     offset
# Return:
# RetAdd:                        _affine_encrypt_off_ra
# Params: *P   plaintext         _affine_encrypt_off_p
#         len  plaintext.length  _affine_encrypt_off_len
#         *C   ciphertext        _affine_encrypt_off_c
#         a    affine scale      _affine_encrypt_off_a
#         b    affine shift      _affine_encrypt_off_b
#         n    # of code chars   _affine_encrypt_off_n
# Pre   :
# Post  : contents of $t0-$t8 changed, $ra changed
#
##############################################
.data
_affine_encrypt_block_pointer:
_affine_encrypt_base_ra:
    .word 0
_affine_encrypt_base_p:
    .word 0
_affine_encrypt_base_c:
    .word 0
_affine_encrypt_base_len:
    .word 0
_affine_encrypt_base_a:
    .word 0
_affine_encrypt_base_b:
    .word 0
_affine_encrypt_base_n:
    .word 0
_affine_encrypt_block_bottom:

    .set _affine_encrypt_off_ra =
         _affine_encrypt_base_ra - _affine_encrypt_block_pointer
    .set _affine_encrypt_off_p =
         _affine_encrypt_base_p - _affine_encrypt_block_pointer
    .set _affine_encrypt_off_c =
         _affine_encrypt_base_c - _affine_encrypt_block_pointer
    .set _affine_encrypt_off_len =
         _affine_encrypt_base_len - _affine_encrypt_block_pointer
    .set _affine_encrypt_off_a =
         _affine_encrypt_base_a - _affine_encrypt_block_pointer
    .set _affine_encrypt_off_b =
         _affine_encrypt_base_b - _affine_encrypt_block_pointer
    .set _affine_encrypt_off_n =
         _affine_encrypt_base_n - _affine_encrypt_block_pointer
    .set _affine_encrypt_block_size =
         _affine_encrypt_block_bottom - _affine_encrypt_block_pointer
.text
_affine_encrypt:

#
# Setup
#
# t0 = current char index
# t1 = *p
# t2 = *c
# t3 = len
# t4 = a
# t5 = b
# t6 = n
# t7 = current char
# t8 = effective address
#
la  $t1, _affine_encrypt_block_pointer
lw  $t2, _affine_encrypt_off_c($t1)
lw  $t3, _affine_encrypt_off_len($t1)
bgtz $t3,_affine_encrypt_len_ok
break $zero  #error stop execution
_affine_encrypt_len_ok
lw  $t4, _affine_encrypt_off_a($t1)
lw  $t6, _affine_encrypt_off_n($t1)

#
# Data validity
#
# see if gcd(a,n)=1
mov  $t5, $t4
mov  $t0, $t6
break $zero  # MIPS error
break $zero
# Euclid's alg
_affine_encrypt_Euclid:
divu $t5,$t0
mov  $t5,$t0
mfhi $t0
bgez $t0,_affine_encrypt_Euclid
subi $t5,$t5,1
beqz $t5,_affine_encrypt_ab_ok
break $zero
_affine_encrypt_ab_ok:

#
# Finish loads
#
lw  $t5, _affine_encrypt_off_b($t1)
lw  $t1, _affine_encrypt_off_p($t1)
mov $t0,$zero

#
# main loop
#
# get char, scale, shift, mod, then store
#
_affine_encrypt_loop:
add  $t8,$t0,$t1
lbu  $t7,0($t8)
mulu $t7,$t4
mflo $t7
add  $t7,$t7,$t5
divu $t7,$t6
mfhi $t7
add  $t8,$t0,$t2
sb   $t7,0($t8)
addi $t0,$t0,1
sle  $t8,$t0,$t3
beqz $t8,_affine_encrypt_loop

#
# Return
#
la $t1,_affine_encrypt_block_pointer
lw $ra,_affine_encrypt_off_ra($t1)
jr $ra
\end{verbatim}

\section{Stack Passing}

On some machines you can/must manually allocate your own stack using .bss and .skip.  On MIPS the stack is predefined and the OS initializes the stack pointer for you.  We are going to define two macros, push and pop.  To define a macro we use .macro and .endmacro.

\begin{verbatim}
.macro push arg1
  addui $sp,$sp,-4  # allocate space
  sw    arg1,0($sp) # place contents
.endmacro
\end{verbatim}

\begin{verbatim}
.macro pop arg1
  lw    arg1,0($sp) # get contents
  addui $sp,$sp,4   # deallocate space
.endmacro
\end{verbatim}

Let's consider Euclid's algorithm for finding the GCD of two numbers

\begin{enumerate}
\item Let a,b be positive numbers
\item a=b and b=a mod b
\item repeat 2 until b=0
\item gcd=a
\end{enumerate}

\begin{tabular}{lrrp{24pt}lrr}
iteration & a  & b  && iteration & a  & b  \\ \cline{1-3}\cline{5-7}
1         & 15 & 12 && 1         & 49 & 84 \\
2         & 12 & 3  && 2         & 84 & 49 \\
3         & 3  & 0  && 3         & 49 & 35 \\
          &    &    && 4         & 35 & 14 \\
          &    &    && 5         & 14 & 7  \\
          &    &    && 6         & 7  & 0  \\
\end{tabular}

\begin{verbatim}
###############################################
#
# _euclid_alg_gcd
#
# Author: Keith Schubert
# Date  : Nov 4, 2005
# Desc  : greatest common divisor
# Method: Euclid's Algorithm, recursive
#         var       offset
# Return: gcd       _euclid_alg_gcd_off_gcd
# RetAdd: ra        _euclid_alg_gcd_off_ra
# Params: a         _euclid_alg_gcd_off_a
#         b         _euclid_alg_gcd_off_b
# Pre   :
# Post  : contents of $t0-$t8 changed, $ra changed
#
##############################################
_euclid_alg_gcd:

\end{verbatim}



\subsection{Towers of Hanoi}

Implement a recursive function to solve the towers of Hanoi in MIPS.

{\color{ans}
\begin{verbatim}
     #
     # hanoi
     #
     # Frame: Return address
     #        *Answer
     #        Answer Size
     #        Number of disks
     #        Free
     #        Destination
     #        Source
     .set hanoi_off_ra=0
     .set hanoi_off_ans=4
     .set hanoi_off_size=8
     .set hanoi_off_num=12
     .set hanoi_off_free=16
     .set hanoi_off_dest=20
     .set hanoi_off_source=24
     .set hanoi_allocate=-28
     .set hanoi_deallocate=28
     .set newline="\n"
     .set arrow=">"
     hanoi:
       lw $t0,hanoi_off_num($t0)
       subi $t0,$t0,1
       blez $t0,done
       #
       # move stack-1 to free
       mov $fp,$sp
       addiu $sp,$sp,hanoi_allocate
       sw $t0,hanoi_off_num($sp)    # num-1
       lw $t0,hanoi_off_ans($fp)    # same string
       sw $t0,hanoi_off_ans($sp)
       lw $t0,hanoi_off_size($fp)   # same size
       sw $t0,hanoi_off_size($sp)
       lw $t0,hanoi_off_free($fp)   # new dest=free
       sw $t0,hanoi_off_dest($sp)
       lw $t0,hanoi_off_dest($fp)   # new free=dest
       sw $t0,hanoi_off_free($sp)
       lw $t0,hanoi_off_source($fp) # source same
       sw $t0,hanoi_off_source($sp)
       la $t0,back1                 # return address
       sw $t0,hanoi_off_ra($sp)
       j hanoi
       back1:
       #don't deallocate yet, we are calling another in a sec
       #
       # store "source>dest\nNull"
       lw $t1,hanoi_off_ans($sp)
       lw $t0,hanoi_off_size($sp)
       add $t1,$t1,$t0
       lw $t2,hanoi_off_source($sp)
       sb $t2,0($t1)
       li $t2,arrow
       sb $t2,1($t1)
       lw $t2,hanoi_off_dest($sp)
       sb $t2,2($t1)
       li $t2,newline
       sb $t2,3($t1)
       sb $zero,4($t1)
       addi $t0,$t0,4
       sw $t0,hanoi_off_size($sp)
       #
       # move stack-1 to dest
       lw $t0,hanoi_off_dest($fp)     # same dest
       sw $t0,hanoi_off_dest($sp)
       lw $t0,hanoi_off_source($fp)   # new free=source
       sw $t0,hanoi_off_free($sp)
       lw $t0,hanoi_off_free($fp)     # new source=free
       sw $t0,hanoi_off_source($sp)
       la $t0,back2                   # return address
       sw $t0,hanoi_off_ra($sp)
       j hanoi
       back2:
       addiu $sp,$sp,hanoi_deallocate
       lw $ra,hanoi_off_ra($sp)
       jr $ra

     done:
       # store "source>dest\nNull"
       lw $t1,hanoi_off_ans($sp)
       lw $t0,hanoi_off_size($sp)
       add $t1,$t1,$t0
       lw $t2,hanoi_off_source($sp)
       sb $t2,0($t1)
       li $t2,arrow
       sb $t2,1($t1)
       lw $t2,hanoi_off_dest($sp)
       sb $t2,2($t1)
       li $t2,newline
       sb $t2,3($t1)
       sb $zero,4($t1)
       addi $t0,$t0,4
       sw $t0,hanoi_off_size($sp)
       lw $ra,hanoi_off_ra($sp)
       jr $ra
\end{verbatim}

}

\subsection{Tracing Code}
The code that follows, implements the algorithm
\beqn
n_{k+1}=\left\{\begin{matrix} 3n_k+1       & \hbox{if }n_k\hbox{ is odd} \\
                       \frac{n_k}{2} & \hbox{if }n_k\hbox{ is even}\end{matrix}\right.
\eeqn
in MIPS.  Trace the code by showing how the register values change.  What is the value that is returned?  Note: this code is a somewhat famous problem in number theory.  The problem is to prove that starting at any number, the algorithm will bring you to 1.

\begin{verbatim}
     !   code                    $t0  |  $a0  |  $v0
     !                                |   3   |
     !--------------------------------------------
     secret:                    !     |       |
       bgtz $a0, ok             !     |       |
       break $zero              !     |       |
     ok:                        !     |       |
       addi $v0,$zero,1         !     |       |
       subi $t0,$a0,1           !     |       |
       beqz $t0, end            !     |       |
     loop:                      !     |       |
       addi $v0,$v0,1           !     |       |
       andi $t0,$a0,1           !     |       |
       beqz $t0, even           !     |       |
       sll  $t0,$a0,1           !     |       |
       add  $a0,$a0,$t0         !     |       |
       addi $a0,$a0,1           !     |       |
       b loop                   !     |       |
     even:                      !     |       |
       sra  $a0,$a0,1           !     |       |
       subi $t0,$a0,1           !     |       |
       bgtz $t0, loop           !     |       |
     end:
\end{verbatim}

{\color{ans}
I will show changes on successive loops by placing a comma and then the new value

\begin{verbatim}
     #   code                    $t0             |  $a0            |  $v0
     #                                           |   3             |
     #-----------------------------------------------------------------------------
     secret: bgtz $a0, ok      #                | 3               |
             break $zero       #                |                 |
     ok:     addi $v0,$zero,1  #                | 3               | 1
             subi $t0,$a0,1    # 2              | 3               | 1
             beqz $t0, end     # 2              | 3               | 1
     loop:   addi $v0,$v0,1    # 2,6,4 ,10,7,3,1| 3,10,5 ,16,8,4,2| 2,3,4,5,6,7,8
             andi $t0,$a0,1    # 1,0,1 ,0 ,0,0,0| 3,10,5 ,16,8,4,2| 2,3,4,5,6,7,8
             beqz $t0, even    # 1,0,1 ,0 ,0,0,0| 3,10,5 ,16,8,4,2| 2,3,4,5,6,7,8
             sll  $t0,$a0,1    # 6  ,10         | 3   ,5          | 2  ,4
             add  $a0,$a0,$t0  # 6  ,10         | 9   ,15         | 2  ,4
             addi $a0,$a0,1    # 6  ,10         | 10  ,16         | 2  ,4
             b loop            # 6  ,10         | 10  ,16         | 2  ,4
     even:   sra  $a0,$a0,1    #   0   ,0 ,0,0,0|   5    ,8 ,4,2,1|   3  ,5,6,7,8
             subi $t0,$a0,1    #   4   ,7 ,3,1,0|   5    ,8 ,4,2,1|   3  ,5,6,7,8
             bgtz $t0, loop    #   4   ,7 ,3,1,0|   5    ,8 ,4,2,1|   3  ,5,6,7,8
     end:
\end{verbatim}

Returns 8.
}
