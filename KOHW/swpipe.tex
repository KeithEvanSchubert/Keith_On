\documentclass{article}

\begin{document}
Consider the following code to calculate the Fibonacci numbers.

\begin{verbatim}
    top: add r4, r3, r2
         mov r2, r3
         mov r3, r4
         addi r1, r1, -1
         brgtz r1, top
\end{verbatim}

The first three instructions are the data manipulations, and the last two are loop overhead (indexing and branching).  There is a large amount of wasted effort spent in moving data around.  Consider two loops worth of just the data manipulation portions.

\begin{verbatim}
         add r4, r3, r2
         mov r2, r3
         mov r3, r4
         add r4, r3, r2
         mov r2, r3
         mov r3, r4
\end{verbatim}

Note that the ``mov'' commands are only to set up the problem for the next loop.  In particular the contents of r2 are removed and the contents of r3 and r4 are shuffled.  Consider the following change.

\begin{verbatim}
         add r2, r3, r2
         add r4, r3, r2
         mov r3, r4
\end{verbatim}

The contents of the registers are the same at the end of the loop, as the original, but considerable savings have been achieved.  by noting the last mov command only shifts the results of the second add, we note that it is equivalent to the following

\begin{verbatim}
         add r2, r3, r2
         add r3, r3, r2
\end{verbatim}

Thus by unrolling we can see the loop is equivalent to


\begin{verbatim}
    top: add r2, r3, r2
         add r3, r3, r2
         addi r1, r1, -2
         brgtz r1, top
         mov r4, r3
         breqz r1, exit
         mov r4, r2
   exit:
\end{verbatim}

Note the last three commands are cleanup only, so two iterations of the original loop can be done in less instructions than the unoptimized code. The loop can be scheduled efficiently on a two pipeline machine as

\begin{tabular}{lll}
top:  & add r2, r3, r2 & addi r1, r1, -2 \\
      & add r3, r3, r2 & bgtqz r1, top   \\
      & mov r4, r3     & breqz r1, exit  \\
      & mov r4, r2     &                 \\
exit: &                &                 \\
\end{tabular}



%\begin{eqnarray}
%f(n)   &=& f(n-1)+f(n-2) \\
%f(n+1) &=& f(n)+f(n-1) \\
%       &=& 2f(n-1)+f(n-2) \\
%       &=& 3f(n-2)+2f(n-3) \\
%       &=& 5f(n-3)+3f(n-4) \\
%       &=& 8f(n-4)+5f(n-5) \\
%       &=& 8f(n-4)+5f(n-6)+ 5f(n-7) \\
%       &=& 8f(n-4)+10f(n-7)+ 5f(n-8)
%\end{eqnarray}

%r1 = r2 + r3    r4 = r5 + r6     r5 = sla r4,8        r6 = r7 + r8     r7 = sla r1,2


\section{Software Pipelining}

Returning to the original code

\begin{verbatim}
    top: add r4, r3, r2
         mov r2, r3
         mov r3, r4
         addi r1, r1, -1
         brgtz r1, top
\end{verbatim}

And let us again consider two iterations of the Fibonacci number loop.

\begin{verbatim}
         add r4, r3, r2
         mov r2, r3
         mov r3, r4
         add r4, r3, r2
         mov r2, r3
         mov r3, r4
\end{verbatim}

First note that each pair of moves can be done simultaneously.

\begin{verbatim}
         add r4, r3, r2
         mov r2, r3         mov r3, r4
         add r4, r3, r2
         mov r2, r3         mov r3, r4
\end{verbatim}

Now we will move the second add ahead in the scheduling so it is simultaneous with the first moves.

\begin{verbatim}
         add r4, r3, r2
         mov r2, r3         mov r3, r4         add r4, r4, r3
         mov r2, r3         mov r3, r4
\end{verbatim}

Now note that the \texttt{mov r2, r3} commands are useless and can be dropped.

\begin{verbatim}
         add r4, r3, r2
         mov r3, r4         add r4, r4, r3
         mov r3, r4
\end{verbatim}

This suggests the following parallel execuation

\begin{tabular}{llll}
mov r2, r3  & add r3, r3, r2 & addi r1, r1, -1 & brgtz r1, top \\
\end{tabular}



\begin{tabular}{rrrr}
time  &r3      & r2   & r1  \\
0     & 1      & 1    & 3   \\
1     & 2      & 1    & 2  \\
2     & 3      & 2    & 1 \\
3     & 5      & 3    & 0 \\
\end{tabular}


\subsection{Example}
Consider the following code

\begin{verbatim}
    top: ld r2, 0(r1)
         addi r3, r2, 1
         st r3, 0(r1)
         addi r1, r1, 4
         brlt r1, r4, top
\end{verbatim}

\begin{tabular}{lll}
st r3, 0(r1) & addi r3, r2, 1 & ld r2, 8(r1) \\
\end{tabular}

\end{document}