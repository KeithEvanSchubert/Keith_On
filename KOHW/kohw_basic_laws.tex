\chapter{Basic Laws}

\section{Coulomb's Law}

\begin{eqnarray}
F
&=& \frac{kq_1q_2}{r^2} \\
&=& \frac{q_1q_2}{4\pi\varepsilon_0r^2}
\end{eqnarray}
Note: Negative forces attract and positive repel.

Coulomb's constant, $k$ is given by
\begin{eqnarray}
k &=& \frac{1}{4\pi\varepsilon_0} \\
&\approx& 9\times 10^9 \;[N\cdot m^2/C^2]
\end{eqnarray}

\section{Maxwell's Laws}

Maxwell's four Laws have individual names:
\begin{description}
  \item[Gauss' Law of Electricity]
  \item[Gauss' Law of Magnetism]
  \item[Faraday's Law of Induction] Basis of electrical generators and inductors
  \item[Ampere's Law]
\end{description}

The symbols used are:
\begin{description}
  \item[$E$] Electric field
  \item[$B$] Magnetic field
  \item[$D$] Electric displacement
  \item[$H$] Magnetic field strength
  \item[$\rho$] charge density
  \item[$\varepsilon$] permittivity
  \item[$\varepsilon_0$] permittivity of free space
  \item[$\mu$] permeability
  \item[$\mu_0$] permeability of free space
  \item[$M$] Magnetization
  \item[$i$] electric current
  \item[$J$] current density
  \item[$c$] speed of light, $c=\frac{1}{\sqrt{\mu_0\varepsilon_0}}$.
  \item[$P$] Polarization
  \item[$k$] Coulomb's constant, $k=\frac{1}{4\pi\varepsilon_0}$.
\end{description}

\subsection{Gauss' Law for Electricity}

Integral Form
\begin{eqnarray}
\oint\vec{E}\cdot\vec{A}
&=& \frac{q}{\varepsilon_0}
\end{eqnarray}

Differential Form
\begin{eqnarray}
\nabla\cdot D
&=& \rho
\end{eqnarray}
where $D$ is
\begin{description}
\item[General Case] $D = \varepsilon_0E+P$
\item[Free Space] $D = \varepsilon_0E$
\item[Isotropic Linear Dielectric] $D = \varepsilon E$
\end{description}


\subsection{Gauss' Law for Magnetism}

Integral Form
\begin{eqnarray}
\oint\vec{B}\cdot\vec{A}
&=& 0
\end{eqnarray}

Differential Form
\begin{eqnarray}
\nabla\cdot B
&=& 0
\end{eqnarray}


\subsection{Faraday's Law of Induction}

Simplified Form
\begin{eqnarray}
V=-N\frac{\Delta(BA)}{\Delta t}
\end{eqnarray}
\begin{description}
  \item[N] number of turns in the coil
  \item[B] Magnetic Field
  \item[A] The cross sectional area (perpendicular to the magnetic field)
  \item[t] Time
  \item[V] Voltage or EMF (electro-motive force)
\end{description}

Integral Form
\begin{eqnarray}
\oint\vec{E}d\vec{s}
&=& -\frac{d\Phi_B}{dt}\\
&=& EMF
\end{eqnarray}

Differential Form
\begin{eqnarray}
\nabla\times E
&=& -\frac{\partial B}{\partial t}
\end{eqnarray}


\subsection{Ampere's Law}

Integral Form
\begin{eqnarray}
\oint B\cdot ds
&=& \mu_0i+\frac{1}{c^2}\frac{\partial}{\partial t}\int E\cdot dA
\end{eqnarray}

Differential Form
\begin{eqnarray}
\nabla\times H
&=& J+\frac{\partial D}{\partial t}
\end{eqnarray}


where $D$ is
\begin{description}
\item[General Case] $D = \varepsilon_0E+P$
\item[Free Space] $D = \varepsilon_0E$
\item[Isotropic Linear Dielectric] $D = \varepsilon E$
\end{description}
and $H$ is
\begin{description}
\item[General Case] $B = \mu_0(H+M)$
\item[Free Space] $B = \mu_0H$
\item[Isotropic Linear Magnetic Medium] $B = \mu H$
\end{description} 