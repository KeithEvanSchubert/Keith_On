\chapter{What is Engineering?}


\section{Fun Quotes}
Thanks to Harry T. Roman of East Orange, N.J., USA, who compiled the following 21 definitions of Engineering.

\Quote{The application of science to the common purpose of life.}{Count Rumford (1799)}

\Quote{Engineering is the art of directing the great sources of power in nature for the use and convenience of man.}{Thomas Tredgold (1828)}

\Quote{It would be well if engineering were less generally thought of, and even defined, as the art of constructing. In a certain sense it is rather the art of not constructing; or, to define it rudely but not inaptly, it is the art of doing that well with one dollar which any bungler can be with two after a fashion.}{A. M. Wellington (1887)}

\Quote{Engineering is the art of organizing and directing men and controlling the forces and materials of nature for the benefit of the human race.}{Henry G. Stott (1907)}

\Quote{Engineering is the science of economy, of conserving the energy, kinetic and potential, provided and stored up by nature for the use of man. It is the business of engineering to utilize this energy to the best advantage, so that there may be the least possible waste.}{Willard A. Smith (1908)}

\Quote{Engineering is the conscious application of science to the problems of economic production.}{H. P. Gillette (1910)}

\Quote{Engineering is the art or science of utilizing, directing or instructing others in the utilization of the principles, forces, properties and substance of nature in the production, manufacture, construction, operation and use of things ... or of means, methods, machines, devices and structures ...}{Alfred W. Kiddle (1920)}

\Quote{Engineering is the practice of safe and economic application of the scientific laws governing the forces and materials of nature by means of organization, design and construction, for the general benefit of mankind.}{S. E. Lindsay (1920)}

\Quote{Engineering is an activity other than purely manual and physical work which brings about the utilization of the materials and laws of nature for the good of humanity.}{R. E. Hellmund (1929)}

\Quote{Engineering is the science and art of efficient dealing with materials and forces ... it involves the most economic design and execution ... assuring, when properly performed, the most advantageous combination of accuracy, safety, durability, speed, simplicity, efficiency, and economy possible for the conditions of design and service.}{J. A. L. Waddell, Frank W. Skinner, and H. E. Wessman (1933)}

\Quote{Engineering is the professional and systematic application of science to the efficient utilization of natural resources to produce wealth.}{T. J. Hoover and J. C. L. Fish (1941)}

\Quote{The activity characteristic of professional engineering is the design of structures, machines, circuits, or processes, or of combinations of these elements into systems or plants and the analysis and prediction of their performance and costs under specified working conditions.}{M. P. O'Brien (1954)}

\Quote{The ideal engineer is a composite ... He is not a scientist, he is not a mathematician, he is not a sociologist or a writer; but he may use the knowledge and techniques of any or all of these disciplines in solving engineering problems.}{N. W. Dougherty (1955)}

\Quote{Engineers participate in the activities which make the resources of nature available in a form beneficial to man and provide systems which will perform optimally and economically.}{L. M. K. Boelter (1957)}

\Quote{The engineer is the key figure in the material progress of the world. It is his engineering that makes a reality of the potential value of science by translating scientific knowledge into tools, resources, energy and labor to bring them into the service of man ... To make contributions of this kind the engineer requires the imagination to visualize the needs of society and to appreciate what is possible as well as the technological and broad social age understanding to bring his vision to reality.}{Sir Eric Ashby (1958)}

\Quote{The engineer has been, and is, a maker of history.}{James Kip Finch (1960)}

\Quote{Engineering is the profession in which a knowledge of the mathematical and natural sciences gained by study, experience, and practice is applied with judgment to develop ways to utilize, economically, the materials and forces of nature for the benefit of mankind.}{Engineers Council for Professional Development (1961/1979)}

\Quote{Engineering is the professional art of applying science to the optimum conversion of natural resources to the benefit of man.}{Ralph J. Smith (1962)}

\Quote{Engineering is not merely knowing and being knowledgeable, like a walking encyclopedia; engineering is not merely analysis; engineering is not merely the possession of the capacity to get elegant solutions to non-existent engineering problems; engineering is practicing the art of the organized forcing of technological change ... Engineers operate at the interface between science and society ...}{Dean Gordon Brown; Massachusetts Institute of Technology (1962)}

\Quote{The story of civilization is, in a sense, the story of engineering - that long and arduous struggle to make the forces of nature work for man's good.}{L. Sprague DeCamp (1963)}

\Quote{Engineering is the art or science of making practical.}{Samuel C. Florman (1976)}

\subsection{Class Quotes}


\Quote{Engineers use vision and resources to create a design that is cost-efficient and maximizes performance.}{Jonathan Carranza, Jason Fredrick, Charles Korman, Sarah Wade (2008)}

\Quote{Engineering is a creative and altruistic way to benefit society using broad knowledge and tools efficiently to design technology.}{Maged Assad, Diane Bernal, Erin Powers, and Amber Thomas (2008)}

\Quote{If you are going to reinvent the wheel, do it as optimally, efficiently as possible... and damnit make a better wheel! And remember when designing embrace the laziness.}{Tim Castelli, Steven Parker, and Sachithra Udunuwarage (2008)}

\section{Fields}

The ``Big Three'':
\begin{description}
  \item[Civil] The name comes from civilian rather than military engineering, but is used of large construction projects such as buildings and roads.
  \item[Electrical] Design of electrical or electronic components.  Includes such areas as analog and digital circuits, electro-magnetics, signals, power, controls, estimation.  This has many sub-fields such as ours - computer engineering.
  \item[Mechanical] Design of machines and devices.  Typically includes such areas as statics, dynamics, materials, fluids, and thermodynamics.  This has many sub-fields such as industrial and aerospace.
\end{description}

\section{Profession}

Engineering is a true profession, as opposed to many areas of study often so called because engineering has a licensing process.  To become licensed, you must pass the EIT/FE\footnote{Engineer-In-Training/Fundamentals of Engineering exam is a one day, 8 hour exam with around a 20\% pass rate when I took it.} exam, then work a number of years under a professional engineer (PE) or in certain fields (such a aerospace) or get education credit (ABET accredited schools only)\footnote{The number of years required varies by the area you are trying to get a license in.}, then pass a second exam (PE exam)\footnote{A 2 day exam of about 8 hours a day.}.  Typically continuing education courses are required for renewal.

As a profession, engineering is overseen by state laws and professional societies.  Ethics in engineering is thus not an arbitrary concept, it is a professional standard and requirement on our actions. 