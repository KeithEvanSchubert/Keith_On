\chapter{Engineering Ethics}\label{ch-eng-ethics}

Engineering ethics are really more professional codes of conduct than true ethical systems.  The main difference is that codes of conduct specify conduct rather than specifying the way to evaluate a situation.  Amongst professional codes of conduct, there are two general methods, specific and general.  Specific codes of conduct are typified by the Association of Computer Machinist's (ACM) code of conduct, which is very detailed in what is to be done in different situations.  On the other extreme is the Institute of Electrical and Electronic Engineer's (IEEE) code of conduct, which has only some general principles.

\section{IEEE Code of Ethics}

As per IEEE Bylaw I-104.14, membership in IEEE in any grade shall carry the obligation to abide by the IEEE Code of Ethics (IEEE Policy 7.8) as stated below.



\vspace{.1in}
We, the members of the IEEE, in recognition of the importance of our technologies in affecting the quality of life throughout the world, and in accepting a personal obligation to our profession, its members and the communities we serve, do hereby commit ourselves to the highest ethical and professional conduct and agree:
\begin{itemize}
\item to accept responsibility in making decisions consistent with the safety, health and welfare of the public, and to disclose promptly factors that might endanger the public or the environment;
\item to avoid real or perceived conflicts of interest whenever possible, and to disclose them to affected parties when they do exist;
\item to be honest and realistic in stating claims or estimates based on available data;
\item to reject bribery in all its forms;
\item to improve the understanding of technology, its appropriate application, and potential consequences;
\item to maintain and improve our technical competence and to undertake technological tasks for others only if qualified by training or experience, or after full disclosure of pertinent limitations;
\item to seek, accept, and offer honest criticism of technical work, to acknowledge and correct errors, and to credit properly the contributions of others;
\item to treat fairly all persons regardless of such factors as race, religion, gender, disability, age, or national origin;
\item to avoid injuring others, their property, reputation, or employment by false or malicious action;
\item to assist colleagues and co-workers in their professional development and to support them in following this code of ethics.
\end{itemize}

Approved by the IEEE Board of Directors

February 2006




\section{ACM Code of Ethics and Professional Conduct
}


Adopted by ACM Council 10/16/92.


\vspace{.2in}\noindent{\large Preamble}\vspace{.1in}

Commitment to ethical professional conduct is expected of every member (voting members, associate members, and student members) of the Association for Computing Machinery (ACM).

This code, consisting of 24 imperatives formulated as statements of personal responsibility, identifies the elements of such a commitment. It contains many, but not all, issues professionals are likely to face. Section 1 outlines fundamental ethical considerations, while Section 2 addresses additional, more specific considerations of professional conduct. Statements in Section 3 pertain more specifically to individuals who have a leadership role, whether in the workplace or in a volunteer capacity such as with organizations like ACM. Principles involving compliance with this Code are given in Section 4.

The Code shall be supplemented by a set of Guidelines, which provide explanation to assist members in dealing with the various issues contained in the Code. It is expected that the Guidelines will be changed more frequently than the Code.

The Code and its supplemented Guidelines are intended to serve as a basis for ethical decision making in the conduct of professional work. Secondarily, they may serve as a basis for judging the merit of a formal complaint pertaining to violation of professional ethical standards.

It should be noted that although computing is not mentioned in the imperatives of Section 1, the Code is concerned with how these fundamental imperatives apply to one's conduct as a computing professional. These imperatives are expressed in a general form to emphasize that ethical principles which apply to computer ethics are derived from more general ethical principles.

It is understood that some words and phrases in a code of ethics are subject to varying interpretations, and that any ethical principle may conflict with other ethical principles in specific situations. Questions related to ethical conflicts can best be answered by thoughtful consideration of fundamental principles, rather than reliance on detailed regulations.




\vspace{.2in}\noindent{\large 1. GENERAL MORAL IMPERATIVES.}\vspace{.1in}

As an ACM member I will ....

\vspace{.1in}\noindent\textbf{1.1 Contribute to society and human well-being.}

This principle concerning the quality of life of all people affirms an obligation to protect fundamental human rights and to respect the diversity of all cultures. An essential aim of computing professionals is to minimize negative consequences of computing systems, including threats to health and safety. When designing or implementing systems, computing professionals must attempt to ensure that the products of their efforts will be used in socially responsible ways, will meet social needs, and will avoid harmful effects to health and welfare.

In addition to a safe social environment, human well-being includes a safe natural environment. Therefore, computing professionals who design and develop systems must be alert to, and make others aware of, any potential damage to the local or global environment.

\vspace{.1in}\noindent\textbf{1.2 Avoid harm to others.}

``Harm'' means injury or negative consequences, such as undesirable loss of information, loss of property, property damage, or unwanted environmental impacts. This principle prohibits use of computing technology in ways that result in harm to any of the following: users, the general public, employees, employers. Harmful actions include intentional destruction or modification of files and programs leading to serious loss of resources or unnecessary expenditure of human resources such as the time and effort required to purge systems of ``computer viruses.''

Well-intended actions, including those that accomplish assigned duties, may lead to harm unexpectedly. In such an event the responsible person or persons are obligated to undo or mitigate the negative consequences as much as possible. One way to avoid unintentional harm is to carefully consider potential impacts on all those affected by decisions made during design and implementation.

To minimize the possibility of indirectly harming others, computing professionals must minimize malfunctions by following generally accepted standards for system design and testing. Furthermore, it is often necessary to assess the social consequences of systems to project the likelihood of any serious harm to others. If system features are misrepresented to users, coworkers, or supervisors, the individual computing professional is responsible for any resulting injury.

In the work environment the computing professional has the additional obligation to report any signs of system dangers that might result in serious personal or social damage. If one's superiors do not act to curtail or mitigate such dangers, it may be necessary to ``blow the whistle'' to help correct the problem or reduce the risk. However, capricious or misguided reporting of violations can, itself, be harmful. Before reporting violations, all relevant aspects of the incident must be thoroughly assessed. In particular, the assessment of risk and responsibility must be credible. It is suggested that advice be sought from other computing professionals. See principle 2.5 regarding thorough evaluations.

\vspace{.1in}\noindent\textbf{1.3 Be honest and trustworthy.}

Honesty is an essential component of trust. Without trust an organization cannot function effectively. The honest computing professional will not make deliberately false or deceptive claims about a system or system design, but will instead provide full disclosure of all pertinent system limitations and problems.

A computer professional has a duty to be honest about his or her own qualifications, and about any circumstances that might lead to conflicts of interest.

Membership in volunteer organizations such as ACM may at times place individuals in situations where their statements or actions could be interpreted as carrying the ``weight'' of a larger group of professionals. An ACM member will exercise care to not misrepresent ACM or positions and policies of ACM or any ACM units.

\vspace{.1in}\noindent\textbf{1.4 Be fair and take action not to discriminate.}

The values of equality, tolerance, respect for others, and the principles of equal justice govern this imperative. Discrimination on the basis of race, sex, religion, age, disability, national origin, or other such factors is an explicit violation of ACM policy and will not be tolerated.

Inequities between different groups of people may result from the use or misuse of information and technology. In a fair society,all individuals would have equal opportunity to participate in, or benefit from, the use of computer resources regardless of race, sex, religion, age, disability, national origin or other such similar factors. However, these ideals do not justify unauthorized use of computer resources nor do they provide an adequate basis for violation of any other ethical imperatives of this code.

\vspace{.1in}\noindent\textbf{1.5 Honor property rights including copyrights and patent.}

Violation of copyrights, patents, trade secrets and the terms of license agreements is prohibited by law in most circumstances. Even when software is not so protected, such violations are contrary to professional behavior. Copies of software should be made only with proper authorization. Unauthorized duplication of materials must not be condoned.

\vspace{.1in}\noindent\textbf{1.6 Give proper credit for intellectual property.}

Computing professionals are obligated to protect the integrity of intellectual property. Specifically, one must not take credit for other's ideas or work, even in cases where the work has not been explicitly protected by copyright, patent, etc.

\vspace{.1in}\noindent\textbf{1.7 Respect the privacy of others.}

Computing and communication technology enables the collection and exchange of personal information on a scale unprecedented in the history of civilization. Thus there is increased potential for violating the privacy of individuals and groups. It is the responsibility of professionals to maintain the privacy and integrity of data describing individuals. This includes taking precautions to ensure the accuracy of data, as well as protecting it from unauthorized access or accidental disclosure to inappropriate individuals. Furthermore, procedures must be established to allow individuals to review their records and correct inaccuracies.

This imperative implies that only the necessary amount of personal information be collected in a system, that retention and disposal periods for that information be clearly defined and enforced, and that personal information gathered for a specific purpose not be used for other purposes without consent of the individual(s). These principles apply to electronic communications, including electronic mail, and prohibit procedures that capture or monitor electronic user data, including messages,without the permission of users or bona fide authorization related to system operation and maintenance. User data observed during the normal duties of system operation and maintenance must be treated with strictest confidentiality, except in cases where it is evidence for the violation of law, organizational regulations, or this Code. In these cases, the nature or contents of that information must be disclosed only to proper authorities.

\vspace{.1in}\noindent\textbf{1.8 Honor confidentiality.}

The principle of honesty extends to issues of confidentiality of information whenever one has made an explicit promise to honor confidentiality or, implicitly, when private information not directly related to the performance of one's duties becomes available. The ethical concern is to respect all obligations of confidentiality to employers, clients, and users unless discharged from such obligations by requirements of the law or other principles of this Code.




\vspace{.2in}\noindent{\large 2. MORE SPECIFIC PROFESSIONAL RESPONSIBILITIES.}\vspace{.1in}

As an ACM computing professional I will ....

\vspace{.1in}\noindent\textbf{2.1 Strive to achieve the highest quality, effectiveness and dignity in both the process and products of professional work.}

Excellence is perhaps the most important obligation of a professional. The computing professional must strive to achieve quality and to be cognizant of the serious negative consequences that may result from poor quality in a system.

\vspace{.1in}\noindent\textbf{2.2 Acquire and maintain professional competence.}

Excellence depends on individuals who take responsibility for acquiring and maintaining professional competence. A professional must participate in setting standards for appropriate levels of competence, and strive to achieve those standards. Upgrading technical knowledge and competence can be achieved in several ways:doing independent study; attending seminars, conferences, or courses; and being involved in professional organizations.

\vspace{.1in}\noindent\textbf{2.3 Know and respect existing laws pertaining to professional work.}

ACM members must obey existing local, state,province, national, and international laws unless there is a compelling ethical basis not to do so. Policies and procedures of the organizations in which one participates must also be obeyed. But compliance must be balanced with the recognition that sometimes existing laws and rules may be immoral or inappropriate and, therefore, must be challenged. Violation of a law or regulation may be ethical when that law or rule has inadequate moral basis or when it conflicts with another law judged to be more important. If one decides to violate a law or rule because it is viewed as unethical, or for any other reason, one must fully accept responsibility for one's actions and for the consequences.

\vspace{.1in}\noindent\textbf{2.4 Accept and provide appropriate professional review.}

Quality professional work, especially in the computing profession, depends on professional reviewing and critiquing. Whenever appropriate, individual members should seek and utilize peer review as well as provide critical review of the work of others.

\vspace{.1in}\noindent\textbf{2.5 Give comprehensive and thorough evaluations of computer systems and their impacts, including analysis of possible risks.}

Computer professionals must strive to be perceptive, thorough, and objective when evaluating, recommending, and presenting system descriptions and alternatives. Computer professionals are in a position of special trust, and therefore have a special responsibility to provide objective, credible evaluations to employers, clients, users, and the public. When providing evaluations the professional must also identify any relevant conflicts of interest, as stated in imperative 1.3.

As noted in the discussion of principle 1.2 on avoiding harm, any signs of danger from systems must be reported to those who have opportunity and/or responsibility to resolve them. See the guidelines for imperative 1.2 for more details concerning harm,including the reporting of professional violations.

\vspace{.1in}\noindent\textbf{2.6 Honor contracts, agreements, and assigned responsibilities.}

Honoring one's commitments is a matter of integrity and honesty. For the computer professional this includes ensuring that system elements perform as intended. Also, when one contracts for work with another party, one has an obligation to keep that party properly informed about progress toward completing that work.

A computing professional has a responsibility to request a change in any assignment that he or she feels cannot be completed as defined. Only after serious consideration and with full disclosure of risks and concerns to the employer or client, should one accept the assignment. The major underlying principle here is the obligation to accept personal accountability for professional work. On some occasions other ethical principles may take greater priority.

A judgment that a specific assignment should not be performed may not be accepted. Having clearly identified one's concerns and reasons for that judgment, but failing to procure a change in that assignment, one may yet be obligated, by contract or by law, to proceed as directed. The computing professional's ethical judgment should be the final guide in deciding whether or not to proceed. Regardless of the decision, one must accept the responsibility for the consequences.

However, performing assignments ``against one's own judgment'' does not relieve the professional of responsibility for any negative consequences.

\vspace{.1in}\noindent\textbf{2.7 Improve public understanding of computing and its consequences.}

Computing professionals have a responsibility to share technical knowledge with the public by encouraging understanding of computing, including the impacts of computer systems and their limitations. This imperative implies an obligation to counter any false views related to computing.

\vspace{.1in}\noindent\textbf{2.8 Access computing and communication resources only when authorized to do so.}

Theft or destruction of tangible and electronic property is prohibited by imperative 1.2 - ``Avoid harm to others.'' Trespassing and unauthorized use of a computer or communication system is addressed by this imperative. Trespassing includes accessing communication networks and computer systems, or accounts and/or files associated with those systems, without explicit authorization to do so. Individuals and organizations have the right to restrict access to their systems so long as they do not violate the discrimination principle (see 1.4). No one should enter or use another's computer system, software, or data files without permission. One must always have appropriate approval before using system resources, including communication ports, file space, other system peripherals, and computer time.




\vspace{.2in}\noindent{\large 3. ORGANIZATIONAL LEADERSHIP IMPERATIVES.}\vspace{.1in}

As an ACM member and an organizational leader, I will ....

\begin{quote}
BACKGROUND NOTE: \emph{This section draws extensively from the draft IFIP Code of Ethics,especially its sections on organizational ethics and international concerns. The ethical obligations of organizations tend to be neglected in most codes of professional conduct, perhaps because these codes are written from the perspective of the individual member. This dilemma is addressed by stating these imperatives from the perspective of the organizational leader. In this context ``leader'' is viewed as any organizational member who has leadership or educational responsibilities. These imperatives generally may apply to organizations as well as their leaders. In this context ``organizations'' are corporations, government agencies,and other ``employers,'' as well as volunteer professional organizations.}
\end{quote}

\vspace{.1in}\noindent\textbf{3.1 Articulate social responsibilities of members of an organizational unit and encourage full acceptance of those responsibilities.}

Because organizations of all kinds have impacts on the public, they must accept responsibilities to society. Organizational procedures and attitudes oriented toward quality and the welfare of society will reduce harm to members of the public, thereby serving public interest and fulfilling social responsibility. Therefore,organizational leaders must encourage full participation in meeting social responsibilities as well as quality performance.

\vspace{.1in}\noindent\textbf{3.2 Manage personnel and resources to design and build information systems that enhance the quality of working life.}

Organizational leaders are responsible for ensuring that computer systems enhance, not degrade, the quality of working life. When implementing a computer system, organizations must consider the personal and professional development, physical safety, and human dignity of all workers. Appropriate human-computer ergonomic standards should be considered in system design and in the workplace.

\vspace{.1in}\noindent\textbf{3.3 Acknowledge and support proper and authorized uses of an organization's computing and communication resources.}

Because computer systems can become tools to harm as well as to benefit an organization, the leadership has the responsibility to clearly define appropriate and inappropriate uses of organizational computing resources. While the number and scope of such rules should be minimal, they should be fully enforced when established.

\vspace{.1in}\noindent\textbf{3.4 Ensure that users and those who will be affected by a system have their needs clearly articulated during the assessment and design of requirements; later the system must be validated to meet requirements.}

Current system users, potential users and other persons whose lives may be affected by a system must have their needs assessed and incorporated in the statement of requirements. System validation should ensure compliance with those requirements.

\vspace{.1in}\noindent\textbf{3.5 Articulate and support policies that protect the dignity of users and others affected by a computing system.}

Designing or implementing systems that deliberately or inadvertently demean individuals or groups is ethically unacceptable. Computer professionals who are in decision making positions should verify that systems are designed and implemented to protect personal privacy and enhance personal dignity.

\vspace{.1in}\noindent\textbf{3.6 Create opportunities for members of the organization to learn the principles and limitations of computer systems.}

This complements the imperative on public understanding (2.7). Educational opportunities are essential to facilitate optimal participation of all organizational members. Opportunities must be available to all members to help them improve their knowledge and skills in computing, including courses that familiarize them with the consequences and limitations of particular types of systems.In particular, professionals must be made aware of the dangers of building systems around oversimplified models, the improbability of anticipating and designing for every possible operating condition, and other issues related to the complexity of this profession.




\vspace{.2in}\noindent{\large 4. COMPLIANCE WITH THE CODE.}\vspace{.1in}

As an ACM member I will ....

\vspace{.1in}\noindent\textbf{4.1 Uphold and promote the principles of this Code.}

The future of the computing profession depends on both technical and ethical excellence. Not only is it important for ACM computing professionals to adhere to the principles expressed in this Code, each member should encourage and support adherence by other members.

\vspace{.1in}\noindent\textbf{4.2 Treat violations of this code as inconsistent with membership in the ACM.}

Adherence of professionals to a code of ethics is largely a voluntary matter. However, if a member does not follow this code by engaging in gross misconduct, membership in ACM may be terminated.




\vspace{.2in}This Code and the supplemental Guidelines were developed by the Task Force for the Revision of the ACM Code of Ethics and Professional Conduct: Ronald E. Anderson, Chair, Gerald Engel, Donald Gotterbarn, Grace C. Hertlein, Alex Hoffman, Bruce Jawer, Deborah G. Johnson, Doris K. Lidtke, Joyce Currie Little, Dianne Martin, Donn B. Parker, Judith A. Perrolle, and Richard S. Rosenberg. The Task Force was organized by ACM/SIGCAS and funding was provided by the ACM SIG Discretionary Fund. This Code and the supplemental Guidelines were adopted by the ACM Council on October 16, 1992.


This Code may be published without permission as long as it is not changed in any way and it carries the copyright notice. Copyright \copyright 1997, Association for Computing Machinery, Inc.


