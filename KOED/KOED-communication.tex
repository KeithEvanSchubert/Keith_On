\chapter{Communication}

\section{Weekly Status Report}
\subsection{Team Leaders}
Weekly status reports have four basic sections: scheduled tasks, accomplished tasks, meetings, project status, goals/assignments for next period.  Much of this revolves around the schedule/Gantt chart.  The entire report should be 1-2 pages in length.  Shorter is better if it covers the points.

In the first section, you should give a brief statement of what was the goals for the week, and how they align with or derive from the schedule/Gantt chart.  The person or people assigned should be included.  Brief means enough detail is given to understand what was to happen and how it fits in the larger program.  Do not give so much that it takes to much to read (a paragraph or two is normal), or so little that the reader can't judge what was done and how it will impact things.  Note that the schedule can get modifications and tasks can get added, deleted, or amended, so it is important to cite the date or revision of the schedule/Gantt chart.

In the next section you should cover what tasks were actually accomplished, and if this met the goals.  Be sure to cite key players, as this is how people get recognition and thus raises and promotions.  It is the goal of a manager to help your employees improve and get rewards.  These should also be brief (read: get to the point), as whoever is reading this will probably be reading several of these.  People need to know precisely what has been accomplished, not a narrative on the accomplishment.  You can give a short statement on why something was difficult if it is only occasionally done.

The meetings section should state the attendance, reason (main agenda items) and results or conclusions.

Next, the weekly status report should show the progress incrementally and cumulatively, i.e. what progress did you make this week as a percent/fraction of the weeks goals, and what is your total progress as a percent/fraction of the total project to date.  This should sum up in a numerical way what has been accomplished this week and where the entire project stands.  Including an updated Gantt chart is optional, but can be very illuminating.

Finally, the status report should have the assignments or goals for each member of the team by name, giving the task, numerical estimate of completion, and task title.

\subsection{Team Members}

Each team member needs to send their status reports so that the team leader can send a team status report to their superiors, such as project engineers in industry, or professors and company liaisons in college design programs.  Team members have three things they have to report
\begin{enumerate}
\item Goals for the week.  These are usually assigned by the team leader, or at least with the team leader's cognizance, but you are trying to make their life easy and show them that you understood your tasks.
\item  Accomplishments for the week.  This could be a percentage of work done for a task that is larger or the whole task if you finished it.  Let them know if it is ahead or behind schedule and any positive notes.  Be brief (not rambling) but don't cut out important facts.  This is a bit of an art, but basically you are trying to let them know the great things you did, and what is going on, but not take so long they don't want to read it.  As a good guide try two sentences per task, the first stating what amount is done, the second stating descriptively what was accomplished.
\item Assignments for the next week.
\end{enumerate}
I have included a sample of a good status report in figure~\ref{letter-member-status-report}.  Note the problem/alteration/roadblock was put in italics to emphasize it.

\begin{figure*}[p]
\begin{center}
\caption{Sample Team Member Status Report.}\label{letter-member-status-report}
\vspace{.1in}
\begin{minipage}{4in}
\begin{flushright}
Keith Evan Schubert\\
Associate Engineer\\
Widget Project
\end{flushright}

\textbf{Goals:}

All tasks are from Widget Project Gantt chart, 4-17-08.
\begin{enumerate}
\item Finish task 3, model system dynamics widget driven by oscillator circuit.
\item Do 20\% of task 4, analyze stability of widget.
\item Do task 8, select widget antenna.
\end{enumerate}


\textbf{Accomplishments:}

All tasks are from Widget Project Gantt chart, 4-17-08.
\begin{enumerate}
\item Task 3, model system dynamics widget driven by oscillator circuit is 100\% complete.  The modeling took one day longer than anticipated because nonlinearities in the oscillator proved to be too large to ignore, necessitating a more complicated model.  The non-linearity enters only as an input, but can be handled by augmenting the states.  The results are in technical report TR-00743-23: Widget Dynamics.
\item Task 4, analyze stability of widget is 10\% done, as the non-linear states, which handle the inputs, have been analyzed.  The non-linear states form a stable, passive manifold that is bounded-input, bounded-output stable.  This will still create complexity in calculating the closed loop control law, which might require an extra week to examine robust control methods if the non-linear control proves infeasible.
\item Task 8, select widget antenna is 100\% complete.  The Antenna Industries, 9'' whip antenna model \# 91-471128-678 rev B was the best price-performance antenna and is in the middle of its product life.
\end{enumerate}

\textbf{Assignments:}

All tasks are from Widget Project Gantt chart, 4-17-08.
\begin{enumerate}
\item Continue task 4, analyze stability of widget.  Achievement of 70\% completion was the original goal, but \emph{given the alterations in the problem complexity an adjustment to 50\% completion is suggested}, which would entail completion up to nonlinear analysis, but not a generation of a candidate control law.
\end{enumerate}

\end{minipage}
\end{center}
\end{figure*}

\section{Log Book}
Log books are a necessary legal documentation of work that are crucial for a variety of cases including patents.  To be useful in court the pages must not be removed or inserted (thus no binders).  Blank pages cannot be left, and all pages must be used in sequence.  You cannot leave any large blank space, as this would allow for future tampering.  Items cannot be ``whited out'' or scribbled over.  Any corrections must be done by a single, thin line through the error.

Each entry should have a descriptive heading with the date, and should be legible and understandable.  The log book must contain all work done, including: meeting notes, sketches, ideas, concepts, calculations, phone call notes, research notes, test data, etc.  Logbooks are primarily a daily record of what you thought and did, what your goals were, and how you were progressing.  It ought to be more than a diary of general activity.  Make sure you label sketches/drawings.  Include relevant dimensions or scale. Include assembly diagram for context.  Note block diagrams are particularly useful.

Think of future users, and evaluate your logbook from this perspective.  Make sure you add enough detail to the notes of design specifications and estimates to make them reconstructible later.  If you use an outside formula or result, cite the source so others can find it.  Make sure you include enough detail to facilitate future users reconstruction of your efforts.  Note even your dead ends and why you abandoned them, as this information is of particular assistance.  Relevant loose materials can be attached to a logbook if small enough, or they can be cited in a clear distinguishable way, and stored in a different location (such as a notebook or filing cabinet).  Identify contacts and communications by name, company, location/number, and date for future reference.

\section{Presentations}

A good talk starts with lots of preparation.  You want it to look like it is no work, so you have to do lots of work beforehand so it seems effortless.

You have to pick a theme and one key point for every 10-20 minutes.  Everything else must support this.  Keep the goal and theme in mind.  Make slides that support you, not that steal (or ruin) the show.  Find ways to illustrate your talk in several different engagement styles: pictures, stories, thought experiments (gedanken experiments), etc.

Make sure your slides and so on are readable by everyone in the room.

Keep your slides simple- no fancy transitions and effects- as they will not carry over from one machine to another.  Honestly talks are either glitzy or substantive (or neither) - be substantive.  Substantive talks can be cleanly elegant, but I have not seen one that is glitzy.

Don't put lots of text on the page.  You will tend to read it, and so will your audience.  You will be boring and they won't be paying attention to anything you say.  Use text sparingly in phrases to highlight ideas and keep people on track.

Avoid notes, try to know your speech.  Memorization is a plausible alternative, but knowing it means you have moved beyond mere memorization to understanding and an ability to paraphrase it.  This will heighten respect and confidence from your audience.

Keep track of the time.  No one wants to be bored by a long talk.  Stay on track, but look comfortable on doing this.  One technique I use is having optional stories or info for slides so I can add or subtract and my audience is none the wiser.  I can then use these thoughts in conversations after the presentation is over, if I didn't use them.

The most important slide is the first one, because it has your name.  Don't skip over it or move on too fast.  I suggest you give a short bio on yourself (1 min or so).  One of your main goals is to let people know who you are - it introduces the rest of your work and provides you future opportunities.  Don't sound like an egoist, make it a casual introduction so they get to know you.  It is a nice icebreaker, and separates experienced speakers from novices.

The next most important section is the intro, followed by the conclusion.  Again, most people skip this.  Don't. It is your chance to frame how they look at the work.  Why do they care?  How should the think about this?  Introduce your key points- they will need to hear them at least three times to remember them.  Make both clear.  The intro allows you to set them thinking in whatever way you want. The conclusion is your last chance to clean up their understanding and thoughts, don't miss it.

The first thing people will see is how you are dressed and groomed.  To ensure a good talk make sure you dress appropriately and are well groomed.  It sounds shallow; I know.  This is the first thing people see and it will affect how they think of you\footnote{I am not arguing if this is right or wrong, but rather I am saying you are foolish not to take such a simple step to improve how people view you.}.  Notice that I said you should dress appropriately, not dress as fancy as you can.  You want to look professional, and basically on par to slightly better than your audience.  The earlier you are in your career, the better you have to dress to get respect.  You should never look shabby.  The one exception to the on par rule is if you ever get to the point you are the undisputed expert in a technical audience, or the owner of the company, etc., in which case you can dress nice but causal and it comes off showing your special status.  Don't try this if you aren't in the position to pull it off, but if you can, it is a sign of prestige.

Be calm.  No really, be calm.  This is the single biggest failure: looking nervous, talking fast, skipping things, etc.  Know your stuff, then speak clearly, concisely, and authoritatively.  Speak at a slightly slower than normal pace, and put pauses for emphasis.  Speed and pauses are like font size and white space: a paper that has tiny fonts and no blank space is ugly and unreadable.  Similarly a rushed talk with no pauses is displeasing to the ears and will not get its point across.

Don't name drop without an interesting story that relates to the subject at hand, but then do so.

Use consistent terms unless you have a really good reason, like trying to show that terminology is arbitrary, and even then do it sparingly.  Inconsistent terminology confuses people.  There is almost always an extreme in all directions, so here is the flip: don't make things needlessly consistent.  A little variety opens things up to creative ways to examine things and prevents disciplinary myopia (and sometimes even near magical adherence to arbitrary conventions).

If you are scared then pick a few friendly faces out in the audience and move between watching them.

If you are doing a group presentation you should prepare more and practice so the talk doesn't look stiff.  Give nice transitions - introduce the next speaker and their area.  Shake hands, give a pat on the arm, share a warm smile, or something else that shows you are all friends and colleagues.  It will build confidence in your work and make others interested in knowing you.  People like friendly people. 

\section{Reports}

Engineering reports and technical writing in general are unique beasts.  Many jokes exist about how bad engineers write, some of which has to do with our emphasis on technical correctness over style, but some has to do with what engineering reports are.  We are not trying to be pretty, flashy, or entertaining in a report, we are trying to say accurate statements both concisely and atomically\footnote{I am stealing the term atomic from computer science and adapting it to writing.}.  Concisely means to say something correctly in the most efficient way (read least number of words).  We value a short, information-packed statement that is still understandable.  By atomically, I mean each paragraph must be able to be read independent of the rest of the report, because that is what will happen. People will just look up the sub-section that has a title that makes them think it has the information they want, read that or parts of it till they find the information to use.  You need to write with this in mind.  Much of this is contained in the following quote (it only misses the random access part).

\begin{quote}
How to make engineers write concisely with sentences? By combining journalism with the technical report format. In a newspaper article, the paragraphs are ordered by importance, so that the reader can stop reading the article at whatever point they lose interest, knowing that the part they have read was more important than the part left unread.

State your message in one sentence. That is your title. Write one paragraph justifying the message. That is your abstract. Circle each phrase in the abstract that needs clarification or more context. Write a paragraph or two for each such phrase. That is the body of your report. Identify each sentence in the body that needs clarification and write a paragraph or two in the appendix. Include your contact information for readers who require further detail.

\begin{flushright}-- William A. Wood (email), September 8, 2005\end{flushright}
\end{quote}
