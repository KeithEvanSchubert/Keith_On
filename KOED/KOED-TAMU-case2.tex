The following is from Texas A\&M University's Engineering Ethics Cases with Numerical Problems.  The case study is from an NSF \& Bovay Fund Sponsored Workshop on August 14-18, 1995, and thus is freely available for use in non-profit educational texts like this one.  This one is Electrical Engineering Case 2: Allowing Defective Chips to go to Market, by Jeremy Hanzlik (JHanzlik@tamu.edu).  

\subsection{Narrative}

Shane, a production line engineer, is checking every chip for quality control (QC). His workers are finding defective chips once every 150 chips. The defective chips must either be sent back for repair or thrown away. His manager, Rob, has mandated that all defective chips be discarded. Rob walks over to Shane's line and has a conversation in which he says, ``Why some other lines sink more dollars into a failed chip I can't understand. We only make 25 cents off of each chip anyway! Spending an additional \$2.00 per chip to repair it only means more money down the drain. Shane, in our line of work we can't afford to flush money down the drain.''

The following afternoon Rob informs Shane that his line is discarding too many chips, ``One chip every hundred and fifty is unacceptable! This is becoming a substantial cost to the company. I believe that it would be more beneficial to allow all the chips to go out the door without checking.''

Shane asks, ``What about the defective chips? Won't customers complain?''

Rob replies, ``Yeah, yeah, but that's not your problem. The company has a return department that will replace them as customers complain.'' Rob further estimates that allowing defective chips on the market will yield a \$416,000 profit for the company.

\subsubsection{Facts:}
\begin{itemize}
\item The line produces 100,000 chips per year.
\item Every chip is purchased.
\item Chips cost \$9.00 to produce.
\item Chip testing runs \$4.00 per chip.
\item Chip repair (manpower and material) is \$2.00/chip.
\item This repair cost includes re-testing.
\item Profit per chip is \$0.25 after testing.
\item There are fifteen full time employees and two part-time employees working under Shane.
\item Shane's manager has been with the company for about 7 years.
\item Shane has been working Rob's management for several years and has had relatively good relations with him.
\end{itemize}
\subsubsection{Additional information regarding Shane's line:}

The engineer's line performs the final inspection between the bond wires, which attach the chips to prongs and spot plates, (the prongs protrude from the final product). The chips are then encased in molding compound for final packaging. You may assume that all defects can be traced to faulty bond wire attachment and not to the chip itself. This is because the chips have been previously tested, before the bond wires were attached.

\subsection{Numerical and/or Design Problem(s)}
\begin{enumerate}
\item  If Xanthum, Inc. orders 15,000 chips from Shane's production line, what percent of the chips may fail?
\item Do you believe this is an acceptable failure rate: from the perspective of Xanthum? from the perspective of the manufacturer? Why (not)?
\item If Shane's line produces 100,000 chips per year how much will it cost to:
  \begin{enumerate}
  \item Test and repair any defective chip?
  \item Test all chips and discard the defective chips?
  \item Test no chips and replace defective customers chips on an as-returned basis?
  \end{enumerate}
\item Is Rob's estimate reasonable? What about his assertion that it is cheaper to ship the defective chips?
\end{enumerate}

\subsection{Questions on Ethics and Professionalism}
\begin{enumerate}
\item What issues are involved in following Rob's recommendation?
\item Is it acceptable to follow Rob's suggested course of action, based on your calculations above?
\item How should the engineer, Shane, present his case to Rob, his superior, if he has a differing opinion?
\end{enumerate}

\subsection{Additional Scenario}
To be introduced in lecture after the students have completed the work above on their own.

The chips ordered by Xanthum, Inc. are to be placed in aircraft navigation units, and Shane's boss still believes that the failures are inconsequential. Rob claims, ``They always have backup navigation systems anyway. Besides, they fail less than one percent of the time! You should know that. Just calculate the percent of chips that will fail.''
\begin{enumerate}
\item  What flaws can be found (based on your previous calculations and present observations) with Rob's argument? Hint: look at the logic used in Rob's statement.
\item Perform a (utilitarian) cost/benefit analysis based on the above data. How much will it cost the company in litigation, etc? Make any necessary assumptions such as dollar values for your calculations, as long as your assumptions are not in direct conflict with the stated facts above.
\item How does this scenario influence your response to question 2 from part III?
\item How can the engineer, Shane, constructively present an argument against his superior's opinion?
\end{enumerate}

\subsection{Solutions for Numerical Problems}

Part I
\begin{enumerate}
\item 1/150 chips will not work so: 1/150 = 0.67\% may not work.
\item This question in neither right nor wrong. It is meant to provoke thought for later ethical issues.
\item
   \begin{enumerate}
   \item test all: 100,000 x \$4.00 = \$400,000 (testing cost)

        repair cost: 667 bad chips x \$2.00 repair cost = \$1,334

        profit on repaired chips: 667 bad chips x -\$1.75 profit loss = -\$1,167.25

        profit on good chips: (100,000 - 667) x \$0.25 = \$24,833.25

        net profit: \$24,833.25 - \$1,167.25 = \$23,666

   \item test all: 100,000 chips x \$4.00 profit = \$400,000 (testing cost)

       discard cost: \$0.00

       profit on discarded chips: 667 bad chips x (\$9.00 + \$4.00 - \$0.25) = -\$8,504.25

       profit on good chips: (100,000 � 667) x \$0.25 = \$24,833.25

       net profit: \$24,833.25 - \$8,504.25 = \$16,329

  \item There are two polar scenarios based on different student assumptions (and equally correct answers based upon differing assumptions as to the number of chips returned):

``Best Case'' - no returns of defective chips

    Not test any chips: 100,000 x \$4.25 = \$425,000

``Worst Case'' - all defective chips are returned

    To find the number of chips which actually generate profit:

       100,000 - 667 returns = 99,333 - 667 replacements = 98,666 chips to generate dollars

       Profit from "satisfied" customers: 98,666 x \$4.25 = \$419,330.5

       Original profit on the 667 returned chips: 667 x \$4.25 = \$2,834.75

       Profit from replacement chips: 667 x -\$9.00 = -\$6,003

       Net profit: \$419,330.5 + \$2,834.75 - \$6,003 = \$416,162.25
   \end{enumerate}
\item Here the student should see that the answer is clearly NO. The manager, Rob, has based his calculation on the improbable assumption that no chips will be returned. Because of this, his estimate for profit is too high. The manager is also wrong to advocate discarding chips. It is more profitable if the chips can be repaired, and therefore the line loses potential dollars if the chips are discarded.
\end{enumerate}

\subsection{Solutions to Ethics and Professionalism Questions}
\begin{enumerate}
\item Issues involved in the narrative above include deceiving the public, since the public is sold chips with no precautions taken to ensure the correct assembly of the microchips. A secondary issue involved here could be public safety: what if the part is used to build a critical device such as a navigation computer for missiles or airplanes?
\item The answer depends on the assumptions made by the student. If the student conceives this issue as a safety issue, then it would clearly not be acceptable to follow the superior's suggested course of action. The answer also depends on whether the student perceives the decision as a proper decision for managers or a proper one for engineers to make. Is this a safety issue? If the welfare of the public will not be compromised then the decision is an economic one, which falls, in the domain of management. If however, the assumption is made that the chips involve public safety, engineers should make the decision, because engineering ethical standards require that public safety be protected.
\item As an engineer, Shane has an obligation to be a loyal employee, but also to protect public safety and to uphold the standards of quality. Shane needs to try to find a creative middle way in which all of these competing obligations can be satisfied. Shane may be able to satisfy these demands by making an argument of the following type. First, the manager's estimates of the profits resulting from shipping out the defective chips were for a best-case scenario. Therefore, it can be asserted that actual profits should be less; especially if there is a return cost involved with the defective chips in terms of personnel, papertrails, and loss of reputation. Second, by repairing the chips, the loss is significantly minimized in comparison to discarding chips: the line is still profitable and the company's reputation for quality could help increase the overall corporate value. So a creative middle way would be for the engineer to suggest repairing all chips, and present his calculations to his boss to justify this suggestion. This solves the problem of the lines losing money, turns a reasonable profit (although nowhere near what could be made via the worst case scenario for the first year), and establishes a tradition of quality within the company.
\end{enumerate}

\subsection{Reference}

Jaegar, Richard C., Introduction to Microelectronic Fabrication, Addison-Wesley Publishing Co., pp. 153-172 (Feb, 1991).
