\chapter{Differential Equations}\label{c-DifEq}

Most practical problems will be described by a differential equation.
We will not in general know the form of the solution, but we usually
can find how they change with respect to each other.  From this basis
we would like to be able to find the actual solution.

\section{General Introduction}

Consider the general differential equation
\beq
\dot y(x)=f(x,y(x)). \label{DifEq-e1}
\eeq
Using basic calculus we see that the solution is given by
\beqn
y(x)=\int f(x,y(x))dx+c.
\eeqn
Often this is not very useful in solving equations however.  The way
to get practical solutions for this problem is covered in
differential equations, so I will just mention a few.

\subsection{Existence}
The problem~\ref{DifEq-e1} has a solution on an interval $x_{0}\leq x\leq
\min(b_{x},c)$ and $y_{0}\leq y\leq b_{y}$ if the function $f$ is continuous
on the interval for $c=\frac{\|y-y_{0}\|}{\max_{x,y}\left(\frac{\partial f(x,y)}
{\partial y}\right)}$.
\bfig
{\tt    \setlength{\unitlength}{0.92pt}
\begin{picture}(202,139)
\thinlines    \put(152,21){$c$}
              \put(155,104){\line(0,-1){73}}
              \put(40,31){\framebox(134,73){}}
              \put(40,31){\dashbox{5}(95,55){}}
              \put(135,85){\circle*{4}}
              \put(170,21){$b_x$}
              \put(28,101){$b_y$}
              \put(132,21){$x$}
              \put(33,82){$y$}
              \put(10,29){$y_0$}
              \put(38,12){$x_0$}
              \put(23,31){\vector(1,0){169}}
              \put(40,23){\vector(0,1){106}}
              \put(52,50){$f(x,y(x))$}
\end{picture}}
\efig{Existence requirements}{DifEq-f1}


\section{Euler's Method}

Euler was without a doubt one of the greatest mathematical minds to have ever lived.  He did work in almost every area you can imagine including numerical methods.  His method is simple and easy for people to use.  Unfortunately, as the old numerics folk theorem goes, ``that which is good for a person is bad for a computer and vise versa'' Euler's method is unstable numerically\footnote{The forward method actually, the backward method of Euler is stable though still far from a good method.}.  To introduce the method, say we knew a point, say $(x_0,y(x_0))$, we could denote our first point by the y-coordinate.
\beqn
y_{0} & = & y(x_0)
\eeqn
Now say we wanted to find the value of the function a short distance away, say at $x_0+h$.  It would be reasonable to do a first order taylor approximation.  Noting that $\dot y = f(x,y)$ allows us to write this in simple terms.
\beqn
y_{1} & = & y(x_{0}+h) \\
      & = & y_{0}+hf(x_{0},y_{0})
\eeqn
We could extend this process to find
\beqn
y_{n} & = & y(x_{0}+nh) \\
      & = & y_{n-1}+hf(x_{n-1},y_{n-1}) \\
      & = & y_{n-2}+hf(x_{n-2},y_{n-2})+hf(x_{n-2}+h,y_{n-2}+hf(x_{n-2},y_{n-2}))
\eeqn
Note that the third (last) term involves estimating based on an estimate.  It is here that the problem comes because the errors can build.  It is easier to show the problem by taking the \textit{Z}-Transform of the second line
\beqn
y_{n} & = & y_{n-1}+hf(x_{n-1},y_{n-1}) \\
Y & = & z^{-1}Y+hF(z^{-1}X,z^{-1}Y)
\eeqn
Assuming we are dealing with linear functions (a common assumption) we can pull the $z^{-1}$ out.
Error is given by
\beqn
Y(x)-y_{h}(x) & = & hD(x)+{\cal O}(h^{2}) \\
D'(x) & = & g(x)D(x)+\frac{1}{2}Y''(x), \qquad D(x_{o})=0 \\
g(x) & = & \left.\frac{\partial f(x,z)}{\partial z}\right|_{z=Y(x)}
\eeqn


\section{Runge-Kutta}

While Euler's method is nice and simple, it is far from the best.
Higher order Taylor methods can be derived but these require
evaluating multiple derivatives.  Even Richardson's extrapolation has
limits on its abilities.

Consider the Taylor series of $y$.
\beqn
y(x+h) & = & y(x)+hy'(x)+\frac{h^{2}}{2}y''(x)+\ldots \\
 & = & y(x)+hf(x,y)+\frac{h^{2}}{2}f'(x,y)+\ldots \\
 & = & y(x)+hf(x,y)+\frac{h^{2}}{2}\left(f_{x}(x,y)+f_{y}(x,y)f(x,y)\right)
        +\ldots \\
 & = & y(x)+hf(x,y)+\frac{h^{2}}{2}\left(f_{x}(x,y)+f_{y}(x,y)f(x,y)\right)
        +\ldots \\
 & = & y(x)+hf(x,y)+ah\left(\frac{h}{2a}f_{x}(x,y)+
           \frac{h}{2a}f(x,y)f_{y}(x,y)\right)
        +\ldots
\eeqn
Now we consider the Taylor series in two variables of $f(x,y)$.
\beqn
f(x+h,y+k) & = & \sum_{n=0}^{\infty}\frac{1}{n!}
 \left[h\frac{\partial}{\partial x}+k\frac{\partial}{\partial y}\right]^{n}f(x,y) \\
 & = & f(x,y)
 +\left[h\frac{\partial}{\partial x}+k\frac{\partial}{\partial y}\right]f(x,y)
 + \ldots \\
 & = & f(x,y)
 +hf_{x}(x,y)+kf_{y}(x,y)
 + \ldots
\eeqn
Rearranging we find
\beqn
f(x+h,y+k) -f(x,y)
  & = &
hf_{x}(x,y)+kf_{y}(x,y)
 + \ldots .
\eeqn
We use this in our Taylor series of $y$.
\beqn
y(x+h)
  & = & y(x)+hf(x,y)+ah\left(\frac{h}{2a}f_{x}(x,y)+\frac{h}{2a}f(x,y)f_{y}(x,y)\right)
        +\ldots \\
  & = & y(x)+hf(x,y)+h\left(af(x+\frac{h}{2a},y+\frac{h}{2a}f(x,y)) -af(x,y)\right)
        +\ldots \\
  & = & y(x)+h(1-a)f(x,y)+ahf(x+\frac{h}{2a},y+\frac{h}{2a}f(x,y))
        +\ldots \\
  & = & y(x)+h(1-a)f(x,y)+ahf(x+b,y+bf(x,y))
        +\ldots \\
& & \qquad b=\frac{h}{2a} \qquad 0\leq a \leq 1
\eeqn
We have defined this in order to take advantage of the many varieties
of second order R-K.  The most common are: Midpoint (a=1), Modified
Euler (a=1/2), and Heun's method (a=3/4).

You can also do R-K for higher orders.  The most common is fourth
order.  The algebra is tedious so I will just present the result.
\beqn
y(x+h) = y(x)+\frac{1}{6}\left(F_{1}+2F_{2}+2F_{3}+F_{4}\right)
\eeqn
with
\beqn
F_{1} & = & hf(x,y) \\
F_{2} & = & hf(x+\frac{h}{2},y+\frac{F_{1}}{2}) \\
F_{3} & = & hf(x+\frac{h}{2},y+\frac{F_{2}}{2}) \\
F_{4} & = & hf(x+h,y+F_{3})
\eeqn

\section{Fehlberg's Method}

We have seen that the local error (error in one step due mostly to
truncation) is one order of magnitude better than the global error,
in general.  Often we use Richardson's Error formula to find an
estimate of the local error ($T_{n}$) so we can adjust the step size
to keep things nice.  For instance Richardson's Error for Euler's
method gives us
\beqn
Y(x)-y_{h}(x) & \approx & hD(x) \\
Y(x)-y_{2h}(x) & \approx & 2hD(x) \\
Y(x)-y_{2h}(x) & \approx & 2(Y(x)-y_{h}(x)) \\
Y(x) & \approx & 2y_{h}(x)-y_{2h}(x) \\
Y(x)-y_{h}(x) & \approx & (2y_{h}(x)-y_{2h}(x))-y_{h}(x) \\
 & \approx & y_{h}(x)-y_{2h}(x).
\eeqn
This gives us a reasonable extrapolation formula, and estimate of the
error.  Another way to estimate the error would be to look at two
estimates from different order methods.  For instance you could do a
fourth and a fifth order R-K estimate at each step and use the
difference to bound the error.  If the error at any step became to
large then you would decrease the step size and try again.  This idea
is Fehlberg's method, and it is the basis of most modern ode solvers.
This is why Matlab has ode23 and ode45.

\section{Adams-Bashforth}

Up till now we have been looking at solving the differential equation
directly.  We could however just integrate both sides.
\beqn
y' & = & f(x,y) \\
\int_{x_{n}}^{x_{n+1}}y'dx & = & \int_{x_{n}}^{x_{n+1}}f(x,y)dx \\
y(x_{n+1})-y(x_{n}) & = & \int_{x_{n}}^{x_{n+1}}f(x,y)dx \\
y(x_{n+1}) & = & y(x_{n}) + \int_{x_{n}}^{x_{n+1}}f(x,y)dx
\eeqn
Our task is now reduced to trying to find the remaining integral,
which we can use the ideas we had from last chapter.  Adams-Bashforth
of order $m$ uses a polynomial approximation to $f(x,y)$ at the points
$x_{n}$, $x_{n-1}$, \ldots, $x_{n-m+1}$.  Consider the second order
Adams-Bashforth method.
\beqn
f(x,y(x)) & \approx &
  \frac{x_{n}-x}{h}f(x_{n-1})+\frac{x-x_{n-1}}{h}f(x_{n}) \\
\int_{x_{n}}^{x_{n+1}}f(x,y)dx & \approx &
  \int_{x_{n}}^{x_{n+1}}\left(
  \frac{x_{n}-x}{h}f(x_{n-1})+\frac{x-x_{n-1}}{h}f(x_{n})
  \right)dx \\
  & \approx &
  \left.
  \frac{x_{n}x-\frac{1}{2}x^{2}}{h}f(x_{n-1})+
  \frac{\frac{1}{2}x^{2}-x_{n-1}x}{h}f(x_{n})
  \right|_{x_{n}}^{x_{n+1}} \\
  & \approx &
  \frac{x_{n}h-\frac{x_{n+1}^{2}-x_{n}^{2}}{2}}{h}f(x_{n-1})+
  \frac{\frac{x_{n+1}^{2}-x_{n}^{2}}{2}-x_{n-1}h}{h}f(x_{n}) \\
  & \approx &
  \frac{x_{n}h-\frac{x_{n+1}^{2}-x_{n+1}x_{n}+x_{n+1}x_{n}-x_{n}^{2}}{2}}{h}f(x_{n-1}) \\
  & & \qquad +
  \frac{\frac{x_{n+1}^{2}-x_{n+1}x_{n}+x_{n+1}x_{n}-x_{n}^{2}}{2}-x_{n-1}h}{h}f(x_{n}) \\
  & \approx &
  \frac{x_{n}h-\frac{x_{n+1}h+x_{n}h}{2}}{h}f(x_{n-1})+
  \frac{\frac{x_{n+1}h+x_{n}h}{2}-x_{n-1}h}{h}f(x_{n}) \\
  & \approx &
  \frac{2x_{n}-(x_{n+1}+x_{n})}{2}f(x_{n-1})+
  \frac{x_{n+1}+x_{n}-2x_{n-1}}{2}f(x_{n}) \\
  & \approx &
  \frac{x_{n}-x_{n+1}}{2}f(x_{n-1})+
  \frac{x_{n+1}-x_{n-1}+x_{n}-x_{n-1}}{2}f(x_{n}) \\
  & \approx &
  \frac{-h}{2}f(x_{n-1})+\frac{2h+h}{2}f(x_{n}) \\
  & \approx &
  \frac{h}{2}(3f(x_{n})-f(x_{n-1}))
\eeqn
This is kind of ugly, so it would be nice to have a faster way of
handling things, especially as the dimensions increase.  Luckily there
is just such a technique.  The method of undetermined coefficients.  We start by assuming the general form we want, in this case,
\beqn
\int_{x_{n}}^{x_{n+1}}f(x,y)dx & \approx & af(x_n)+bf(x_{n-1}
\eeqn

We would like the approximation to work perfectly for constant and linear terms so:

Constant term, $f(x,y)=1$
\beqn
\int_{x_{n}}^{x_{n+1}}1dx & = & a\cdot 1+b\cdot 1 \\
h &=& a+b.
\eeqn

Linear term, $f(x,y)=x$
\beqn
\int_{x_{n}}^{x_{n+1}}xdx & = & a\cdot x_n+b\cdot x_{n-1} \\
\frac{x_{n+1}^2-x_n^2}{2} &=& a\cdot x_n-a\cdot x_{n-1}+a\cdot x_{n-1}+b\cdot x_{n-1} \\
\frac{x_{n+1}^2-x_nx_{n+1}+x_nx_{n+1}-x_n^2}{2} &=& a\cdot(x_n-x_{n-1})+(a+b)\cdot x_{n-1}.
\eeqn
Now noting that $a+b=h$
\beqn
\frac{x_{n+1}(x_{n+1}-x_n)+x_n(x_{n+1}-x_n)}{2} &=& a\cdot(h)+(h)\cdot x_{n-1} \\
\frac{x_{n+1}h+x_nh}{2} &=& a\cdot h+h\cdot x_{n-1} \\
\frac{x_{n+1}+x_n-2x_{n-1}}{2} &=& a \\
\frac{x_{n+1}-x_{n-1}+x_n-x_{n-1}}{2} &=& a \\
\frac{2h+h}{2} &=& a \\
\frac{3h}{2} &=& a.
\eeqn

Thus since $a+b=h$,
\beqn
b=\frac{-h}{2}
\eeqn
which is what we found before.

\section{Adams-Moulton}

Adams-Bashforth considered that we knew only up to $f(x,y)$ only at points up to $x_n$.  What if we assume we can use $x_n$ or a near approximation?  Let us again consider just the simple case of linear approximations to function, though we could use any order of polynomial we liked (if we want to do the work).
\beqn
f(x,y(x)) & \approx &
  \frac{x-x_{n}}{h}f(x_{n+1})+\frac{x_{n+1}-x}{h}f(x_{n})
\eeqn
Using this we can solve the integral.
\beqn
\int_{x_{n}}^{x_{n+1}}f(x,y)dx & \approx &
  \int_{x_{n}}^{x_{n+1}}\left(
  \frac{x-x_{n}}{h}f(x_{n+1})+\frac{x_{n+1}-x}{h}f(x_{n})
  \right)dx \\
  & \approx &
  \left.
  \frac{\frac{1}{2}x^{2}}{h}f(x_{n+1}-x_{n}x)+
  \frac{x_{n+1}x-\frac{1}{2}x^{2}}{h}f(x_{n})
  \right|_{x_{n}}^{x_{n+1}} \\
  & \approx &
  \frac{\frac{x_{n+1}^{2}-x_n^{2}}{2}-x_nx_{n+1}+x_n^2}{h}f(x_{n+1})+
  \frac{x_{n+1}^2-x_{n+1}x_n-\frac{x_{n+1}^{2}-x_n^{2}}{2}}{h}f(x_{n}) \\
  & \approx &
  \frac{x_{n+1}^{2}-x_n^{2}-2x_nx_{n+1}+2x_n^2}{2h}f(x_{n+1})+
  \frac{2x_{n+1}^2-2x_{n+1}x_n-x_{n+1}^{2}+x_n^{2}}{2h}f(x_{n}) \\
  & \approx &
  \frac{x_{n+1}^{2}-2x_nx_{n+1}+x_n^2}{2h}f(x_{n+1})+
  \frac{x_{n+1}^2-2x_{n+1}x_n-+x_n^{2}}{2h}f(x_{n}) \\
  & \approx &
  \frac{(x_{n+1}-x_n)^2}{2h}f(x_{n+1})+
  \frac{(x_{n+1}-x_n)^2}{2h}f(x_{n}) \\
  & \approx &
  \frac{h^2}{2h}f(x_{n+1})+\frac{h^2}{2h}f(x_{n}) \\
  & \approx &
  \frac{h}{2}(f(x_{n+1})+f(x_{n}))
\eeqn


\section{Stability \& Stiff Equations}
A good start for looking at stiff equations is to examine the
stability of our methods.  Consider Forward Euler for
\beqn
y'=-200y, \qquad y(1)=e^{-200}.
\eeqn
The solution can easily be seen to be $e^{-200x}$.  Forward Euler
gives us
\beqn
y_{i+1} & = & y_{i}-200*h*y_{i} \\
 & = & y_{i}(1-200*h).
\eeqn
We note that this is stable if $\| 1-200*h \|<1$.  Since $h>0$ we
must have $h<0.01$.  This is a smooth, monotonically decreasing
function that is less than $2\time10^{-87}$ and greater than zero.
Despite the smoothness and flatness, we have to take very small
steps.  To see this look at Figure~\ref{stabfig}.
\begin{figure}[h]
\begin{center}
\leavevmode
\hbox{
\epsfxsize=4in
\epsffile{stab.eps}}
\end{center}
\caption{Instability in Euler's Method}
\label{stabfig}
\end{figure}
Note that Backward Euler does not have the same problem.  It is
defined by
\beqn
y_{i+1} & = & y_{i}-200*h*y_{i+1} \\
 & = & y_{i}\frac{1}{1+200*h},
\eeqn
which is stable for all $h>0$. Stability is not the same thing as
stiffness, but they are related.  Stiffness is due to multiple scales
of the terms, for instance $e^{-x}$, $e^{-200x}$.  Consider the
following equation
\beqn
0=\ddot{y}+1001\dot{y}+1000y, \qquad y(0)=1, \qquad \dot{y}(0)=-1.
\eeqn
The solution can easily be seen to be $e^{-t}$, which is nice in any
definition.  We solve the equation using a 4th order R-K method.  The
results are in Figure~\ref{stifffig}.
\begin{figure}[h]
\begin{center}
\leavevmode
\hbox{
\epsfxsize=6in
\epsffile{stiff.eps}}
\end{center}
\caption{Stiff Equation}
\label{stifffig}
\end{figure}

What is going on?  We need to look at the eigenvalues.

Define the intermediate variable $z=\dot y$, and the equation becomes
\beqn
0&=&\dot z +1001z+1000y \\
\dot z&=&-1001z-1000y.
\eeqn
Putting this in matrix form
\beqn
\dot{\bmat y\\z \emat} &=&\bmat 0 & 1 \\ -1000 & -1001 \emat \bmat y\\z\emat
\eeqn
The eigenvalues are (-1,-1000).  Since the eigenvalues differ by three orders of magnitude we can expect stiffness, which is what Figure~\ref{stifffig} shows.

