The fundamental theorem of Calculus tells us that an integral of a 
function can be expressed in terms of the anti-derivative of the 
function.  Unfortuneately, not all functions have anti-derivatives 
that are expressable in known functions.  One of the most famous is 
the Gaussian probability distribution, which is given by
\beqn
e^{-\left(\frac{x-\mu}{\sigma}\right)^{2}}.
\eeqn
The anti-derivative of this important and frequently occuring function 
is unknown.  How do we handle it?  That is the subject of this chapter.

\section{Riemann}
We recall from Calculus that the integral is defined as
\beqn
\int_{a}^{b}{f(x)dx} = 
\lim_{n\rightarrow\infty}\sum_{j=1}^{n}f(p_{j})(x_{j}-x_{j-1}).
\eeqn
Now assume that all $n$ of the $x_{j}$ are evenly spaced on $[a,b]$.  We 
can then write
\beqn
h & = & \frac{b-a}{n} \\
  & = & x_{j}-x_{j-1}.
\eeqn
We can use this to get an expression for the Riemann Sum
\beqn
\int_{a}^{b}{f(x)dx}
 & = & 
\lim_{n\rightarrow\infty}\sum_{j=1}^{n}f(p_{j})(x_{j}-x_{j-1}) \\
 & = & 
\lim_{n\rightarrow\infty}\sum_{j=1}^{n}f(p_{j})h \\
 & = & 
\lim_{n\rightarrow\infty}h\sum_{j=1}^{n}f(p_{j}).
\eeqn
To evaluate the integral numerically we are not able to take the 
limit, so we get
\beqn
\int_{a}^{b}{f(x)dx} 
 & \approx & 
h\sum_{j=1}^{n}f(p_{j}).
\eeqn
The exact size of $n$ for the approximation to be good is a key aspect 
of numerical integration.  Note also that I have not specified what 
$p_{j}$ is, as this form allows you to do a left, right, mid-point, 
maximum, or minimum.  The basic idea here is that we are approximating 
the function by a constant on the interval.

\setlength{\lll}{\textwidth}
\addtolength{\lll}{-2\fboxsep}
\addtolength{\lll}{-2\fboxrule}
\noindent
\fbox{%
\begin{minipage}{\lll}

\beqn
\int_{a}^{b}{f(x)dx} 
 & \approx & 
h\sum_{j=1}^{n}f(p_{j}).
\eeqn

\end{minipage}}
