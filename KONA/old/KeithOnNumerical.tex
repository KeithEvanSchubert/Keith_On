\documentstyle[epsf]{book}
\topmargin .5in
\textheight 7.5in
\textwidth 5.5in
\evensidemargin .5in
\oddsidemargin .5in

\makeindex

% Define mathematical symbols and notation
\def\RE{\mathop{\rm I\mkern-3.2mu R}\nolimits}
\def\FE{\mathop{\rm I\mkern-3.2mu F}\nolimits}
\def\CO{\mathop{\rm \mkern0.7mu \raisebox{0.5pt}{\hbox{\vrule height 7pt}}
   \mkern-5.5mu C}\nolimits}
\def\rddots{\mathinner{\mkern1mu\raise1pt\vbox{\kern7pt\hbox{.}}\mkern2mu
   \raise4pt\hbox{.}\mkern2mu\raise7pt\hbox{.}\mkern1mu}}
\def\beq{\begin{eqnarray}}
\def\eeq{\end{eqnarray}}
\def\beqn{\begin{eqnarray*}}
\def\eeqn{\end{eqnarray*}}
\def\beqf{
  \newlength{\lll}
  \setlength{\lll}{\textwidth}
  \addtolength{\lll}{-2\fboxsep}
  \addtolength{\lll}{-2\fboxrule}
  \noindent
  \fbox{%
  \begin{minipage}{\lll}
  \vspace{-\abovedisplayskip}
  }
\def\eeqf{\end{minipage}}}
\def\bfr{\begin{flushright}}
\def\efr{\end{flushright}}
\def\bfl{\begin{flushleft}}
\def\efl{\end{flushleft}}
\def\bq{\begin{quote}}
\def\eq{\end{quote}}
\def\bt{\begin{center}
        \begin{tabular}}
\def\et{\end{tabular}
        \end{center}}
\def\keseq#1{\underline{#1}}

\newcommand{\diag}{{\rm diag}}
\newcommand{\fl}{{\rm fl}}
\newcommand{\In}{{\rm In}}
\newcommand{\Null}{{\cal N}}
\newcommand{\Pf}{\bfl {\it Proof:} \efl}
\newcommand{\Pfend}{\bfr $\diamondsuit$ SDG $\diamondsuit$ \efr}
\newcommand{\Ra}{{\cal R}}
\newcommand{\rank}{{\rm rank}}
\newcommand{\REn}{\RE^n}
\newcommand{\REm}{\RE^m}
\newcommand{\REnn}{\RE^{n \times n}}
\newcommand{\REmm}{\RE^{m \times m}}
\newcommand{\REmn}{\RE^{m \times n}}
\newcommand{\REnm}{\RE^{n \times m}}
\newcommand{\REmnr}{\RE^{m \times n}_r}
\newcommand{\sgn}{{\rm sgn}}
\newcommand{\sig}{{\rm sig}}
\newcommand{\Sp}{{\rm Sp}}
\newcommand{\Tr}{{\rm Tr}}
\newcommand{\vvec}{{\rm vec}}

\newcommand{\no}{\noindent}  
\newcommand{\ba}{\left[ \begin{array}}  
\newcommand{\ea}{\\ \end{array} \right]}  
\newcommand{\bb}{\cite} 
\newcommand{\VE}[1]{\in {\bf R}^{{#1}}} 
\newcommand{\EP}[2]{\in {\bf R}^{{#1} \times {#2}}}
\newcommand{\diago}{\mbox{\rm diag}}
\newcommand{\gammaopt}{\gamma_{opt}}
\newcommand{\xopt}{x_{opt}}
\newcommand{\xgam}{x(\gamma)}
\newtheorem{problem}{Problem}
\newlength{\lll}

\title{
  {\Huge
  Keith On \ldots \\
  \huge
  Numerical Analysis
  \normalsize}
}
\author{
  K.E. Schubert \\
  Visiting Assistant Professor \\
  Department of Mathematics \\
  University of Redlands
}
\date{}

\begin{document}

\newtheorem{definition}{Definition}[chapter]
\newtheorem{theorem}[definition]{Theorem}
\newtheorem{lemma}[definition]{Lemma}
\newtheorem{corollary}[definition]{Corollary}
\baselineskip=1.05\normalbaselineskip

\maketitle
\tableofcontents
\listoffigures
\listoftables
\newpage


\pagenumbering{arabic}

\chapter{Preliminaries}\label{c-Prelim}
%\section{Taylor Polynomials}
We want an easier way of calculating a difficult function.  To this end we 
want to find a function that is similar to our original that we can 
calculate.  Taylor polynomials are one such type of functions with an easy 
calculation and intuition.  To find the Taylor polynomials we match the 
derivatives of the two polynomials at a particular point.  We are in 
essence enforcing a smoothness criterion at the point of interest.

Thus the general expression for the Taylor series is:

Example
Problem 1.1-3(c) 


Example
Problem 1.1-8
 

Remainder
The Taylor Series obviously has errors in its approximation.  If the 
original function is in $C_{n+1}$ on the interval _�x�_ (with $a$ in the 
interval) then the remainder (or error) is given by

with cx between a and x.  To get an error bound we assume that cx is the 
worst possible.
Example
Problem 1.2-3(a)
In this case n=1 so the worst case would be if cos(c) were -1.

Example
Prove problem 8.
Multiplying Polynomials
Straightforward:
a1*x*x*...*x
This takes k multiplications for a monomial of size k. so for a polynomial 
with monomials up to size n it would take n(n+1)/2 multiplications.
Storing:
Calculate x2=x*x, x3=x*x2, etc.
This takes 2n-1 multiplications.

Nesting:
b0=a0+xb1
b1=a1+xb2
bn=an
Each step takes 1 multiply so this method takes only n multiplications.

The real savings come when you have to calculate a large polynomial many 
times.


Binary
In any number system, the position of a digit relative to the decimal 
place specifies the integer power of the base we must multiple the digit 
by to get its value.  So for base 10
,
and for base 2
.
This gives us one way to convert numbers.  For instance, we can convert 
binary to decimal by expanding the binary number in this way.  Thus using 
the above to convert binary (10.01) to decimal we find,

Note that the "2" we are using is the base of binary in decimal form, and 
this is why we went from binary to decimal.  In binary, its form would be 
"10" and ten would be "1010". Therefore, we could go to binary by, 
expanding this out with ten in binary.  The problem with this method is it 
is clumsy to use since we do not do squaring, cubing, etc. easily in base 
2.  Another problem is that 0.1 is an infinitely repeating decimal in 
binary so it is a pain to deal with 10-1!  Instead, we convert decimal to 
binary as follows.  
1) Split your number into a.b
2) For the whole number part (a)
a) Divide 2 into a and note the quotient and remainder as q1,r1 (a=2*q1+r1)
b) As long as the quotient from above is not zero, divide it by 2 and 
record the quotient and remainder as qi,ri (with i denoting the current 
step).  Repeat.
c) The binary equivalent of a is rnrn-1...r2r1.  Basically we have done 
our nested polynomial evaluation backwards with x=2, and the coefficients 
being the remainders.
3) For the fractional part (b)
a) Multiply 2*b, and record the unit value as a1.  Denote b-a1=b1.
b) If bi does not equal zero, multiply it by 2, denoting the units digit 
by ai+1 and the difference bi-ai+1=bi+1.  Repeat until the difference is 
zero (this may never happen so be looking for patterns to get repeating 
fractions).
c) The fractional part, b, is a1a2a3a4...
4) The full answer is thus rnrn-1...r2r1.a1a2a3a4...
Hexadecimal
This is often made to sound more intimidating than it is.  Hexadecimal 
numbers are simply base 16, but this can be handled nicely since 24=16.  
All you have to do is group binary digits into groups of 4 and use the 
conversion table:
Bin
Hex
Dec
Bin
Hex
Dec
0000
0
0
1000
8
8
0001
1
1
1001
9
9
0010
2
2
1010
A
10
0011
3
3
1011
B
11
0100
4
4
1100
C
12
0101
5
5
1101
D
13
0110
6
6
1110
E
14
0111
7
7
1111
F
15
Floating point numbers
While the book discusses single precision numbers, they are essentially 
never used, as double precision is so much better and readily available.  
We will assume IEEE double precision floating point representation, as it 
is the standard.  IEEE floating point numbers have the form
,
where


Single Precision
Double Precision
P
24
53
Emin
-126
-1022
Emax
127
1023
Bias
127
1023
Thus IEEE is represented in memory as a sign bit, exponent bits (8 or 11), 
and mantissa bits (23 or 52).  The mantissa is composed of all the bj.  A 
few things to note about IEEE arithmetic.
1. The exponent stored is E=e-Bias
2. �0 is encoded by Emin-1 and f=0
3. Denormalized numbers are encoded by Emin-1 and f�0
4. �_ is encoded by Emax+1 and f=0
5. NAN is encoded by Emax+1 and f�0
Approximating the Reals
To approximate the real number x, we define the function fl(x) as, 0 when 
x=0, and the nearest element in floating point to x otherwise.  Finding 
nearest elements requires a rounding scheme (rounding or 
"chopping"/truncating) and a tie breaker procedure (usually round away 
from zero).

Bounding Errors
To bound the error in approximating the real number x, we need to consider 
the floating point number, fl(x), used to approximate x.  First we note 
that a real number x, is written in binary as
,
where s is the sign, f has as many digits as needed, and e is any 
integer.  Note that e will be different for IEEE, which normalizes to 
1�fr<2 with an implicit 1 at the start; than the non-standard forms, which 
normalize to 0.5� fr<1 with no assumed leading 1.  We will assume that e 
is within the permitted bounds for simplicity.  The floating-point 
representation is 
.
We can now write the difference as

For the moment, we will consider the difference (fr-f).  Note that we are 
dealing with normalized numbers with n bits of accuracy and an implicit 
leading 1 (IEEE arithmetic), while the book deals with numbers normalized 
between a half and one, with no implicit 1, so for us
.
Note that the digit to the left of the decimal in f is assumed to be 1, 
the only exception is when fr=1.1111... which would have f=10.000...0.  
Technically it would actually have f=1.000...0 and the exponent would be 
(e+1) but since we are keeping the exponent e we keep the simplification.  
Note that this is equivalent to rounding 9.5 to 10.  Anyway, our real 
concern is the worst case of the difference, which is in all cases given by
.
Note that the 1 is in the (n+1)st place after the decimal.  We rewrite 
this using floating point notation as
.
We now stick this back into the expression for the difference between x 
and fl(x) and obtain an upper bound by taking absolute value
.
Similarly to get a lower bound we take the negative of the absolute value, 
and find

Now we note that the size of x is
.
For the book's form of the mantissa, we would have
.
The relative error is thus
.
Matlab Programming
There are two basic ways to interact with Matlab: command line execution, 
and M-files. Yes there are others such as MEX-files, Simulink, and several 
interfacing programs, but they are not relevant to us.  
We will primarily be concerned with the use of M-files, because they are 
the most helpful.  Command line execution is really just for quick 
operations and checking of segments of code.  Matlab syntax is a high 
level programming language that interacts with a series of numerical 
libraries (most notably LinPack, EisPack, and BLAS).  Like most 
programming languages we have two types of programs that can be written.  
A regular program, which is written as you would type commands on the 
command line, is the most basic type and is often the way you will start 
homework problems and other projects.  Functions, which are sub-programs 
called by another program (even recursively by other functions), are 
probably the most useful, as they allow you to extend the language by 
defining new operations.  One of the main goals of this class is for you 
to walk away with a library of Matlab functions that you can use to do a 
variety of tasks.  So how do you specify which you want?  You will get a 
regular program unless you start the M-file with the command function.  
The syntax is
function a=name(x,y,..., z)
or
function [a,b,...,c]=name(x,y,..., z)
The second form returns multiple values.
Matlab gives us several command structures also: for, while, and 
if-elseif-else.  To see how these work lets use the programs I passed out 
last time as an example.

Homework:
Convert the Fortran program in 3.1 into Matlab syntax.
Do problems 9, 13, 14 from section 3.1

Propagation of Error
We have seen that representing the real numbers on a computer involves 
errors.  When we use floating point numbers in a calculation rather than 
the actual numbers the errors can grow.  The errors caused by using 
floating point approximations are called propagated errors.  Two ways of 
bounding propagation errors exist.  The forward method involves explicitly 
calculating the errors and is called interval arithmetic.  The backward 
method involves finding a condition number, which gives a bound on how big 
the error can grow.
Interval Arithmetic
Let's consider the error in a computation between the true values (xT, yT) 
and the approximate values (xA, yA).  We only know the approximate values 
and the error bounds

Note that the error is could be positive or negative so we must consider 
the positive and negative bounds.  First, we will look at the error for 
addition or subtraction.

Now let's consider multiplication.  

It is easy to see that this can quickly become very hard to deal with.  
Consider for instance multiplying two n-by-n matrices, which would involve 
$n^{3}$ multiplies.  Keeping track of all of them would rapidly become 
impossible.  We will consider one final operation, namely division.

Again we can see that things can become very complicated quickly.
Condition Number
We will now consider the problem of evaluating a function, f(x), at an 
approximate rather than true value.  To do this we will require our 
function to be continuous on [xT,xA] and differentiable on (xT,xA).  We 
can thus use the mean value theorem to see

We now note that since c is between the true and approximate values, and 
that the interval is on the order of 10-16 for IEEE double-precision 
arithmetic.  We can thus assume c is approximately xA.

The derivative of f(x) at xA, is called the condition number and shows how 
the error of the approximation will influence the error of the 
calculation.  The condition number is nice in that it cleanly handles the 
error bounds.  It is not as precise as the error in the interval 
arithmetic, but it is tractable even for large matrix operations, which 
will involve the norms of the matrices rather than the elements.  Quite a 
savings!
Sums
We have spoken a lot about summation, but we want to look at one final 
area of sums before we move on.  Consider the following summation:

In real numbers it doesn't matter if we add the 45's first or the 100000.  
In floating point numbers it does matter!  Floating point numbers are not 
associative.  To see this consider a 4 decimal place accuracy machine that 
uses rounding, and is nicely implemented.  In this case we see that 
100000+45=100000
so if we add as stated we find the sum is 100000 for the series (rather 
than 100180).  If we add the 45's first we find that 
45+45+45+45=180.
Then 
100000+180=100200.
A much better result.  These sums occur in a variety of places, from 
standard series, to evaluating integrals, to inner products of vector, and 
matrix multiplication.  In short you should be aware of the lack of the 
associative property.


\newpage
\chapter{Zero Finding}\label{c-zero}
Almost every interesting problem in mathematics can be reduced to trying 
to find the zeros of a function.  The next several classes will be spent 
examining how we find zeros.  In general, you cannot explicitly solve for 
the zeros so you need to make iterative procedures to find them.  Today we 
will look at two methods: bisection and Newton's method.
\section{Bisection}
Bisection is a nice method in that it is guaranteed to converge and you 
can state exactly how many iterations it will take.


\section{Newton's Method}
Newton's Method essentially is an algebraic re-writing of the tangent line 
of a function at a point.

We can then use Taylor's formula to obtain an error bound.
\section{Secant}
Newton's Method requires the knowledge of the first derivative of the 
function.  Often the derivative is very complicated to evaluate and 
will take a long (relatively anyway) time to do so.  In many cases the 
first derivative may not be available.  In some cases it might not even 
exist at all points in the interval of interest.  Even when it is 
available it could be near zero which would cause numerical problems 
in evaluating it, even if it is in the region of convergence.  For 
all of these regions a new method was devised, which drew on the 
material leading up to calculus.  

Recall that the tangent line was 
found as the limit of a series of secant lines.  We can say that the 
derivative can thus be approximated by
\beqn
f(x)\approx\frac{f(x_{1})-f(x_{2})}{x_{1}-x_{2}}.
\eeqn
Thus if we know two points, we can approximate the funciton by a 
straight line between them and use the x-intercept as the next point 
to evaluate.  We now need two points instead of one and a 
derivative.  We refer to this as a two-point method because of the 
need of multiple points.  We will need two estimates to begin our 
evaluation.  Given two initial gueses, $x_{0}$ and $x_{1}$, the slope, 
$m$, is given by
\beqn
m=\frac{f(x_{1})-f(x_{0})}{x_{1}-x_{0}}.
\eeqn
Using this we find the next point, $x_{2}$ by using the point-slope 
form of a line
\beqn
f(x_{2})-f(x_{1}) & = & 
  \frac{f(x_{1})-f(x_{0})}{x_{1}-x_{0}}(x_{2}- x_{1}) \\
x_{2}- x_{1} & = & 
  \frac{x_{1}-x_{0}}{f(x_{1})-f(x_{0})}(f(x_{2})-f(x_{1})) \\
x_{2} & = & 
  x_{1}-f(x_{1})\frac{x_{1}-x_{0}}{f(x_{1})-f(x_{0})}. 
\eeqn
We thus have the equation for the next estimate:
\beq
x_{n+1} = 
  x_{n}-f(x_{n})\frac{x_{n}-x_{n-1}}{f(x_{n})-f(x_{n-1})}. \label{eq-sec1}
\eeq
Note that you can store the previous function evaluation and then you 
will not need to do two function evaluations per iteration.

Now we want to calculate the error.  To do this we will subtract 
eq~\ref{eq-sec1} from $\alpha=\alpha$.
\beqn
e_{n+1} & = & \alpha - x_{n+1} \\
  & = & \alpha -
    \left(x_{n}-f(x_{n})\frac{x_{n}-x_{n-1}}{f(x_{n})-f(x_{n-1})}\right) \\
  & = & \alpha -\frac{f(x_{n})x_{n-1}-f(x_{n-1})x_{n}}{f(x_{n})-f(x_{n-1})} \\
  & = & \frac{f(x_{n})(\alpha -x_{n-1})-f(x_{n-1})(\alpha -x_{n})}
       {f(x_{n})-f(x_{n-1})} \\
  & = & \frac{f(x_{n})e_{n-1}-f(x_{n-1})e_{n}}{f(x_{n})-f(x_{n-1})} \\
  & = & e_{n} e_{n-1}\frac{\frac{f(x_{n})}{e_{n}}-\frac{f(x_{n-1})}{e_{n-1}}}
       {f(x_{n})-f(x_{n-1})} \\
  & = & e_{n} e_{n-1}\frac{\frac{f(x_{n})}{e_{n}}-\frac{f(x_{n-1})}{e_{n-1}}}
       {x_{n}-x_{n-1}}\frac{x_{n}-x_{n-1}}{f(x_{n})-f(x_{n-1})} \\
  & \approx & e_{n} e_{n-1}\frac{\frac{f(x_{n})}{e_{n}}-\frac{f(x_{n-1})}{e_{n-1}}}
       {x_{n}-x_{n-1}}\frac{1}{f'(\alpha)}
\eeqn
We need to evaluate $\frac{f(x_{n})}{e_{n}}$, so we will use Taylor's 
Theorem for $f(x)$ evaluated at $\alpha$.  We find that
\beqn
\frac{f(x_{n})}{e_{n}} & = & 
\frac{f(\alpha)+(\alpha-x_{n})f'(\alpha)+\frac{1}{2}
  (\alpha-x_{n})^{2}f''(\alpha)+{\cal O}((\alpha-x_{n})^{3})}{e_{n}} \\
& = & 
\frac{e_{n}f'(\alpha)+\frac{1}{2}e_{n}^{2}f''(\alpha)
   +{\cal O}(e_{n}^{3})}{e_{n}} \\
& = & 
f'(\alpha)+\frac{1}{2}e_{n}f''(\alpha)+{\cal O}(e_{n}^{2})
\eeqn
Resuming our evaluation of $e_{n+1}$ we find
\beqn
e_{n+1} 
  & \approx & e_{n} e_{n-1}\frac
  {f'(\alpha)+\frac{1}{2}e_{n}f''(\alpha)+{\cal O}(e_{n}^{2})
  -f'(\alpha)-\frac{1}{2}e_{n-1}f''(\alpha)+{\cal O}(e_{n-1}^{2})}
       {x_{n}-x_{n-1}}\frac{1}{f'(\alpha)} \\
  & = & e_{n} e_{n-1}\frac{\frac{1}{2}e_{n}f''(\alpha)
   -\frac{1}{2}e_{n-1}f''(\alpha)+{\cal O}(e_{n-1}^{2})}
       {x_{n}-x_{n-1}}\frac{1}{f'(\alpha)} \\
  & = & e_{n} e_{n-1}\frac{\frac{1}{2}(e_{n}-e_{n-1})f''(\alpha)
   +{\cal O}(e_{n-1}^{2})}{x_{n}-x_{n-1}}\frac{1}{f'(\alpha)} \\
  & = & e_{n} e_{n-1}\frac{\frac{1}{2}(x_{n}-x_{n-1})f''(\alpha)
   +{\cal O}(e_{n-1}^{2})}{x_{n}-x_{n-1}}\frac{1}{f'(\alpha)} \\
  & = & e_{n} e_{n-1}(\frac{1}{2}f''(\alpha)
   +{\cal O}(e_{n-1}^{2}))\frac{1}{f'(\alpha)} \\
  & \approx & e_{n}e_{n-1}\frac{f''(\alpha)}{2f'(\alpha)} \\
  & \approx & e_{n}e_{n-1}M.
\eeqn
This is similar to Newton's method which suggests that
\beqn
  e_{n+1}=Ae_{n}^{c},
\eeqn
which implies
\beqn
  e_{n}=A^{-1}e_{n-1}^{c^{-1}}.
\eeqn
Substituting and collecting terms we find
\beqn
B=e_{n}^{1-c+c^{-1}}.
\eeqn
Since the left hand side is a constant the exponent must be zero, or 
$c$ must be the golden ratio.
This implies that the secant method converges superlinearly.
\section{Regula Falsi}


\section{Fixed Points}
A fixed point is a point in the domain of a function, which maps its 
domain back into its domain, that satisfies $\alpha=C(\alpha)$.  
Since $\alpha$ does not change when it is mapped by the function it is 
fixed, hence the name.  We need to look at what 
the idea that underlies fixed points: contractions.  A contraction 
$y=C(x)$, is a mapping from a closed interval in $X$ into another closed 
interval in $Y$ with the property that for some $b=C(a)$ (usually $a$ 
and $b$ are both the origin but it is not required), $\|\cdot\|_{x}$ 
a norm on $X$, and $\|\cdot\|_{y}$ a norm on $Y$ we have:
\beqn
\| x-a\|_{x} > \| y-b\|_{y} = \| C(x)-C(a)\|_{y}
\eeqn
for all $x\in X$ and $y\in Y$.  Usually we have $X$ and $Y$ are $\Re$ 
and $a=b$, which gives us that $|x-a|>|C(x)-a|$.  Take the derivative of both 
sides and we see 
\beqn
1>|C'(x)|.
\eeqn
This brings up a key point, we must have that the magnitude of the 
function's slope is less than 1.  If you think about this it makes 
sense, as for slope magnitudes greater than one there will be growth 
and we are looking a funcions which shrink things.  While this is a 
simple idea, it has many profound implications.  The book proves 
nicely how the uniqueness of solution, convergence, etc..  One thing 
that should be highlated has to do with rate of convergence.  Given a 
contraction defined on an interval $[a,b]$ with some point, $\alpha = 
C(\alpha)\in[a,b]$ called a fixed point, we can define the iteration 
$x_{n+1}=C(x_{n})$.  We then have (using the mean value theorem)
\beqn
\alpha-x_{n+1} 
 & = & C(\alpha)-C(x_{n}) \\
 & = & C'(d)(\alpha-x_{n}) \\
|\alpha-x_{n+1}| 
 & < & |\alpha-x_{n}|.
\eeqn
We have linear convergence from this.  Consider the following paradox.

Let a function $g(x)$ be defined by
\beqn
g(x)=x-\frac{f(x)}{f'(x)}
\eeqn
and let $f(x)$ have a single root in some interval $[a,b]$.  From the 
book we know this must have a fixed point in the interval and the 
iteration $x_{n+1}=g(x_{n})$ will converge to the fixed point.  This 
method thus has linear convergence from what we have proven above.  
This iteration is Newton's Method though, so it has Quadratic 
convergence.  What gives?  The convergence of a fixed point algorithm 
is at least linear but it can be better if $C'(\alpha)=0$.  Notice 
that the derivative of $g(x)$ is given by
\beqn
g'(x) & = & 1-\frac{(f'(x))^{2}-f(x)f''(x)}{(f'(x))^{2}} \\
      & = & \frac{f(x)f''(x)}{(f'(x))^{2}}.
\eeqn
Note that for $x=\alpha$ we trivially have that $g'(\alpha)=0$, which 
satisfies our requirement for faster convergence.

How can I get a function $g(x)$ that satisfies the requirements?  
Many ways exist but consider the following.  For a function $f(x)$ with 
a zero at $x=\alpha$ in an interval $[a,b]$, that has 
$\beta=\max_{x\in[a,b]}|f'(x)|$, we define the iteration
\beqn
x_{n+1}=x_{n}-\gamma sign(f'(x_{n}))f(x_{n})
\eeqn
with $0<\gamma\beta<2$.  We then see that
\beqn
g(x) & = & x-\gamma sign(f'(x))f(x) \\
g'(x) & = & 1-\gamma sign(f'(x))f'(x) \\
      & = & 1-\gamma |f'(x)|
\eeqn
and thus $1>g'(x)>-1$.  Note that if we choose $\gamma$ such that 
$0<\gamma\beta<1$ then $1>g'(x)>0$ and the sequence 
$\{x_{i}\}_{i=0}^{\infty}$ converges to $\alpha$ from one side (no 
alternating).  The parameter $\gamma$ is refered to as the {\bf step 
size}.  As a final note, we can use Aitken's $\Delta^{2}$ method as 
outlined in the book to refine the estimate $x_{n}$.  Replace $\alpha$ 
with $\hat{x}_{n}$ and you have a refinement and acceleration method 
that will work on any linearly convergent algorithm.  It can thus be 
used on general fixed point methods.

Homework 
4.3: 6, 13
\newpage
\section{Continuation Methods}
One of the essential problems in root finding is to find a good place 
to start.  We have spoken about the progressively doubling intervals 
till we find a sign change.  I mentioned this was not the fastest or 
best, but would work.  I wanted to give you what I think is one of the 
best.  It is refered to as a continuation method or sometimes a 
homotopy.

A homotopy, $h$, is a continuous connection between two functions, $f$ and 
$g$, that maps one space, $X$, to another, $Y$:
\beqn
h:[0,1]\times X\rightarrow Y
\eeqn
such that $h(0,x)=g(x)$ and $h(1,x)=f(x)$.  Two simple homotopies we will 
use are listed below.
\begin{enumerate}
\item
\beqn
h(\lambda,x)=\lambda f(x)+(1-\lambda)g(x)
\eeqn
\item
\beqn
h(\lambda,x) & = & \lambda f(x)+(1-\lambda)(f(x)-f(\alpha_{0})) \\
             & = & f(x)-(1-\lambda)f(\alpha_{0})
\eeqn
\end{enumerate}
The first one is the most general.  Assume we want to find the roots 
of $f$, but we know the roots of $g$.  By picking a sequence of 
$\lambda$ values from zero to one, we will slowly make the roots move 
from the known positions of $g$ to the unknown positions of $f$.  We 
usually try to pick $g$ so it has the same number of roots as the 
function $f$.

The second method is a frequently used one if I don't want to find a 
function $g$.  We are in essence biasing the original fucntion so that 
at $\alpha_{0}$ the homotopy has a root for $\lambda=0$.  This gives a nice 
starting point.  The following theorem tells us when this will work.

\begin{theorem}[Ortega and Rheinboldt]
If $f:\REn\rightarrow\REn$ is continuously differentiable and if 
$\|[f'(x)]^{-1}\|\leq M$ on $\REn$, then for any $\alpha_{0}\in\REn$ there 
is a unique curve $\{\alpha(\lambda):0\leq\lambda\leq 1\}$ in $\REn$ such 
that $f(\alpha(\lambda))-(1-\lambda)f(\alpha_{0})=0$, with $0\leq\lambda\leq 
1$.  The function $\lambda\mapsto \alpha(\lambda)$ is a continuously 
differentiable solution to the initial value problem 
$\alpha'=-[f'(\alpha)]^{-1}f(\alpha_{0})$, where $\alpha(0)=\alpha_{0}$.
\end{theorem}

Essentially this tells us if $f$ is smooth and the first derivative 
doesn't get too close to zero then you can use this start one of our 
rootfinding methods, for instance Newton's Method.  Often this method 
is solved by using a numerical integration technique which we will 
cover in a few weeks.  For instance if Euler's method is used then it 
turns out to generate Newton's Method in $\lambda$!

\section{Multiple Roots}
One thing that always caused us problems in all our methods is 
multiple roots.  I will present a simple technique for handling this 
case.  Let our function, $f$, with root of multiplicity, $2$, be given 
by
\beqn
f(x)=(x-\alpha)^{2}f_{1}(x),
\eeqn
where $f_{1}$ has no root at $\alpha$.  Take the derivative of $f(x)$ 
to obtain
\beqn
f'(x) & = & (x-\alpha)f_{1}(x)+(x-\alpha)^{2}f'_{1}(x) \\
      & = & (x-\alpha)(f_{1}(x)+(x-\alpha)f'_{1}(x)) \\
      & = & (x-\alpha)f_{2}(x),
\eeqn
where $f_{2}(x)$ has no root at $\alpha$.  We now have a funcition 
with a single root at the same place that the original function had a 
double root.  This can be done for higher multiplicity roots, and 
does not require knowing $\alpha$ as we are taking the derivative then 
finding $\alpha$ using one of our techniques.

\section{Sensitivity}
This is refered to as stability of the roots in the books, but it is 
more closely related to the sensitivity of a differential equation to 
perturbations in its coefficients.  For instance, consider a famous 
problem due to Wilkinson.
\begin{problem}[Wilkinson]
Find the roots of the polynomial $f(x)$ given by
\beqn
f(x) & = & (x-1)(x-2)\cdots(x-20) \\
     & = & x^{20}-210x^{19}+\cdots +20!
\eeqn
The roots are clearly one through twenty.  Perturb the coefficient 
$-210$ to $-210-2^{-23}$.  The change is in one coefficient only, and 
that in the $7^{th}$ decimal place.  The roots are now
\bt{l c c}
$1.000000000$ & $6.000006944$ & $10.095266145\pm 0.643500904j$ \\
$2.000000000$ & $6.999697234$ & $11.793633881\pm 1.652329728j$ \\
$3.000000000$ & $8.007267603$ & $13.992358137\pm 2.518830070j$ \\
$4.000000000$ & $8.917250249$ & $16.730737466\pm 2.812624894j$ \\
$4.999999928$ & $20.846908101$ & $19.502439400\pm 1.940330347j$ 
\et
The problem is not roundoff.  The roots of high-order coefficients can 
be extremely sensitive to changes in the coefficients.  This is a 
problem particularly when the coefficients are experimentally 
determined.
\end{problem}

\newpage
\chapter{Interpolation and Approximation}\label{c-IntApp}
We will now look at the problem of finding a polynomial to fit a set 
of points.  The points could come from measurements in an experiment, 
or it could come from a complex function we want to approximate.  In 
either case we will begin by considering the case where we want our 
polynomial to be exact at these values.  An obvious question is why 
the emphasis on polynomials, when so many other functions exist.  
Indeed we do see the use of other basis (sin and cos in Fourier for 
example), but still polynomials hold a special place in many 
applications.  One major reason is the Theorem of Wiestrass from Real 
Analysis.  It basically says that polynomials can approximate any 
function (assuming you use the entire basis).

\section{Lagrange Interpolation Basis}
Probably the nicest way to visualize the interpolation polynomials is 
to consider the Lagrange interpolation basis functions.  For the set 
of points, $\{x_{0}, x_{1}, \ldots, x_{n}\}$ define the following polynomial:
\beqn
L_{i}(x)=\frac{\prod_{j\ne i}(x-x_{j})}{\prod_{j\ne i}(x_{i}-x_{j})}.
\eeqn
We note in particular that $L_{i}(x_{j})=\delta_{i,j}$, which allows 
us to get the interpolation polynomial nicely.  The interpolating polynomial 
is then given by
\beqn
P_{n}(x)=\sum_{i=0}^{n}y_{i}L_{i}(x).
\eeqn
The importance of the Lagrange basis giving us the Kronecker delta 
function cannot be over-emphasized, as it is the essential idea in 
getting the solution.

Often the points are selected to be evenly spaced due to constraints 
in the basic system.  While this is not the best for errors, it is 
often a physical necessity (for example many data samplers are 
constrained this way).  In this case we can simplify the expression 
using 
\beqn
\mu=\frac{x-x_{0}}{x_{1}-x_{0}}.
\eeqn
This is covered well in the book.

\section{Divided Difference}
Divided difference is a similar method to Taylor approximation but 
instead of matching derivatives exactly at a point, it nearly 
approximates the derivative to exactly match certain points.  The 
result is the same as Lagrange's formula.
\beqn
F[x_{0},x_{1}] & = & \frac{f(x_{1})-f(x_{0})}{x_{1}-x_{0}} \\
F[x_{0},x_{1},\cdots,x_{n}]
 & = & \frac{F[x_{1},\cdots,x_{n}]-F[x_{0},\cdots,x_{n-1}]}{x_{n}-x_{0}}
\eeqn

\beqn
P_{1}(x) & = & f(x_{0})+(x-x_{0})F[x_{0},x_{1}] \\
P_{k+1} & = & 
P_{k}+(x-x_{0})(x-x_{1})\cdots(x-x_{k})F[x_{0},x_{1},\cdots,x_{k+1}]
\eeqn

Homework:

Section 5.1: 5, 9, 13

Section 5.2: 2, 3, 7

\section{Error}
The key area to note from here is that the error is given by either 
of the following formulas.
\beqn
f(x)-P_{n}(x)
 & = & 
\prod_{i=0}^{n}(x-x_{i})\frac{f^{(n+1)}(c_{x})}{(n+1)!} \\
 & = & 
\prod_{i=0}^{n}(x-x_{i})F[x_{0},x_{1},\cdots,x_{n},x]
\eeqn
The important part of this is to note that these are themselves 
polynomials of order $n+1$.  Consider the plot of a polynomial with 
equi-spaced roots.  It is trivial to note that the height of the peaks 
between the roots is bigger towards the outside of the interval.
\section{Splines}
For splines we want to fit a cubic polynomial for each interval so 
that the first and second derivatives between two sections match on 
the boundary.  Following the books derivation we get the formula for 
the polynomial on the interval $[x_{j-1},x_{j}]$ to be
\beqn
s(x) & = & 
       a_{1}(x_{j}-x)^{3}+a_{0}(x-x_{j-1})^{3}
      +b_{1}(x_{j}-x)+b_{0}(x-x_{j-1}) \\
a_{i} & = & \frac{M_{j-i}}{6(x_{j}-x_{j-1})} \\
b_{i} & = & \frac{y_{j-i}-\frac{1}{6}M_{j-i}(x_{j}-x_{j-1})^{2}}{(x_{j}-x_{j-1})}
\eeqn
The only thing that we need is to calculate $M_{i}$ for the natural 
cubic spline, which is done by 
requiring $M_{1}=M_{n}=0$ and solving the following matrix system
\beqn
Ax & = & b \\
A & = & 
\left[\matrix{
\alpha_{2} & \beta_{2}  &  0        & \cdots       & 0 \cr
\beta_{2}  & \alpha_{3} & \beta_{3} & \ddots       & \vdots \cr
0          & \beta_{3}  & \ddots    & \ddots       & 0 \cr
\vdots     & \ddots     & \ddots    & \alpha_{n-2} & \beta_{n-2} \cr
0          & \cdots     & 0         & \beta_{n-2}  & \alpha_{n-1}
}\right] \\
x & = & 
\left[\matrix{
M_{2} \cr
\vdots \cr
M_{n-1}
}\right] \qquad 
b = 
\left[\matrix{
\gamma_{2}-\gamma_{1} \cr
\vdots \cr
\gamma_{n-1}-\gamma_{n-2}
}\right] \\
\alpha_{i} & = & \frac{x_{i+1}-x_{i-1}}{3}
\qquad
\beta_{i} = \frac{x_{i+1}-x_{i}}{6}
\qquad
\gamma_{i} = \frac{y_{i+1}-y_{i}}{x_{i+1}-x_{i}}
\eeqn
We can also find the $M_{i}$ for the not-a-knot cubic spline, which is 
often preferred by solving a similar system
\beqn
Ax & = & b \\
A & = & 
\left[\matrix{
\psi_{1}   & \beta_{1}  &  0         &  0         & \cdots       & 0            & 0\cr
\beta_{1}  & \alpha_{2} & \beta_{2}  &  0         & \cdots       & 0            & 0 \cr
0          & \beta_{2}  & \alpha_{3} & \beta_{3}  & \ddots       & \vdots       & \vdots \cr
0          & 0          & \beta_{3}  & \ddots     & \ddots       & 0            & 0 \cr
\vdots     & \vdots     & \ddots     & \ddots     & \alpha_{n-2} & \beta_{n-2}  & 0 \cr
0          & 0          & \cdots     & 0          & \beta_{n-2}  & \alpha_{n-1} & \beta_{n-1} \cr
0          & 0          & \cdots     & 0          & 0            & \beta_{n-1}  & \phi_{2}
}\right] \\
x & = & 
\left[\matrix{
M_{1} \cr
M_{2} \cr
\vdots \cr
M_{n-1} \cr
M_{n}
}\right] \qquad 
b = 
\left[\matrix{
\gamma_{1}-f'(x_{1}) \cr
\gamma_{2}-\gamma_{1} \cr
\vdots \cr
\gamma_{n-1}-\gamma_{n-2} \cr
f'(x_{n})-\gamma_{n-1}
}\right] \\
\alpha_{i} & = & \frac{x_{i+1}-x_{i-1}}{3}
\qquad
\beta_{i} = \frac{x_{i+1}-x_{i}}{6}
\qquad
\gamma_{i} = \frac{y_{i+1}-y_{i}}{x_{i+1}-x_{i}} \\
\psi_{1} & = & \frac{x_{2}-x_{1}}{3}
\qquad
\phi_{2} = \frac{x_{n}-x_{n-1}}{3}  
\eeqn
or (if you don't know the derivative) 
\beqn
Ax & = & b \\
A & = & 
\left[\matrix{
\psi_{1}   & \psi_{2}   &  0         &  0         & \cdots       & 0            & 0 \cr
\beta_{1}  & \alpha_{2} & \beta_{2}  &  0         & \cdots       & 0            & 0 \cr
0          & \beta_{2}  & \alpha_{3} & \beta_{3}  & \ddots       & \vdots       & \vdots \cr
0          & 0          & \beta_{3}  & \ddots     & \ddots       & 0            & 0 \cr
\vdots     & \vdots     & \ddots     & \ddots     & \alpha_{n-2} & \beta_{n-2}  & 0 \cr
0          & 0          & \cdots     & 0          & \beta_{n-2}  & \alpha_{n-1} & \beta_{n-1} \cr
0          & 0          & \cdots     & 0          & 0            & \phi_{2}     & \phi_{1}
}\right] \\
x & = & 
\left[\matrix{
M_{1} \cr
M_{2} \cr
\vdots \cr
M_{n-1} \cr
M_{n}
}\right] \qquad 
b = 
\left[\matrix{
\psi_{3} \cr
\gamma_{2}-\gamma_{1} \cr
\vdots \cr
\gamma_{n-1}-\gamma_{n-2} \cr
\phi_{3}
}\right] \\
\alpha_{i} & = & \frac{x_{i+1}-x_{i-1}}{3}
\qquad
\beta_{i} = \frac{x_{i+1}-x_{i}}{6}
\qquad
\gamma_{i} = \frac{y_{i+1}-y_{i}}{x_{i+1}-x_{i}} \\
\xi_{1} & = & x_{2}-x_{1}
\qquad
\xi_{2} = x_{2}-z_{1}
\qquad
\xi_{3} = z_{1}-x_{1} \\
\psi_{1} & = & \frac{\xi_{2}^{3}-\xi_{1}^{2}\xi_{2}}{6\xi_{1}}
\qquad
\psi_{2} = \frac{\xi_{3}^{3}-\xi_{1}^{2}\xi_{3}}{6\xi_{1}} 
\qquad
\psi_{3} = f(z_{1})-\frac{\xi_{2}y_{1}+\xi_{3}y_{2}}{\xi_{1}} \\
\xi_{4} & = & x_{n}-x_{n-1}
\qquad
\xi_{5} = x_{n}-z_{2}
\qquad
\xi_{6} = z_{2}-x_{n-1} \\
\phi_{1} & = & \frac{\xi_{5}^{3}-\xi_{4}^{2}\xi_{5}}{6\xi_{4}}
\qquad
\phi_{2} = \frac{\xi_{6}^{3}-\xi_{4}^{2}\xi_{6}}{6\xi_{4}}
\qquad
\phi_{3} = f(z_{2})-\frac{\xi_{5}y_{n-1}+\xi_{6}y_{n}}{\xi_{4}}
\eeqn

Note, you can easily enter the matrix $A$ into Matlab by using the 
command diag.  For instance, if you put the entries of $A$ that are on 
the main diagonal into the vector $A1$, the first sub-diagonal into 
$A2$, and the first super-diagonal into $A3$, then in Matlab you enter,
{\it A=diag(A1)+diag(A2,-1)+diag(A3,1);}.

Homework

section 5.3: 7
section 5.4: 3, 5

\newpage

\section{Least Squares Approximation}

Up till know we have dealt with interpolation, where we want to 
exactly match a set of points.  In reality, we are often more 
concerned with having a good overall approximation rather than an 
exact matching at a few points.  There are a lot of ways to 
approximate a function.  In general there are two main areas discrete 
and continuous.  We will cover the discrete case.  The continuous method 
involves some functional analysis and we do not have the time to 
cover it well.  If you are interested it can provide a fun project, 
and I have some good resources you can use.

We proceed with the discrete case.  The discrete case involves 
measuring the function to be approximated at a series of points, and 
then finding the best coefficients in some sense for some functions 
of interest.  

Some sense?  What do I mean by that?  Well, put simply, there are a 
variety of different methods of measuring how good an approximation 
is.  The standard method is the one we will concentrate on, and it is 
called least squares.  As with many things in Math, least squares 
owes its basis to Gauss.  The basic idea is to reduce the sum of the 
squares of the distances from the measurements to the function to be 
fitted at each of the x values.  The last point is very important 
because it is the basis of much of the problems in least squares.  In 
essence the answer you get is dependent on your choice of independent 
variables.  Below is an excerpt from my dissertation which covers what 
we are talking about now.  The key idea to get is that there are 
reasons to look beyond least squares.

Consider the problem of calibrating a gas thermometer.  Gas 
thermometers are based on Charles' law, which states that the volume 
of a fixed mass of gas at a fixed pressure is proportional to its 
temperature.  A simple gas thermometer can be made by trapping some 
gas with a mercury plug in a capillary tube that is open on only one 
end \bb{GenChem}.  The volume is thus proportional to the height of the 
plug.  The equation of the thermometer is thus $hc_{1}=T$, where $h$ is the 
height of the plug, $c_{1}$ is the constant we want to know, and $T$ is the 
absolute temperature.  We place the gas thermometer in a stirred liquid bath 
with a known thermometer.  We heat the bath and take height and temperature 
measurements at various times.  The LS solution gives us 
that $\hat c_{1}=h^{\dagger}T$, but we can see that this minimizes the 
error in the measured temperature, $T$, from the predicted temperature, 
$hh^{\dagger}T$.  By the same token we could use the relation $h=c_{2}T$, 
with $c_{2}=\frac{1}{c_{1}}$.  The LS solution, $\hat c_{2}=T^{\dagger}h$, 
thus minimizes the error between the measured height, $h$, and the predicted 
height $TT^{\dagger}h$.  A problem arises in the LS 
method in that generally $\hat c_{1}\ne\frac{1}{\hat c_{2}}$.  This 
can be seen easily in Figure~\ref{gastherm}.  The slope of the line designated 
temperature errors, is $\hat c_{1}$, while the slope of the line 
designated height errors is $\frac{1}{\hat c_{2}}$.  The line 
designated theoretical is the ``true'' system from which the estimates 
were generated.  It is easy to see that the slopes are not the same, 
and thus $\hat c_{1}\ne\frac{1}{\hat c_{2}}$.  The LS solution does 
not even perfectly handle the case where the system matrix is 
``known'', which gives us cause to be concerned as to how it will 
perform when there are perturbations to the system matrix.

\begin{figure}[h]
\begin{center}
\leavevmode
\hbox{
\epsfxsize=4in
\epsffile{gastherm.eps}}
\end{center}
\caption{Gas Thermometer Example}
\label{gastherm}
\end{figure}

The most well known alternative to least squares is total least 
squares (TLS).  In TLS we look at the perpendicular distance to the 
function.  This handles many of the problems of least squares but is 
more sensitive to errors, as it is ``optimistic'' in how it looks at 
the problem.  A huge body of literature is dedicated to this problem, 
and this is the central area of my dissertation.  While some of these 
other methods are very interesting, we will stick to least squares 
for the moment, but we will remember that problems can occur and so 
if we have problems we know there are things we can do.

Getting back to business we have a set of $m$ points $(x_{i},y_{i})$ and 
a group of $n$ functions $\phi_{i}(x)$ that we want to use to 
approximate the points with.  We thus have $m$ equations to find $n$ 
coefficients.
\beqn
y_{1} & - & \sum_{i=1}^{n}a_{i}\phi_{i}(x_{1}) \\
y_{2} & - & \sum_{i=1}^{n}a_{i}\phi_{i}(x_{2}) \\
& \vdots & \\
y_{m} & - & \sum_{i=1}^{n}a_{i}\phi_{i}(x_{m})
\eeqn
We can rewrite these into a matrix formulation, as
\beqn
Y-\Phi A \\
\eeqn
where 
\beqn
Y & = & \left[\matrix{y_{1} & y_{2} & \cdots & y_{m}}\right]^{T} \\
\Phi & = & \left[\matrix{
\phi_{1}(x_{1}) & \phi_{2}(x_{1}) & \cdots & \phi_{n}(x_{1}) \cr
\phi_{1}(x_{2}) & \phi_{2}(x_{2}) & \cdots & \phi_{n}(x_{2}) \cr
\vdots          & \vdots          & \ddots & \vdots \cr
\phi_{1}(x_{m}) & \phi_{2}(x_{m}) & \cdots & \phi_{n}(x_{m})
}\right] \\
A & = & \left[\matrix{a_{1} & a_{2} & \cdots & a_{m}}\right]^{T}.
\eeqn
At this point we want to minimize the square error which is what the 
2-norm does, so we have $\min_{A}\| Y-\Phi A\|_{2}^{2}$ .  The norm we 
are minimizing is called the 
cost function.  The solution is given by $A=\Phi^{\dagger}Y$, where 
$\Phi^{\dagger}$ is called the pseudo-inverse of $\Phi$.  Prove it?  
Sure!  To avoid getting into some deeper areas of linear algebra we 
will assume that $\Phi$ has linearly independent columns.  This is not 
restrictive, as we usually have a lot of measurements and only a few 
functions we want to fit to them ($m>>n$).

We recall from calculus that the minimum occurs when the gradient (derivative) 
is zero. We thus take the gradient of the cost with respect to $A$ and 
set it equal to zero to obtain 
\beqn
0 & = & 
\nabla_{A}\| Y-\Phi A\|_{2}^{2} \\
 & = & 
\nabla_{A}(Y-\Phi A)^{T}(Y-\Phi A) \\
 & = & 
-\Phi^{T}(Y-\Phi A) \\
 & = & 
\Phi^{T}\Phi A-\Phi^{T}Y \\
\Phi^{T}Y
 & = & 
\Phi^{T}\Phi A
\eeqn
The last line is what is referred to as the normal equation(s).  Note 
that some pluralize it to reflect that the single matrix equation 
reflects $n$ scalar equations.  I don't care, use what you like.  We 
note that if $\Phi$ has linearly independent columns, then 
$(\Phi^{T}\Phi)^{-1}$ exists.
\beqn
\Phi^{T}\Phi A
 & = & 
\Phi^{T}Y \\
A
 & = & 
(\Phi^{T}\Phi)^{-1}\Phi^{T}Y \\
A
 & = & 
\Phi^{\dagger}Y
\eeqn
You might wonder how the last step works.  Some might just call it a 
definition but in reality it is because $(\Phi^{T}\Phi)^{-1}\Phi^{T}$ 
satisfies the four conditions of a pseudo inverse (called the Penrose 
conditions).  
\begin{enumerate}
\item $\Phi\Phi^{\dagger}\Phi=\Phi$
\item $\Phi^{\dagger}\Phi\Phi^{\dagger}=\Phi^{\dagger}$
\item $\Phi\Phi^{\dagger}=(\Phi\Phi^{\dagger})^{T}$
\item $\Phi^{\dagger}\Phi=(\Phi^{\dagger}\Phi)^{T}$
\end{enumerate}
The properties are 
simple and easy to check, and yes, you have to check all four.  Many 
times a candidate matrix fails only one of them.  The first two 
properties tell us that it correctly maps the range spaces from the 
fundamental theorem of linear algebra, and the second two tell us 
the composite maps are symmetric.  The pseudo-inverse always exists 
and is unique.  Additionally, when the true inverse exists, it is the 
pseudo-inverse.  These are just a few of the many reasons to love the 
pseudo-inverse\ldots

The result is established.  The nice thing about how we have handled 
things here is we have not specified what the functions are (they have 
to be linearly independent but that is no problem) or how many of them 
we want to fit.  You can now fit any combination of functions you like.

As an example let's look at linear least squares for the points 
(0,1), (1,2), and (2,3).  We need to find the coefficients $m$, and $b$ 
for the line.  We construct our matrices
\beqn
Y & = & \left[\matrix{1 & 2 & 3}\right]^{T} \\
\Phi & = & \left[\matrix{
1 & 1 & 1 \cr
0 & 1 & 2 
}\right]^{T} \\
A & = & \left[\matrix{b & m}\right]^{T}.
\eeqn

As a second example, consider fitting $e^{ax}$ to (0,1), (1,.5), and 
(2,.25).  To separate the coefficient, $a$, from the variable, $x$ we 
take the natural log of $y_{i}=e^{ax_{i}}$ to obtain $\ln(y_{i})=ax$.  
We can proceed as before now.

As a third example we will consider the second problem where we have 
noise (random errors) in the measurements.  These three examples are 
coded into Matlab by
\begin{list}{}{\leftmargin=3em}\item[]
\begin{verbatim}
Y=[1;2;3];
Ye=log([1;.5;.25]);
Yee=log([1;.5;.25]+.3*rand(3,1));
X=[0;1;2];
One=ones(3,1);
Phi=[One,X];
A=Phi\Y
norm(Y-Phi*A)
Ae=X\Ye
Aee=X\Yee
Xf=0:.05:2;
Yfe=exp(Ae.*Xf);
Yfee=exp(Aee.*Xf);
p1=[-.1,2.1];
p2=[-.1,3.1];
q=[0,0];
subplot(3,1,1)
plot(X,Y,'w*',X,Phi*A,'w-',p1,q,'w-',q,p2,'w-')
axis([p1,p2])
subplot(3,1,2)
plot(X,exp(Ye),'w*',Xf,Yfe,'w-',p1,q,'w-',q,p2,'w-')
axis([p1,p2])
subplot(3,1,3)
plot(X,exp(Yee),'w*',Xf,Yfee,'w-',p1,q,'w-',q,p2,'w-')
axis([p1,p2])
\end{verbatim}
\end{list}
and we get the output below and in Fig~\ref{llsqex}.
\begin{list}{}{\leftmargin=3em}\item[]
\begin{verbatim}
A =
    1.0000
    1.0000
ans =
    0
Ae =
   -0.6931
Aee =
   -0.4650
\end{verbatim}
\end{list}

\begin{figure}[h]
\begin{center}
\leavevmode
\hbox{
\epsfxsize=4in
\epsffile{LinLeastSq1.eps}}
\end{center}
\caption{Least Squares Example}
\label{llsqex}
\end{figure}

Homework 8.6: 1,3

\newpage
\chapter{Integration}\label{c-Integ}
The fundamental theorem of Calculus tells us that an integral of a 
function can be expressed in terms of the anti-derivative of the 
function.  Unfortuneately, not all functions have anti-derivatives 
that are expressable in known functions.  One of the most famous is 
the Gaussian probability distribution, which is given by
\beqn
e^{-\left(\frac{x-\mu}{\sigma}\right)^{2}}.
\eeqn
The anti-derivative of this important and frequently occuring function 
is unknown.  How do we handle it?  That is the subject of this chapter.

\section{Riemann}
We recall from Calculus that the integral is defined as
\beqn
\int_{a}^{b}{f(x)dx} = 
\lim_{n\rightarrow\infty}\sum_{j=1}^{n}f(p_{j})(x_{j}-x_{j-1}).
\eeqn
Now assume that all $n$ of the $x_{j}$ are evenly spaced on $[a,b]$.  We 
can then write
\beqn
h & = & \frac{b-a}{n} \\
  & = & x_{j}-x_{j-1}.
\eeqn
We can use this to get an expression for the Riemann Sum
\beqn
\int_{a}^{b}{f(x)dx}
 & = & 
\lim_{n\rightarrow\infty}\sum_{j=1}^{n}f(p_{j})(x_{j}-x_{j-1}) \\
 & = & 
\lim_{n\rightarrow\infty}\sum_{j=1}^{n}f(p_{j})h \\
 & = & 
\lim_{n\rightarrow\infty}h\sum_{j=1}^{n}f(p_{j}).
\eeqn
To evaluate the integral numerically we are not able to take the 
limit, so we get
\beqn
\int_{a}^{b}{f(x)dx} 
 & \approx & 
h\sum_{j=1}^{n}f(p_{j}).
\eeqn
The exact size of $n$ for the approximation to be good is a key aspect 
of numerical integration.  Note also that I have not specified what 
$p_{j}$ is, as this form allows you to do a left, right, mid-point, 
maximum, or minimum.  The basic idea here is that we are approximating 
the function by a constant on the interval.

\setlength{\lll}{\textwidth}
\addtolength{\lll}{-2\fboxsep}
\addtolength{\lll}{-2\fboxrule}
\noindent
\fbox{%
\begin{minipage}{\lll}

\beqn
\int_{a}^{b}{f(x)dx} 
 & \approx & 
h\sum_{j=1}^{n}f(p_{j}).
\eeqn

\end{minipage}}


\bibliographystyle{plain}
\bibliography{keith}

\end{document}
