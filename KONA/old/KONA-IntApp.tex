We will now look at the problem of finding a polynomial to fit a set 
of points.  The points could come from measurements in an experiment, 
or it could come from a complex function we want to approximate.  In 
either case we will begin by considering the case where we want our 
polynomial to be exact at these values.  An obvious question is why 
the emphasis on polynomials, when so many other functions exist.  
Indeed we do see the use of other basis (sin and cos in Fourier for 
example), but still polynomials hold a special place in many 
applications.  One major reason is the Theorem of Wiestrass from Real 
Analysis.  It basically says that polynomials can approximate any 
function (assuming you use the entire basis).

\section{Lagrange Interpolation Basis}
Probably the nicest way to visualize the interpolation polynomials is 
to consider the Lagrange interpolation basis functions.  For the set 
of points, $\{x_{0}, x_{1}, \ldots, x_{n}\}$ define the following polynomial:
\beqn
L_{i}(x)=\frac{\prod_{j\ne i}(x-x_{j})}{\prod_{j\ne i}(x_{i}-x_{j})}.
\eeqn
We note in particular that $L_{i}(x_{j})=\delta_{i,j}$, which allows 
us to get the interpolation polynomial nicely.  The interpolating polynomial 
is then given by
\beqn
P_{n}(x)=\sum_{i=0}^{n}y_{i}L_{i}(x).
\eeqn
The importance of the Lagrange basis giving us the Kronecker delta 
function cannot be over-emphasized, as it is the essential idea in 
getting the solution.

Often the points are selected to be evenly spaced due to constraints 
in the basic system.  While this is not the best for errors, it is 
often a physical necessity (for example many data samplers are 
constrained this way).  In this case we can simplify the expression 
using 
\beqn
\mu=\frac{x-x_{0}}{x_{1}-x_{0}}.
\eeqn
This is covered well in the book.

\section{Divided Difference}
Divided difference is a similar method to Taylor approximation but 
instead of matching derivatives exactly at a point, it nearly 
approximates the derivative to exactly match certain points.  The 
result is the same as Lagrange's formula.
\beqn
F[x_{0},x_{1}] & = & \frac{f(x_{1})-f(x_{0})}{x_{1}-x_{0}} \\
F[x_{0},x_{1},\cdots,x_{n}]
 & = & \frac{F[x_{1},\cdots,x_{n}]-F[x_{0},\cdots,x_{n-1}]}{x_{n}-x_{0}}
\eeqn

\beqn
P_{1}(x) & = & f(x_{0})+(x-x_{0})F[x_{0},x_{1}] \\
P_{k+1} & = & 
P_{k}+(x-x_{0})(x-x_{1})\cdots(x-x_{k})F[x_{0},x_{1},\cdots,x_{k+1}]
\eeqn

Homework:

Section 5.1: 5, 9, 13

Section 5.2: 2, 3, 7

\section{Error}
The key area to note from here is that the error is given by either 
of the following formulas.
\beqn
f(x)-P_{n}(x)
 & = & 
\prod_{i=0}^{n}(x-x_{i})\frac{f^{(n+1)}(c_{x})}{(n+1)!} \\
 & = & 
\prod_{i=0}^{n}(x-x_{i})F[x_{0},x_{1},\cdots,x_{n},x]
\eeqn
The important part of this is to note that these are themselves 
polynomials of order $n+1$.  Consider the plot of a polynomial with 
equi-spaced roots.  It is trivial to note that the height of the peaks 
between the roots is bigger towards the outside of the interval.
\section{Splines}
For splines we want to fit a cubic polynomial for each interval so 
that the first and second derivatives between two sections match on 
the boundary.  Following the books derivation we get the formula for 
the polynomial on the interval $[x_{j-1},x_{j}]$ to be
\beqn
s(x) & = & 
       a_{1}(x_{j}-x)^{3}+a_{0}(x-x_{j-1})^{3}
      +b_{1}(x_{j}-x)+b_{0}(x-x_{j-1}) \\
a_{i} & = & \frac{M_{j-i}}{6(x_{j}-x_{j-1})} \\
b_{i} & = & \frac{y_{j-i}-\frac{1}{6}M_{j-i}(x_{j}-x_{j-1})^{2}}{(x_{j}-x_{j-1})}
\eeqn
The only thing that we need is to calculate $M_{i}$ for the natural 
cubic spline, which is done by 
requiring $M_{1}=M_{n}=0$ and solving the following matrix system
\beqn
Ax & = & b \\
A & = & 
\left[\matrix{
\alpha_{2} & \beta_{2}  &  0        & \cdots       & 0 \cr
\beta_{2}  & \alpha_{3} & \beta_{3} & \ddots       & \vdots \cr
0          & \beta_{3}  & \ddots    & \ddots       & 0 \cr
\vdots     & \ddots     & \ddots    & \alpha_{n-2} & \beta_{n-2} \cr
0          & \cdots     & 0         & \beta_{n-2}  & \alpha_{n-1}
}\right] \\
x & = & 
\left[\matrix{
M_{2} \cr
\vdots \cr
M_{n-1}
}\right] \qquad 
b = 
\left[\matrix{
\gamma_{2}-\gamma_{1} \cr
\vdots \cr
\gamma_{n-1}-\gamma_{n-2}
}\right] \\
\alpha_{i} & = & \frac{x_{i+1}-x_{i-1}}{3}
\qquad
\beta_{i} = \frac{x_{i+1}-x_{i}}{6}
\qquad
\gamma_{i} = \frac{y_{i+1}-y_{i}}{x_{i+1}-x_{i}}
\eeqn
We can also find the $M_{i}$ for the not-a-knot cubic spline, which is 
often preferred by solving a similar system
\beqn
Ax & = & b \\
A & = & 
\left[\matrix{
\psi_{1}   & \beta_{1}  &  0         &  0         & \cdots       & 0            & 0\cr
\beta_{1}  & \alpha_{2} & \beta_{2}  &  0         & \cdots       & 0            & 0 \cr
0          & \beta_{2}  & \alpha_{3} & \beta_{3}  & \ddots       & \vdots       & \vdots \cr
0          & 0          & \beta_{3}  & \ddots     & \ddots       & 0            & 0 \cr
\vdots     & \vdots     & \ddots     & \ddots     & \alpha_{n-2} & \beta_{n-2}  & 0 \cr
0          & 0          & \cdots     & 0          & \beta_{n-2}  & \alpha_{n-1} & \beta_{n-1} \cr
0          & 0          & \cdots     & 0          & 0            & \beta_{n-1}  & \phi_{2}
}\right] \\
x & = & 
\left[\matrix{
M_{1} \cr
M_{2} \cr
\vdots \cr
M_{n-1} \cr
M_{n}
}\right] \qquad 
b = 
\left[\matrix{
\gamma_{1}-f'(x_{1}) \cr
\gamma_{2}-\gamma_{1} \cr
\vdots \cr
\gamma_{n-1}-\gamma_{n-2} \cr
f'(x_{n})-\gamma_{n-1}
}\right] \\
\alpha_{i} & = & \frac{x_{i+1}-x_{i-1}}{3}
\qquad
\beta_{i} = \frac{x_{i+1}-x_{i}}{6}
\qquad
\gamma_{i} = \frac{y_{i+1}-y_{i}}{x_{i+1}-x_{i}} \\
\psi_{1} & = & \frac{x_{2}-x_{1}}{3}
\qquad
\phi_{2} = \frac{x_{n}-x_{n-1}}{3}  
\eeqn
or (if you don't know the derivative) 
\beqn
Ax & = & b \\
A & = & 
\left[\matrix{
\psi_{1}   & \psi_{2}   &  0         &  0         & \cdots       & 0            & 0 \cr
\beta_{1}  & \alpha_{2} & \beta_{2}  &  0         & \cdots       & 0            & 0 \cr
0          & \beta_{2}  & \alpha_{3} & \beta_{3}  & \ddots       & \vdots       & \vdots \cr
0          & 0          & \beta_{3}  & \ddots     & \ddots       & 0            & 0 \cr
\vdots     & \vdots     & \ddots     & \ddots     & \alpha_{n-2} & \beta_{n-2}  & 0 \cr
0          & 0          & \cdots     & 0          & \beta_{n-2}  & \alpha_{n-1} & \beta_{n-1} \cr
0          & 0          & \cdots     & 0          & 0            & \phi_{2}     & \phi_{1}
}\right] \\
x & = & 
\left[\matrix{
M_{1} \cr
M_{2} \cr
\vdots \cr
M_{n-1} \cr
M_{n}
}\right] \qquad 
b = 
\left[\matrix{
\psi_{3} \cr
\gamma_{2}-\gamma_{1} \cr
\vdots \cr
\gamma_{n-1}-\gamma_{n-2} \cr
\phi_{3}
}\right] \\
\alpha_{i} & = & \frac{x_{i+1}-x_{i-1}}{3}
\qquad
\beta_{i} = \frac{x_{i+1}-x_{i}}{6}
\qquad
\gamma_{i} = \frac{y_{i+1}-y_{i}}{x_{i+1}-x_{i}} \\
\xi_{1} & = & x_{2}-x_{1}
\qquad
\xi_{2} = x_{2}-z_{1}
\qquad
\xi_{3} = z_{1}-x_{1} \\
\psi_{1} & = & \frac{\xi_{2}^{3}-\xi_{1}^{2}\xi_{2}}{6\xi_{1}}
\qquad
\psi_{2} = \frac{\xi_{3}^{3}-\xi_{1}^{2}\xi_{3}}{6\xi_{1}} 
\qquad
\psi_{3} = f(z_{1})-\frac{\xi_{2}y_{1}+\xi_{3}y_{2}}{\xi_{1}} \\
\xi_{4} & = & x_{n}-x_{n-1}
\qquad
\xi_{5} = x_{n}-z_{2}
\qquad
\xi_{6} = z_{2}-x_{n-1} \\
\phi_{1} & = & \frac{\xi_{5}^{3}-\xi_{4}^{2}\xi_{5}}{6\xi_{4}}
\qquad
\phi_{2} = \frac{\xi_{6}^{3}-\xi_{4}^{2}\xi_{6}}{6\xi_{4}}
\qquad
\phi_{3} = f(z_{2})-\frac{\xi_{5}y_{n-1}+\xi_{6}y_{n}}{\xi_{4}}
\eeqn

Note, you can easily enter the matrix $A$ into Matlab by using the 
command diag.  For instance, if you put the entries of $A$ that are on 
the main diagonal into the vector $A1$, the first sub-diagonal into 
$A2$, and the first super-diagonal into $A3$, then in Matlab you enter,
{\it A=diag(A1)+diag(A2,-1)+diag(A3,1);}.

Homework

section 5.3: 7
section 5.4: 3, 5

\newpage

\section{Least Squares Approximation}

Up till know we have dealt with interpolation, where we want to 
exactly match a set of points.  In reality, we are often more 
concerned with having a good overall approximation rather than an 
exact matching at a few points.  There are a lot of ways to 
approximate a function.  In general there are two main areas discrete 
and continuous.  We will cover the discrete case.  The continuous method 
involves some functional analysis and we do not have the time to 
cover it well.  If you are interested it can provide a fun project, 
and I have some good resources you can use.

We proceed with the discrete case.  The discrete case involves 
measuring the function to be approximated at a series of points, and 
then finding the best coefficients in some sense for some functions 
of interest.  

Some sense?  What do I mean by that?  Well, put simply, there are a 
variety of different methods of measuring how good an approximation 
is.  The standard method is the one we will concentrate on, and it is 
called least squares.  As with many things in Math, least squares 
owes its basis to Gauss.  The basic idea is to reduce the sum of the 
squares of the distances from the measurements to the function to be 
fitted at each of the x values.  The last point is very important 
because it is the basis of much of the problems in least squares.  In 
essence the answer you get is dependent on your choice of independent 
variables.  Below is an excerpt from my dissertation which covers what 
we are talking about now.  The key idea to get is that there are 
reasons to look beyond least squares.

Consider the problem of calibrating a gas thermometer.  Gas 
thermometers are based on Charles' law, which states that the volume 
of a fixed mass of gas at a fixed pressure is proportional to its 
temperature.  A simple gas thermometer can be made by trapping some 
gas with a mercury plug in a capillary tube that is open on only one 
end \bb{GenChem}.  The volume is thus proportional to the height of the 
plug.  The equation of the thermometer is thus $hc_{1}=T$, where $h$ is the 
height of the plug, $c_{1}$ is the constant we want to know, and $T$ is the 
absolute temperature.  We place the gas thermometer in a stirred liquid bath 
with a known thermometer.  We heat the bath and take height and temperature 
measurements at various times.  The LS solution gives us 
that $\hat c_{1}=h^{\dagger}T$, but we can see that this minimizes the 
error in the measured temperature, $T$, from the predicted temperature, 
$hh^{\dagger}T$.  By the same token we could use the relation $h=c_{2}T$, 
with $c_{2}=\frac{1}{c_{1}}$.  The LS solution, $\hat c_{2}=T^{\dagger}h$, 
thus minimizes the error between the measured height, $h$, and the predicted 
height $TT^{\dagger}h$.  A problem arises in the LS 
method in that generally $\hat c_{1}\ne\frac{1}{\hat c_{2}}$.  This 
can be seen easily in Figure~\ref{gastherm}.  The slope of the line designated 
temperature errors, is $\hat c_{1}$, while the slope of the line 
designated height errors is $\frac{1}{\hat c_{2}}$.  The line 
designated theoretical is the ``true'' system from which the estimates 
were generated.  It is easy to see that the slopes are not the same, 
and thus $\hat c_{1}\ne\frac{1}{\hat c_{2}}$.  The LS solution does 
not even perfectly handle the case where the system matrix is 
``known'', which gives us cause to be concerned as to how it will 
perform when there are perturbations to the system matrix.

\begin{figure}[h]
\begin{center}
\leavevmode
\hbox{
\epsfxsize=4in
\epsffile{gastherm.eps}}
\end{center}
\caption{Gas Thermometer Example}
\label{gastherm}
\end{figure}

The most well known alternative to least squares is total least 
squares (TLS).  In TLS we look at the perpendicular distance to the 
function.  This handles many of the problems of least squares but is 
more sensitive to errors, as it is ``optimistic'' in how it looks at 
the problem.  A huge body of literature is dedicated to this problem, 
and this is the central area of my dissertation.  While some of these 
other methods are very interesting, we will stick to least squares 
for the moment, but we will remember that problems can occur and so 
if we have problems we know there are things we can do.

Getting back to business we have a set of $m$ points $(x_{i},y_{i})$ and 
a group of $n$ functions $\phi_{i}(x)$ that we want to use to 
approximate the points with.  We thus have $m$ equations to find $n$ 
coefficients.
\beqn
y_{1} & - & \sum_{i=1}^{n}a_{i}\phi_{i}(x_{1}) \\
y_{2} & - & \sum_{i=1}^{n}a_{i}\phi_{i}(x_{2}) \\
& \vdots & \\
y_{m} & - & \sum_{i=1}^{n}a_{i}\phi_{i}(x_{m})
\eeqn
We can rewrite these into a matrix formulation, as
\beqn
Y-\Phi A \\
\eeqn
where 
\beqn
Y & = & \left[\matrix{y_{1} & y_{2} & \cdots & y_{m}}\right]^{T} \\
\Phi & = & \left[\matrix{
\phi_{1}(x_{1}) & \phi_{2}(x_{1}) & \cdots & \phi_{n}(x_{1}) \cr
\phi_{1}(x_{2}) & \phi_{2}(x_{2}) & \cdots & \phi_{n}(x_{2}) \cr
\vdots          & \vdots          & \ddots & \vdots \cr
\phi_{1}(x_{m}) & \phi_{2}(x_{m}) & \cdots & \phi_{n}(x_{m})
}\right] \\
A & = & \left[\matrix{a_{1} & a_{2} & \cdots & a_{m}}\right]^{T}.
\eeqn
At this point we want to minimize the square error which is what the 
2-norm does, so we have $\min_{A}\| Y-\Phi A\|_{2}^{2}$ .  The norm we 
are minimizing is called the 
cost function.  The solution is given by $A=\Phi^{\dagger}Y$, where 
$\Phi^{\dagger}$ is called the pseudo-inverse of $\Phi$.  Prove it?  
Sure!  To avoid getting into some deeper areas of linear algebra we 
will assume that $\Phi$ has linearly independent columns.  This is not 
restrictive, as we usually have a lot of measurements and only a few 
functions we want to fit to them ($m>>n$).

We recall from calculus that the minimum occurs when the gradient (derivative) 
is zero. We thus take the gradient of the cost with respect to $A$ and 
set it equal to zero to obtain 
\beqn
0 & = & 
\nabla_{A}\| Y-\Phi A\|_{2}^{2} \\
 & = & 
\nabla_{A}(Y-\Phi A)^{T}(Y-\Phi A) \\
 & = & 
-\Phi^{T}(Y-\Phi A) \\
 & = & 
\Phi^{T}\Phi A-\Phi^{T}Y \\
\Phi^{T}Y
 & = & 
\Phi^{T}\Phi A
\eeqn
The last line is what is referred to as the normal equation(s).  Note 
that some pluralize it to reflect that the single matrix equation 
reflects $n$ scalar equations.  I don't care, use what you like.  We 
note that if $\Phi$ has linearly independent columns, then 
$(\Phi^{T}\Phi)^{-1}$ exists.
\beqn
\Phi^{T}\Phi A
 & = & 
\Phi^{T}Y \\
A
 & = & 
(\Phi^{T}\Phi)^{-1}\Phi^{T}Y \\
A
 & = & 
\Phi^{\dagger}Y
\eeqn
You might wonder how the last step works.  Some might just call it a 
definition but in reality it is because $(\Phi^{T}\Phi)^{-1}\Phi^{T}$ 
satisfies the four conditions of a pseudo inverse (called the Penrose 
conditions).  
\begin{enumerate}
\item $\Phi\Phi^{\dagger}\Phi=\Phi$
\item $\Phi^{\dagger}\Phi\Phi^{\dagger}=\Phi^{\dagger}$
\item $\Phi\Phi^{\dagger}=(\Phi\Phi^{\dagger})^{T}$
\item $\Phi^{\dagger}\Phi=(\Phi^{\dagger}\Phi)^{T}$
\end{enumerate}
The properties are 
simple and easy to check, and yes, you have to check all four.  Many 
times a candidate matrix fails only one of them.  The first two 
properties tell us that it correctly maps the range spaces from the 
fundamental theorem of linear algebra, and the second two tell us 
the composite maps are symmetric.  The pseudo-inverse always exists 
and is unique.  Additionally, when the true inverse exists, it is the 
pseudo-inverse.  These are just a few of the many reasons to love the 
pseudo-inverse\ldots

The result is established.  The nice thing about how we have handled 
things here is we have not specified what the functions are (they have 
to be linearly independent but that is no problem) or how many of them 
we want to fit.  You can now fit any combination of functions you like.

As an example let's look at linear least squares for the points 
(0,1), (1,2), and (2,3).  We need to find the coefficients $m$, and $b$ 
for the line.  We construct our matrices
\beqn
Y & = & \left[\matrix{1 & 2 & 3}\right]^{T} \\
\Phi & = & \left[\matrix{
1 & 1 & 1 \cr
0 & 1 & 2 
}\right]^{T} \\
A & = & \left[\matrix{b & m}\right]^{T}.
\eeqn

As a second example, consider fitting $e^{ax}$ to (0,1), (1,.5), and 
(2,.25).  To separate the coefficient, $a$, from the variable, $x$ we 
take the natural log of $y_{i}=e^{ax_{i}}$ to obtain $\ln(y_{i})=ax$.  
We can proceed as before now.

As a third example we will consider the second problem where we have 
noise (random errors) in the measurements.  These three examples are 
coded into Matlab by
\begin{list}{}{\leftmargin=3em}\item[]
\begin{verbatim}
Y=[1;2;3];
Ye=log([1;.5;.25]);
Yee=log([1;.5;.25]+.3*rand(3,1));
X=[0;1;2];
One=ones(3,1);
Phi=[One,X];
A=Phi\Y
norm(Y-Phi*A)
Ae=X\Ye
Aee=X\Yee
Xf=0:.05:2;
Yfe=exp(Ae.*Xf);
Yfee=exp(Aee.*Xf);
p1=[-.1,2.1];
p2=[-.1,3.1];
q=[0,0];
subplot(3,1,1)
plot(X,Y,'w*',X,Phi*A,'w-',p1,q,'w-',q,p2,'w-')
axis([p1,p2])
subplot(3,1,2)
plot(X,exp(Ye),'w*',Xf,Yfe,'w-',p1,q,'w-',q,p2,'w-')
axis([p1,p2])
subplot(3,1,3)
plot(X,exp(Yee),'w*',Xf,Yfee,'w-',p1,q,'w-',q,p2,'w-')
axis([p1,p2])
\end{verbatim}
\end{list}
and we get the output below and in Fig~\ref{llsqex}.
\begin{list}{}{\leftmargin=3em}\item[]
\begin{verbatim}
A =
    1.0000
    1.0000
ans =
    0
Ae =
   -0.6931
Aee =
   -0.4650
\end{verbatim}
\end{list}

\begin{figure}[h]
\begin{center}
\leavevmode
\hbox{
\epsfxsize=4in
\epsffile{LinLeastSq1.eps}}
\end{center}
\caption{Least Squares Example}
\label{llsqex}
\end{figure}

Homework 8.6: 1,3
