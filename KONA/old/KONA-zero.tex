Almost every interesting problem in mathematics can be reduced to trying 
to find the zeros of a function.  The next several classes will be spent 
examining how we find zeros.  In general, you cannot explicitly solve for 
the zeros so you need to make iterative procedures to find them.  Today we 
will look at two methods: bisection and Newton's method.
\section{Bisection}
Bisection is a nice method in that it is guaranteed to converge and you 
can state exactly how many iterations it will take.


\section{Newton's Method}
Newton's Method essentially is an algebraic re-writing of the tangent line 
of a function at a point.

We can then use Taylor's formula to obtain an error bound.
\section{Secant}
Newton's Method requires the knowledge of the first derivative of the 
function.  Often the derivative is very complicated to evaluate and 
will take a long (relatively anyway) time to do so.  In many cases the 
first derivative may not be available.  In some cases it might not even 
exist at all points in the interval of interest.  Even when it is 
available it could be near zero which would cause numerical problems 
in evaluating it, even if it is in the region of convergence.  For 
all of these regions a new method was devised, which drew on the 
material leading up to calculus.  

Recall that the tangent line was 
found as the limit of a series of secant lines.  We can say that the 
derivative can thus be approximated by
\beqn
f(x)\approx\frac{f(x_{1})-f(x_{2})}{x_{1}-x_{2}}.
\eeqn
Thus if we know two points, we can approximate the funciton by a 
straight line between them and use the x-intercept as the next point 
to evaluate.  We now need two points instead of one and a 
derivative.  We refer to this as a two-point method because of the 
need of multiple points.  We will need two estimates to begin our 
evaluation.  Given two initial gueses, $x_{0}$ and $x_{1}$, the slope, 
$m$, is given by
\beqn
m=\frac{f(x_{1})-f(x_{0})}{x_{1}-x_{0}}.
\eeqn
Using this we find the next point, $x_{2}$ by using the point-slope 
form of a line
\beqn
f(x_{2})-f(x_{1}) & = & 
  \frac{f(x_{1})-f(x_{0})}{x_{1}-x_{0}}(x_{2}- x_{1}) \\
x_{2}- x_{1} & = & 
  \frac{x_{1}-x_{0}}{f(x_{1})-f(x_{0})}(f(x_{2})-f(x_{1})) \\
x_{2} & = & 
  x_{1}-f(x_{1})\frac{x_{1}-x_{0}}{f(x_{1})-f(x_{0})}. 
\eeqn
We thus have the equation for the next estimate:
\beq
x_{n+1} = 
  x_{n}-f(x_{n})\frac{x_{n}-x_{n-1}}{f(x_{n})-f(x_{n-1})}. \label{eq-sec1}
\eeq
Note that you can store the previous function evaluation and then you 
will not need to do two function evaluations per iteration.

Now we want to calculate the error.  To do this we will subtract 
eq~\ref{eq-sec1} from $\alpha=\alpha$.
\beqn
e_{n+1} & = & \alpha - x_{n+1} \\
  & = & \alpha -
    \left(x_{n}-f(x_{n})\frac{x_{n}-x_{n-1}}{f(x_{n})-f(x_{n-1})}\right) \\
  & = & \alpha -\frac{f(x_{n})x_{n-1}-f(x_{n-1})x_{n}}{f(x_{n})-f(x_{n-1})} \\
  & = & \frac{f(x_{n})(\alpha -x_{n-1})-f(x_{n-1})(\alpha -x_{n})}
       {f(x_{n})-f(x_{n-1})} \\
  & = & \frac{f(x_{n})e_{n-1}-f(x_{n-1})e_{n}}{f(x_{n})-f(x_{n-1})} \\
  & = & e_{n} e_{n-1}\frac{\frac{f(x_{n})}{e_{n}}-\frac{f(x_{n-1})}{e_{n-1}}}
       {f(x_{n})-f(x_{n-1})} \\
  & = & e_{n} e_{n-1}\frac{\frac{f(x_{n})}{e_{n}}-\frac{f(x_{n-1})}{e_{n-1}}}
       {x_{n}-x_{n-1}}\frac{x_{n}-x_{n-1}}{f(x_{n})-f(x_{n-1})} \\
  & \approx & e_{n} e_{n-1}\frac{\frac{f(x_{n})}{e_{n}}-\frac{f(x_{n-1})}{e_{n-1}}}
       {x_{n}-x_{n-1}}\frac{1}{f'(\alpha)}
\eeqn
We need to evaluate $\frac{f(x_{n})}{e_{n}}$, so we will use Taylor's 
Theorem for $f(x)$ evaluated at $\alpha$.  We find that
\beqn
\frac{f(x_{n})}{e_{n}} & = & 
\frac{f(\alpha)+(\alpha-x_{n})f'(\alpha)+\frac{1}{2}
  (\alpha-x_{n})^{2}f''(\alpha)+{\cal O}((\alpha-x_{n})^{3})}{e_{n}} \\
& = & 
\frac{e_{n}f'(\alpha)+\frac{1}{2}e_{n}^{2}f''(\alpha)
   +{\cal O}(e_{n}^{3})}{e_{n}} \\
& = & 
f'(\alpha)+\frac{1}{2}e_{n}f''(\alpha)+{\cal O}(e_{n}^{2})
\eeqn
Resuming our evaluation of $e_{n+1}$ we find
\beqn
e_{n+1} 
  & \approx & e_{n} e_{n-1}\frac
  {f'(\alpha)+\frac{1}{2}e_{n}f''(\alpha)+{\cal O}(e_{n}^{2})
  -f'(\alpha)-\frac{1}{2}e_{n-1}f''(\alpha)+{\cal O}(e_{n-1}^{2})}
       {x_{n}-x_{n-1}}\frac{1}{f'(\alpha)} \\
  & = & e_{n} e_{n-1}\frac{\frac{1}{2}e_{n}f''(\alpha)
   -\frac{1}{2}e_{n-1}f''(\alpha)+{\cal O}(e_{n-1}^{2})}
       {x_{n}-x_{n-1}}\frac{1}{f'(\alpha)} \\
  & = & e_{n} e_{n-1}\frac{\frac{1}{2}(e_{n}-e_{n-1})f''(\alpha)
   +{\cal O}(e_{n-1}^{2})}{x_{n}-x_{n-1}}\frac{1}{f'(\alpha)} \\
  & = & e_{n} e_{n-1}\frac{\frac{1}{2}(x_{n}-x_{n-1})f''(\alpha)
   +{\cal O}(e_{n-1}^{2})}{x_{n}-x_{n-1}}\frac{1}{f'(\alpha)} \\
  & = & e_{n} e_{n-1}(\frac{1}{2}f''(\alpha)
   +{\cal O}(e_{n-1}^{2}))\frac{1}{f'(\alpha)} \\
  & \approx & e_{n}e_{n-1}\frac{f''(\alpha)}{2f'(\alpha)} \\
  & \approx & e_{n}e_{n-1}M.
\eeqn
This is similar to Newton's method which suggests that
\beqn
  e_{n+1}=Ae_{n}^{c},
\eeqn
which implies
\beqn
  e_{n}=A^{-1}e_{n-1}^{c^{-1}}.
\eeqn
Substituting and collecting terms we find
\beqn
B=e_{n}^{1-c+c^{-1}}.
\eeqn
Since the left hand side is a constant the exponent must be zero, or 
$c$ must be the golden ratio.
This implies that the secant method converges superlinearly.
\section{Regula Falsi}


\section{Fixed Points}
A fixed point is a point in the domain of a function, which maps its 
domain back into its domain, that satisfies $\alpha=C(\alpha)$.  
Since $\alpha$ does not change when it is mapped by the function it is 
fixed, hence the name.  We need to look at what 
the idea that underlies fixed points: contractions.  A contraction 
$y=C(x)$, is a mapping from a closed interval in $X$ into another closed 
interval in $Y$ with the property that for some $b=C(a)$ (usually $a$ 
and $b$ are both the origin but it is not required), $\|\cdot\|_{x}$ 
a norm on $X$, and $\|\cdot\|_{y}$ a norm on $Y$ we have:
\beqn
\| x-a\|_{x} > \| y-b\|_{y} = \| C(x)-C(a)\|_{y}
\eeqn
for all $x\in X$ and $y\in Y$.  Usually we have $X$ and $Y$ are $\Re$ 
and $a=b$, which gives us that $|x-a|>|C(x)-a|$.  Take the derivative of both 
sides and we see 
\beqn
1>|C'(x)|.
\eeqn
This brings up a key point, we must have that the magnitude of the 
function's slope is less than 1.  If you think about this it makes 
sense, as for slope magnitudes greater than one there will be growth 
and we are looking a funcions which shrink things.  While this is a 
simple idea, it has many profound implications.  The book proves 
nicely how the uniqueness of solution, convergence, etc..  One thing 
that should be highlated has to do with rate of convergence.  Given a 
contraction defined on an interval $[a,b]$ with some point, $\alpha = 
C(\alpha)\in[a,b]$ called a fixed point, we can define the iteration 
$x_{n+1}=C(x_{n})$.  We then have (using the mean value theorem)
\beqn
\alpha-x_{n+1} 
 & = & C(\alpha)-C(x_{n}) \\
 & = & C'(d)(\alpha-x_{n}) \\
|\alpha-x_{n+1}| 
 & < & |\alpha-x_{n}|.
\eeqn
We have linear convergence from this.  Consider the following paradox.

Let a function $g(x)$ be defined by
\beqn
g(x)=x-\frac{f(x)}{f'(x)}
\eeqn
and let $f(x)$ have a single root in some interval $[a,b]$.  From the 
book we know this must have a fixed point in the interval and the 
iteration $x_{n+1}=g(x_{n})$ will converge to the fixed point.  This 
method thus has linear convergence from what we have proven above.  
This iteration is Newton's Method though, so it has Quadratic 
convergence.  What gives?  The convergence of a fixed point algorithm 
is at least linear but it can be better if $C'(\alpha)=0$.  Notice 
that the derivative of $g(x)$ is given by
\beqn
g'(x) & = & 1-\frac{(f'(x))^{2}-f(x)f''(x)}{(f'(x))^{2}} \\
      & = & \frac{f(x)f''(x)}{(f'(x))^{2}}.
\eeqn
Note that for $x=\alpha$ we trivially have that $g'(\alpha)=0$, which 
satisfies our requirement for faster convergence.

How can I get a function $g(x)$ that satisfies the requirements?  
Many ways exist but consider the following.  For a function $f(x)$ with 
a zero at $x=\alpha$ in an interval $[a,b]$, that has 
$\beta=\max_{x\in[a,b]}|f'(x)|$, we define the iteration
\beqn
x_{n+1}=x_{n}-\gamma sign(f'(x_{n}))f(x_{n})
\eeqn
with $0<\gamma\beta<2$.  We then see that
\beqn
g(x) & = & x-\gamma sign(f'(x))f(x) \\
g'(x) & = & 1-\gamma sign(f'(x))f'(x) \\
      & = & 1-\gamma |f'(x)|
\eeqn
and thus $1>g'(x)>-1$.  Note that if we choose $\gamma$ such that 
$0<\gamma\beta<1$ then $1>g'(x)>0$ and the sequence 
$\{x_{i}\}_{i=0}^{\infty}$ converges to $\alpha$ from one side (no 
alternating).  The parameter $\gamma$ is refered to as the {\bf step 
size}.  As a final note, we can use Aitken's $\Delta^{2}$ method as 
outlined in the book to refine the estimate $x_{n}$.  Replace $\alpha$ 
with $\hat{x}_{n}$ and you have a refinement and acceleration method 
that will work on any linearly convergent algorithm.  It can thus be 
used on general fixed point methods.

Homework 
4.3: 6, 13
\newpage
\section{Continuation Methods}
One of the essential problems in root finding is to find a good place 
to start.  We have spoken about the progressively doubling intervals 
till we find a sign change.  I mentioned this was not the fastest or 
best, but would work.  I wanted to give you what I think is one of the 
best.  It is refered to as a continuation method or sometimes a 
homotopy.

A homotopy, $h$, is a continuous connection between two functions, $f$ and 
$g$, that maps one space, $X$, to another, $Y$:
\beqn
h:[0,1]\times X\rightarrow Y
\eeqn
such that $h(0,x)=g(x)$ and $h(1,x)=f(x)$.  Two simple homotopies we will 
use are listed below.
\begin{enumerate}
\item
\beqn
h(\lambda,x)=\lambda f(x)+(1-\lambda)g(x)
\eeqn
\item
\beqn
h(\lambda,x) & = & \lambda f(x)+(1-\lambda)(f(x)-f(\alpha_{0})) \\
             & = & f(x)-(1-\lambda)f(\alpha_{0})
\eeqn
\end{enumerate}
The first one is the most general.  Assume we want to find the roots 
of $f$, but we know the roots of $g$.  By picking a sequence of 
$\lambda$ values from zero to one, we will slowly make the roots move 
from the known positions of $g$ to the unknown positions of $f$.  We 
usually try to pick $g$ so it has the same number of roots as the 
function $f$.

The second method is a frequently used one if I don't want to find a 
function $g$.  We are in essence biasing the original fucntion so that 
at $\alpha_{0}$ the homotopy has a root for $\lambda=0$.  This gives a nice 
starting point.  The following theorem tells us when this will work.

\begin{theorem}[Ortega and Rheinboldt]
If $f:\REn\rightarrow\REn$ is continuously differentiable and if 
$\|[f'(x)]^{-1}\|\leq M$ on $\REn$, then for any $\alpha_{0}\in\REn$ there 
is a unique curve $\{\alpha(\lambda):0\leq\lambda\leq 1\}$ in $\REn$ such 
that $f(\alpha(\lambda))-(1-\lambda)f(\alpha_{0})=0$, with $0\leq\lambda\leq 
1$.  The function $\lambda\mapsto \alpha(\lambda)$ is a continuously 
differentiable solution to the initial value problem 
$\alpha'=-[f'(\alpha)]^{-1}f(\alpha_{0})$, where $\alpha(0)=\alpha_{0}$.
\end{theorem}

Essentially this tells us if $f$ is smooth and the first derivative 
doesn't get too close to zero then you can use this start one of our 
rootfinding methods, for instance Newton's Method.  Often this method 
is solved by using a numerical integration technique which we will 
cover in a few weeks.  For instance if Euler's method is used then it 
turns out to generate Newton's Method in $\lambda$!

\section{Multiple Roots}
One thing that always caused us problems in all our methods is 
multiple roots.  I will present a simple technique for handling this 
case.  Let our function, $f$, with root of multiplicity, $2$, be given 
by
\beqn
f(x)=(x-\alpha)^{2}f_{1}(x),
\eeqn
where $f_{1}$ has no root at $\alpha$.  Take the derivative of $f(x)$ 
to obtain
\beqn
f'(x) & = & (x-\alpha)f_{1}(x)+(x-\alpha)^{2}f'_{1}(x) \\
      & = & (x-\alpha)(f_{1}(x)+(x-\alpha)f'_{1}(x)) \\
      & = & (x-\alpha)f_{2}(x),
\eeqn
where $f_{2}(x)$ has no root at $\alpha$.  We now have a funcition 
with a single root at the same place that the original function had a 
double root.  This can be done for higher multiplicity roots, and 
does not require knowing $\alpha$ as we are taking the derivative then 
finding $\alpha$ using one of our techniques.

\section{Sensitivity}
This is refered to as stability of the roots in the books, but it is 
more closely related to the sensitivity of a differential equation to 
perturbations in its coefficients.  For instance, consider a famous 
problem due to Wilkinson.
\begin{problem}[Wilkinson]
Find the roots of the polynomial $f(x)$ given by
\beqn
f(x) & = & (x-1)(x-2)\cdots(x-20) \\
     & = & x^{20}-210x^{19}+\cdots +20!
\eeqn
The roots are clearly one through twenty.  Perturb the coefficient 
$-210$ to $-210-2^{-23}$.  The change is in one coefficient only, and 
that in the $7^{th}$ decimal place.  The roots are now
\bt{l c c}
$1.000000000$ & $6.000006944$ & $10.095266145\pm 0.643500904j$ \\
$2.000000000$ & $6.999697234$ & $11.793633881\pm 1.652329728j$ \\
$3.000000000$ & $8.007267603$ & $13.992358137\pm 2.518830070j$ \\
$4.000000000$ & $8.917250249$ & $16.730737466\pm 2.812624894j$ \\
$4.999999928$ & $20.846908101$ & $19.502439400\pm 1.940330347j$ 
\et
The problem is not roundoff.  The roots of high-order coefficients can 
be extremely sensitive to changes in the coefficients.  This is a 
problem particularly when the coefficients are experimentally 
determined.
\end{problem}
