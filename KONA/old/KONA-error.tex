\section{Taylor Polynomials}
We want an easier way of calculating a difficult function.  To this end we 
want to find a function that is similar to our original that we can 
calculate.  Taylor polynomials are one such type of functions with an easy 
calculation and intuition.  To find the Taylor polynomials we match the 
derivatives of the two polynomials at a particular point.  We are in 
essence enforcing a smoothness criterion at the point of interest.

Thus the general expression for the Taylor series is:

Example
Problem 1.1-3(c) 


Example
Problem 1.1-8
 

Remainder
The Taylor Series obviously has errors in its approximation.  If the 
original function is in $C_{n+1}$ on the interval _�x�_ (with $a$ in the 
interval) then the remainder (or error) is given by

with cx between a and x.  To get an error bound we assume that cx is the 
worst possible.
Example
Problem 1.2-3(a)
In this case n=1 so the worst case would be if cos(c) were -1.

Example
Prove problem 8.
Multiplying Polynomials
Straightforward:
a1*x*x*...*x
This takes k multiplications for a monomial of size k. so for a polynomial 
with monomials up to size n it would take n(n+1)/2 multiplications.
Storing:
Calculate x2=x*x, x3=x*x2, etc.
This takes 2n-1 multiplications.

Nesting:
b0=a0+xb1
b1=a1+xb2
bn=an
Each step takes 1 multiply so this method takes only n multiplications.

The real savings come when you have to calculate a large polynomial many 
times.


Binary
In any number system, the position of a digit relative to the decimal 
place specifies the integer power of the base we must multiple the digit 
by to get its value.  So for base 10
,
and for base 2
.
This gives us one way to convert numbers.  For instance, we can convert 
binary to decimal by expanding the binary number in this way.  Thus using 
the above to convert binary (10.01) to decimal we find,

Note that the "2" we are using is the base of binary in decimal form, and 
this is why we went from binary to decimal.  In binary, its form would be 
"10" and ten would be "1010". Therefore, we could go to binary by, 
expanding this out with ten in binary.  The problem with this method is it 
is clumsy to use since we do not do squaring, cubing, etc. easily in base 
2.  Another problem is that 0.1 is an infinitely repeating decimal in 
binary so it is a pain to deal with 10-1!  Instead, we convert decimal to 
binary as follows.  
1) Split your number into a.b
2) For the whole number part (a)
a) Divide 2 into a and note the quotient and remainder as q1,r1 (a=2*q1+r1)
b) As long as the quotient from above is not zero, divide it by 2 and 
record the quotient and remainder as qi,ri (with i denoting the current 
step).  Repeat.
c) The binary equivalent of a is rnrn-1...r2r1.  Basically we have done 
our nested polynomial evaluation backwards with x=2, and the coefficients 
being the remainders.
3) For the fractional part (b)
a) Multiply 2*b, and record the unit value as a1.  Denote b-a1=b1.
b) If bi does not equal zero, multiply it by 2, denoting the units digit 
by ai+1 and the difference bi-ai+1=bi+1.  Repeat until the difference is 
zero (this may never happen so be looking for patterns to get repeating 
fractions).
c) The fractional part, b, is a1a2a3a4...
4) The full answer is thus rnrn-1...r2r1.a1a2a3a4...
Hexadecimal
This is often made to sound more intimidating than it is.  Hexadecimal 
numbers are simply base 16, but this can be handled nicely since 24=16.  
All you have to do is group binary digits into groups of 4 and use the 
conversion table:
Bin
Hex
Dec
Bin
Hex
Dec
0000
0
0
1000
8
8
0001
1
1
1001
9
9
0010
2
2
1010
A
10
0011
3
3
1011
B
11
0100
4
4
1100
C
12
0101
5
5
1101
D
13
0110
6
6
1110
E
14
0111
7
7
1111
F
15
Floating point numbers
While the book discusses single precision numbers, they are essentially 
never used, as double precision is so much better and readily available.  
We will assume IEEE double precision floating point representation, as it 
is the standard.  IEEE floating point numbers have the form
,
where


Single Precision
Double Precision
P
24
53
Emin
-126
-1022
Emax
127
1023
Bias
127
1023
Thus IEEE is represented in memory as a sign bit, exponent bits (8 or 11), 
and mantissa bits (23 or 52).  The mantissa is composed of all the bj.  A 
few things to note about IEEE arithmetic.
1. The exponent stored is E=e-Bias
2. �0 is encoded by Emin-1 and f=0
3. Denormalized numbers are encoded by Emin-1 and f�0
4. �_ is encoded by Emax+1 and f=0
5. NAN is encoded by Emax+1 and f�0
Approximating the Reals
To approximate the real number x, we define the function fl(x) as, 0 when 
x=0, and the nearest element in floating point to x otherwise.  Finding 
nearest elements requires a rounding scheme (rounding or 
"chopping"/truncating) and a tie breaker procedure (usually round away 
from zero).

Bounding Errors
To bound the error in approximating the real number x, we need to consider 
the floating point number, fl(x), used to approximate x.  First we note 
that a real number x, is written in binary as
,
where s is the sign, f has as many digits as needed, and e is any 
integer.  Note that e will be different for IEEE, which normalizes to 
1�fr<2 with an implicit 1 at the start; than the non-standard forms, which 
normalize to 0.5� fr<1 with no assumed leading 1.  We will assume that e 
is within the permitted bounds for simplicity.  The floating-point 
representation is 
.
We can now write the difference as

For the moment, we will consider the difference (fr-f).  Note that we are 
dealing with normalized numbers with n bits of accuracy and an implicit 
leading 1 (IEEE arithmetic), while the book deals with numbers normalized 
between a half and one, with no implicit 1, so for us
.
Note that the digit to the left of the decimal in f is assumed to be 1, 
the only exception is when fr=1.1111... which would have f=10.000...0.  
Technically it would actually have f=1.000...0 and the exponent would be 
(e+1) but since we are keeping the exponent e we keep the simplification.  
Note that this is equivalent to rounding 9.5 to 10.  Anyway, our real 
concern is the worst case of the difference, which is in all cases given by
.
Note that the 1 is in the (n+1)st place after the decimal.  We rewrite 
this using floating point notation as
.
We now stick this back into the expression for the difference between x 
and fl(x) and obtain an upper bound by taking absolute value
.
Similarly to get a lower bound we take the negative of the absolute value, 
and find

Now we note that the size of x is
.
For the book's form of the mantissa, we would have
.
The relative error is thus
.
Matlab Programming
There are two basic ways to interact with Matlab: command line execution, 
and M-files. Yes there are others such as MEX-files, Simulink, and several 
interfacing programs, but they are not relevant to us.  
We will primarily be concerned with the use of M-files, because they are 
the most helpful.  Command line execution is really just for quick 
operations and checking of segments of code.  Matlab syntax is a high 
level programming language that interacts with a series of numerical 
libraries (most notably LinPack, EisPack, and BLAS).  Like most 
programming languages we have two types of programs that can be written.  
A regular program, which is written as you would type commands on the 
command line, is the most basic type and is often the way you will start 
homework problems and other projects.  Functions, which are sub-programs 
called by another program (even recursively by other functions), are 
probably the most useful, as they allow you to extend the language by 
defining new operations.  One of the main goals of this class is for you 
to walk away with a library of Matlab functions that you can use to do a 
variety of tasks.  So how do you specify which you want?  You will get a 
regular program unless you start the M-file with the command function.  
The syntax is
function a=name(x,y,..., z)
or
function [a,b,...,c]=name(x,y,..., z)
The second form returns multiple values.
Matlab gives us several command structures also: for, while, and 
if-elseif-else.  To see how these work lets use the programs I passed out 
last time as an example.

Homework:
Convert the Fortran program in 3.1 into Matlab syntax.
Do problems 9, 13, 14 from section 3.1

Propagation of Error
We have seen that representing the real numbers on a computer involves 
errors.  When we use floating point numbers in a calculation rather than 
the actual numbers the errors can grow.  The errors caused by using 
floating point approximations are called propagated errors.  Two ways of 
bounding propagation errors exist.  The forward method involves explicitly 
calculating the errors and is called interval arithmetic.  The backward 
method involves finding a condition number, which gives a bound on how big 
the error can grow.
Interval Arithmetic
Let's consider the error in a computation between the true values (xT, yT) 
and the approximate values (xA, yA).  We only know the approximate values 
and the error bounds

Note that the error is could be positive or negative so we must consider 
the positive and negative bounds.  First, we will look at the error for 
addition or subtraction.

Now let's consider multiplication.  

It is easy to see that this can quickly become very hard to deal with.  
Consider for instance multiplying two n-by-n matrices, which would involve 
$n^{3}$ multiplies.  Keeping track of all of them would rapidly become 
impossible.  We will consider one final operation, namely division.

Again we can see that things can become very complicated quickly.
Condition Number
We will now consider the problem of evaluating a function, f(x), at an 
approximate rather than true value.  To do this we will require our 
function to be continuous on [xT,xA] and differentiable on (xT,xA).  We 
can thus use the mean value theorem to see

We now note that since c is between the true and approximate values, and 
that the interval is on the order of 10-16 for IEEE double-precision 
arithmetic.  We can thus assume c is approximately xA.

The derivative of f(x) at xA, is called the condition number and shows how 
the error of the approximation will influence the error of the 
calculation.  The condition number is nice in that it cleanly handles the 
error bounds.  It is not as precise as the error in the interval 
arithmetic, but it is tractable even for large matrix operations, which 
will involve the norms of the matrices rather than the elements.  Quite a 
savings!
Sums
We have spoken a lot about summation, but we want to look at one final 
area of sums before we move on.  Consider the following summation:

In real numbers it doesn't matter if we add the 45's first or the 100000.  
In floating point numbers it does matter!  Floating point numbers are not 
associative.  To see this consider a 4 decimal place accuracy machine that 
uses rounding, and is nicely implemented.  In this case we see that 
100000+45=100000
so if we add as stated we find the sum is 100000 for the series (rather 
than 100180).  If we add the 45's first we find that 
45+45+45+45=180.
Then 
100000+180=100200.
A much better result.  These sums occur in a variety of places, from 
standard series, to evaluating integrals, to inner products of vector, and 
matrix multiplication.  In short you should be aware of the lack of the 
associative property.

