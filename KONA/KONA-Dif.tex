\chapter{Differentiation}

Numerical differentiation seems obvious.  In most introductions to calculus, the derivative is introduced as the limit of a series of secant lines.  Probably the most basic method of numerically taking the derivative is based off this formula.  It turns out that while it seems obvious, numerical derivatives are more difficult as we will see.

\section{Derivative Approximation}
Let's start at the basics, the two point method of obtaining the derivative is based off taking the derivative of the two point interpolation polynomial.
\beqn
f(x) & \approx &
   \frac{x_{1}-x}{h}f(x_{0})+\frac{x-x_{0}}{h}f(x_{1}) \\
f'(x) & \approx &
   \frac{-1}{h}f(x_{0})+\frac{1}{h}f(x_{1}) \\
 & = &
   \frac{f(x+h)-f(x)}{h}.
\eeqn
This is only exact when we take the limit as $h$ approaches zero.  We can see how the basic errors are dealt with by expanding the Taylor sequence.
\beqn
f(x+h)=f(x)+hf'(x)+\frac{h^{2}}{2}f''(c)
\eeqn
Using the Taylor series in the formula we obtain that the error in the
formula is
\beqn
E & = & f'(x)-\frac{f(x+h)-f(x)}{h} \\
& = & f'(x)-\frac{f(x)+hf'(x)+\frac{h^{2}}{2}f''(c)-f(x)}{h} \\
& = & f'(x)-\frac{hf'(x)+\frac{h^{2}}{2}f''(c)}{h} \\
& = & f'(x)-f'(x)+\frac{h}{2}f''(c) \\
& = & \frac{h}{2}f''(c).
\eeqn
Two other types of error occur.  First, there is subtractive difference errors.  We note that since $f(x)\approx f(x+h)$, when we
subtract them we will lose precision.  Second, there is error propagation.  Consider the fact that nothing we do in the real world is ever exact, so we actually measure and calculate a nearby solution:
\beqn
f(x) & = & \tilde{f}(x)+\epsilon_{1} \\
f(x+h) & = & \tilde{f}(x+h)+\epsilon_{2}.
\eeqn
Substituting this into our formula we find
\beqn
f'(x) & \approx &
   \frac{f(x+h)-f(x)}{h} \\
 & = &
   \frac{\tilde{f}(x+h)+\epsilon_{2}-\tilde{f}(x)+\epsilon_{1}}{h} \\
 & = &
   \frac{\tilde{f}(x+h)-\tilde{f}(x)}{h}+\frac{\epsilon_{2}-\epsilon_{1}}{h}.
\eeqn
The resulting error is
\beqn
E & = & f'(x)-\frac{f(x+h)-\epsilon_{2}-f(x)+\epsilon_{1}}{h} \\
& = & f'(x)-\frac{f(x)+hf'(x)+\frac{h^{2}}{2}f''(c)-\epsilon_{2}-f(x)+\epsilon_{1}}{h} \\
& = & f'(x)-\frac{hf'(x)+\frac{h^{2}}{2}f''(c)-\epsilon_{2}+\epsilon_{1}}{h} \\
& = & f'(x)-f'(x)+\frac{h}{2}f''(c)+\frac{\epsilon_{1}-\epsilon_{2}}{h} \\
& = & \frac{h}{2}f''(c)+\frac{\epsilon}{h}.
\eeqn
As long as the last term is non-zero, which it will be in general, if $h$ gets to small them the propagation errors dominate, see Figure~\ref{f-derivative_error}.  This is the real problem.  We did not have this problem with numerical integration because we were multiplying by $h$ rather than dividing by it.  The problem is more pronounced with higher order derivatives where you will be dividing by powers of $h$.  This problem motivates the use of integral equations rather than differential ones.  Unfortunately many integral equations are very ill-conditioned.  Anyway, the key idea is that there is a competition between several error types.


\begin{figure}
 \begin{center}
  \begin{lpic}{derivative_error(.5,.5)}
   \lbl[l]{2,75,90;Error}
   \lbl[b]{110,2,0;h}
  \end{lpic}
  \caption{Error in calculating the step size for different step sizes, $h$, for $f''(c)=2$ and $\epsilon=10^{-6}$.}\label{f-derivative_error}
 \end{center}
\end{figure}

\section{Tangent Line}

As of yet we have considered calculating the derivative from above.  Let us consider the derivative from below, and we will solve for it using the equation of a line and noting that the slope of the tangent line in the limit as $h=0$ is the derivative\footnote{We could have just used the same method we used in the last section, but what fun is that?}.
\beqn
y &=& mx+b \\
\eeqn
thus
\beqn
f(x) &=& mx+b \\
f(x-h) &=& m(x-h)+b.
\eeqn
Take the difference
\beqn
f(x)-f(x-h)
 &=& mx - m(x-h) \\
 &=& mh \\
m_{left} &=& \frac{f(x)-f(x-h)}{h}
\eeqn
This is the left approximation of the derivative and
\beqn
m_{right} &=& \frac{f(x+h)-f(x)}{h}
\eeqn
is the right approximation.  We get a better estimate by averaging them.
\beqn
m_{avg}
 &=&\frac{m_{left}+m_{right}}{2} \\
 &=&\frac{\frac{f(x)-f(x-h)}{h}+\frac{f(x+h)-f(x)}{h}}{2} \\
 &=&\frac{f(x)-f(x-h)+f(x+h)-f(x)}{2h} \\
 &=&\frac{f(x+h)-f(x-h)}{2h}
\eeqn
Interestingly the best thing we can do when we have three points is not to use the one we are trying to estimate the derivative of. The reason is, when we use the left approximation of the derivative, we are really estimating the derivative at $x-\frac{h}{2}$, and similarly for the right approximation.  By averaging we get an estimate of the derivative at $x$.

\section{Richardson Extrapolation}



\section{Higher Order Derivatives}

The second derivative is just the derivative of the first derivative.  Since the left and right approximations really give us the approximate derivative at $x\pm\frac{h}{2}$, we can use them, with their separation distance of $h$, to estimate the second derivative at $f(x)$.
\beqn
f''(x)
 &\approx& \frac{f'\left(x+\frac{h}{2}\right)-f'\left(x-\frac{h}{2}\right)}{h} \\
 &\approx& \frac{\frac{f(x+h)-f(x)}{h}-\frac{f(x)-f(x-h)}{h}}{h} \\
 &\approx& \frac{f(x+h)-2f(x)+f(x-h)}{h^2}
\eeqn
To solve for higher derivatives we will need more points: $f(x+2h)$, $f(x-2h)$, etc.  The further we go away the worse the approximation will be though, again showing that higher order derivatives become more inaccurate.

\subsection{Method of Undetermined Coefficients}

In addition to the point the derivative is to be evaluated at, we need one additional point per order of the derivative to be taken.  That is a first derivative takes 2 points, a second derivative takes 3 points, a third derivative takes 4, and so on.  Rather than memorize a bunch of formulas for higher powers, or even iterating the formula as above, we can use the method of undetermined coefficients, which can also be used for integration, interpolation, differential equations, etc.

Say we wanted to find the formula for the second derivative, which we know will require 3 points.
\begin{eqnarray*}
f''(x)=Af(x+h)+Bf(x)+Cf(x-h)
\end{eqnarray*}
We will use the Taylor Series to expand $f(x+h)$ and $f(x-h)$ around $x$.
\begin{eqnarray*}
f(x+h)=f(x)+hf'(x)+\frac{h^2}{2}f''(x)+hot \\
f(x-h)=f(x)-hf'(x)+\frac{h^2}{2}f''(x)+hot
\end{eqnarray*}
We now substitute this into the original expression.
\begin{eqnarray*}
f''(x)
&=&A\left(f(x)+hf'(x)+\frac{h^2}{2}f''(x)+hot\right)+Bf(x)+ \\
&&\qquad C\left(f(x)-hf'(x)+\frac{h^2}{2}f''(x)+hot\right) \\
&=& (A+B+C)f(x)+(A-C)hf'(x)+(A+C)\frac{h^2}{2}f''(x)+hot
\end{eqnarray*}
Equating like derivatives on the left and right we have

\begin{tabular}{ccc}
Derivative & Equation               & Implication \\\hline
$f(x)$     & $A+B+C=0$              & $B=-(A+C)$ \\
$f'(x)$    & $(A-C)h=0$             & $A=C$ \\
$f''(x)$   & $(A+C)\frac{h^2}{2}=1$ & $A=\frac{1}{h^2}$ \\
\end{tabular}

thus
\begin{eqnarray*}
f''(x)&=&\frac{1}{h^2}f(x+h)+\frac{-2}{h^2}f(x)+\frac{1}{h^2}f(x-h)\\
      &=&\frac{f(x+h)-2f(x)+f(x-h)}{h^2}.
\end{eqnarray*}
Note that we found the same formula a different way.
