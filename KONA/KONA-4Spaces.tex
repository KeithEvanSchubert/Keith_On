\chapter{Four Fundamental Spaces}

\section{Definition}
\subsection{Domain}
\subsection{Range}
\subsection{Right Nullspace}
\subsection{Left Nullspace}
\subsection{Putting Them Together}

\section{SVD - Singular Value Decomposition}


\section{Inverse}

\subsection{Pseudo Inverse}
left inverse

right inverse

Moore-Penrose conditions

\subsection{Inverse}
A true inverse only exists when the matrix is square and full rank, in which case the matrix is said to be invertible.

\section{Projections}
\subsection{Properties}
Let $P_a$ be a projection onto the range of $a$.  If this is the case then projecting $a$ onto itself should not change anything.  Thus if I take any vector and project it onto the range of $a$ then project it onto the range of $a$ again, it should be the same as projecting onto the range of $a$ once.  Putting this formally,
\begin{eqnarray}
P_a(P_a x) & = & P_a x, \qquad\forall x\\
P_a^2 x & = & P_a x, \qquad\forall x\\
P_a^2 & = & P_a
\end{eqnarray}
When $P^2=P$, we say the matrix $P$ is idempotent, thus all projectors are idempotent.

If a projector is to be orthogonal it also has to be symmetric, since orthogonal projectors should provide an orthogonal basis for the space they project onto.

\subsection{Using SVD}
Now that we feel comfortable with the SVD, let's use it to solve a practical problem.