\chapter{Using SciLab}

\section{Basics}
When you first start SciLab you will see something like
\begin{verbatim}
                   ==========
                    scilab-2.7.2
        Copyright (C) 1989-2003 INRIA/ENPC
                    ==========



Startup execution:
  loading initial environment

-->
\end{verbatim}
The arrow "$-$$-$$>$" is the command prompt.  SciLab, like MatLab is a command line interface to a mathematics programming environment.  To get started lets do a calculation.
\begin{verbatim}
3+(2+5*4)/11
\end{verbatim}
SciLab performs the calculation and displays the answer.
\begin{verbatim}
 ans  =

    5.
\end{verbatim}
Now lets define a simple variable.
\begin{verbatim}
a=2
\end{verbatim}
SciLab responds with
\begin{verbatim}
 a  =

    2.
\end{verbatim}
Notice anything similar?  The response is almost the same but "ans" has been replaced by the variable name "a".  In fact it is even more similar than that.  When no assignment ("name=") is given, SciLab automatically assigns the result to the variable "ans".  Try using it.
\begin{verbatim}
a*ans
\end{verbatim}
SciLab will tell you that "ans" is now 10.  Lets move on and define a matrix.  Type the following
\begin{verbatim}
  A=[1,2;3,4;5,6]
\end{verbatim}
and press enter. Commas are used to separate elements and semicolons are used to separate rows.  Note that you could also have entered "A" using the alternate notation
\begin{verbatim}
  A=[[1 2];[3 4];[5 6]]
\end{verbatim}
or even (command prompt shown so you won't think something is wrong when it automatically appears, also you do not need to space over like I do to enter the numbers, I just find it easier to read)
\begin{verbatim}
-->  A=[[1 2]
-->     [3 4]
-->     [5 6]]
\end{verbatim}
Thus spaces work like commas and returns work like semicolons.  In any case, SciLab should respond by showing you that it has created the matrix variable as follows
\begin{verbatim}
 A  =

!   1.    2. !
!   3.    4. !
!   5.    6. !
\end{verbatim}
The variable "A" is now defined and can be used.  For instance we might want to define "B" to be "A+A".  Do this by typing
\begin{verbatim}
  B=A+A
\end{verbatim}
SciLab will add the matrices and define "B" to be the result, showing you the answer.
\begin{verbatim}
 B  =

!   2.     4.  !
!   6.     8.  !
!   10.    12. !
\end{verbatim}
This mode is useful for doing simple calculations and testing output.  We will refer to it as the interactive mode.  Since SciLab has an interactive mode that is command driven, it is reasonable to assume it would have a programming interface (we will refer to it as the programming mode).  I will show the use of programming mode later.

\section{Scilab and Matlab Programming}

Scilab is a Matlab look alike.  We will use Scilab as it is free and open source, but it is useful to also know about Matlab.

\subsection{Matlab}
There are two basic ways to interact with Matlab: command line execution, and M-files. Yes there are others such as MEX-files, Simulink, and several interfacing programs, but they are not relevant to us.  We will primarily be concerned with the use of M-files, because they are the most helpful.  Command line execution is really just for quick operations and checking of segments of code.  Matlab syntax is a high level programming language that interacts with a series of numerical libraries (most notably LinPack, EisPack, and BLAS).  Like most programming languages we have two types of programs that can be written.  A regular program, which is written as you would type commands on the command line, is the most basic type and is often the way you will start homework problems and other projects.  Functions, which are sub-programs called by another program (even recursively by other functions), are probably the most useful, as they allow you to extend the language by defining new operations.  One of the main goals of this class is for you to walk away with a library of Matlab functions that you can use to do a variety of tasks.  So how do you specify which you want?  You will get a regular program unless you start the M-file with the command function.  The syntax is

function a=name(x,y,..., z)

or

function [a,b,...,c]=name(x,y,..., z)

The second form returns multiple values.  Matlab gives us several command structures also: for, while, and if-elseif-else.  To see how these work let's use the programs I passed out last time as an example.

Homework:
Convert the Fortran program in 3.1 into Matlab syntax.
Do problems 9, 13, 14 from section 3.1
