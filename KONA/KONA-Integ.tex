\chapter{Integration}\label{c-Integ}

The fundamental theorem of Calculus tells us that an integral of a
function can be expressed in terms of the anti-derivative of the
function.  Unfortunately, not all functions have anti-derivatives
that are expressible in known functions.  One of the most famous is
the Gaussian probability distribution, which is given by
\beqn
e^{-\left(\frac{x-\mu}{\sigma}\right)^{2}}.
\eeqn
The anti-derivative of this important and frequently occurring function
is unknown.  How do we handle it?  That is the subject of this chapter.

\section{Riemann}
We recall from Calculus that the integral is defined as
\beqn
\int_{a}^{b}{f(x)dx} =
\lim_{n\rightarrow\infty}\sum_{j=1}^{n}f(p_{j})(x_{j}-x_{j-1}).
\eeqn
Now assume that all $n$ of the $x_{j}$ are evenly spaced on $[a,b]$.  We
can then write
\beqn
h & = & \frac{b-a}{n} \\
  & = & x_{j}-x_{j-1}.
\eeqn
We can use this to get an expression for the Riemann Sum
\beqn
\int_{a}^{b}{f(x)dx}
 & = &
\lim_{n\rightarrow\infty}\sum_{j=1}^{n}f(p_{j})(x_{j}-x_{j-1}) \\
 & = &
\lim_{n\rightarrow\infty}\sum_{j=1}^{n}f(p_{j})h \\
 & = &
\lim_{n\rightarrow\infty}h\sum_{j=1}^{n}f(p_{j}).
\eeqn
To evaluate the integral numerically we are not able to take the
limit, so we get
\beqn
\int_{a}^{b}{f(x)dx}
 & \approx &
h\sum_{j=1}^{n}f(p_{j}).
\eeqn
The exact size of $n$ for the approximation to be good is a key aspect
of numerical integration.  Note also that I have not specified what
$p_{j}$ is, as this form allows you to do a left, right, mid-point,
maximum, or minimum.  The basic idea here is that we are approximating
the function by a constant on the interval.

\beqn
\int_{a}^{b}{f(x)dx}
 & \approx &
h\sum_{j=1}^{n}f(p_{j}).
\eeqn

Now we want to think about the error.  First we want to get an
expression for the integral in terms of an exact sum.  To do this we
use the fundamental theorem of calculus, add nothing, rearrange, and
use the mean value theorem.
\beqn
\int_{a}^{b}f(x)dx
& = &
F(b)-F(a) \\
& = &
F(x_{n})-F(x_{0}) \\
& = &
F(x_{n})+\sum_{i=1}^{n-1}(F(x_{i})-F(x_{i}))-F(x_{0}) \\
& = &
\sum_{i=1}^{n}(F(x_{i})-F(x_{i-1})) \\
& = &
\sum_{i=1}^{n}f(c_{i})(x_{i}-x_{i-1}) \\
& = &
\sum_{i=1}^{n}f(c_{i})h \\
& = &
h\sum_{i=1}^{n}f(c_{i})
\eeqn
This expression is exact so we can take it and subtract the
expression for the midpoint method.  We will assume the function has a
derivative and we will note that since we have picked the midpoint
that any other point in the interval is within half the width of the
interval of the midpoint.
\beqn
E
& = &
h\sum_{i=1}^{n}f(c_{i}) - h\sum_{i=1}^{n}f(p_{i}) \\
& = &
h\sum_{i=1}^{n}(f(c_{i}) - f(p_{i})) \\
& = &
h\sum_{i=1}^{n}f'(d_{i})(c_{i} - p_{i}) \\
& \leq &
h\sum_{i=1}^{n}f'(d_{i})\frac{h}{2} \\
& = &
\frac{h^{2}}{2}\sum_{i=1}^{n}f'(d_{i})
\eeqn
Recall that $h$ is inversely proportional to $n$, so we have that the
Error is inversely proportional to the square of $n$.

\section{Trapezoid}


\beqn
\int_{a}^{b}{f(x)dx}
 & \approx &
h\left(\frac{f(x_{0})+f(x_{n})}{2}+\sum_{j=1}^{n}f(x_{j})\right)
\eeqn


\section{Simpson}

\beqn
\int_{a}^{b}{f(x)dx}
 & \approx &
\frac{h}{3}\left(f(x_{0})+f(x_{n})+2\sum_{j=1}^{\frac{n}{2}-1}f(x_{2j})
+4\sum_{j=1}^{\frac{n}{2}}f(x_{2j-1})\right)
\eeqn

\section{Richardson}

Define the value of the integral to be $I$ and the numeric
approximation by some method at $n$ node points to be $I_{n}$.  We note that the error is of the form
\beqn
E
& = &
I-I_{n} \\
& = &
\frac{c}{n^{p}},
\eeqn
with $c$ a constant dependent on the function and $p$ a power dependent
on the method.  For midpoint and trapezoidal methods $p=2$, while for
Simpson's method $p=4$.  If we double the number of points then the
error would be
\beqn
E
& = &
I-I_{2n} \\
& = &
\frac{c}{2^{p}n^{p}} \\
& = &
\frac{1}{2^{p}}(I-I_{n}).
\eeqn
Solving for $I$ we find
\beqn
I-I_{2n}
& = &
\frac{1}{2^{p}}(I-I_{n}) \\
I-\frac{1}{2^{p}}I
& = &
I_{2n}\frac{1}{2^{p}}I_{n} \\
\frac{2^{p}-1}{2^{p}}I
& = &
I_{2n}-\frac{1}{2^{p}}I_{n} \\
I
& = &
\frac{2^{p}}{2^{p}-1}I_{2n}-\frac{1}{2^{p}-1}I_{n} \\
I
& = &
\frac{1}{2^{p}-1}\left(2^{p}I_{2n}-I_{n}\right).
\eeqn
This is Richardson's Extrapolation formula, and it will typically
give a big improvement to any of the methods.  We can use this
estimate of the error to calculate the error
\beqn
E
& = &
I-I_{2n} \\
& = &
\frac{2^{p}}{2^{p}-1}I_{2n}-\frac{1}{2^{p}-1}I_{n}-I_{n} \\
& = &
\frac{2^{p}-(2^{p}-1)}{2^{p}-1}I_{2n}-\frac{1}{2^{p}-1}I_{n} \\
& = &
\frac{1}{2^{p}-1}I_{2n}-\frac{1}{2^{p}-1}I_{n} \\
& = &
\frac{1}{2^{p}-1}(I_{2n}-I_{n}).
\eeqn
This is Richardson's error formula.  We note that we can get a
convergence rate by doing a little algebra on what we have already
found.
\beqn
I-I_{2n}
& = &
\frac{1}{2^{p}}(I-I_{n}) \\
\frac{I-I_{n}}{I-I_{2n}}
& = &
2^{p}
\eeqn
This is a nice equation but in general it is not calculable, as we
don't know $I$ and might not know p.  We can handle this by considering
the quantity
\beqn
\frac{I_{2n}-I_{n}}{I_{4n}-I_{2n}}
& = &
\frac{I_{2n}-I_{n}+I-I}{I_{4n}-I_{2n}+I-I} \\
& = &
\frac{(I-I_{n})-(I-I_{2n})}{(I-I_{2n})-(I-I_{4n})} \\
& = &
\frac{(I-I_{n})-\frac{1}{2^{p}}(I-I_{n})}{\frac{1}{2^{p}}(I-I_{n})-\frac{1}{4^{p}}(I-I_{n})} \\
& = &
\frac{1-\frac{1}{2^{p}}}{\frac{1}{2^{p}}(1-\frac{1}{2^{p}})} \\
& = &
\frac{1}{\frac{1}{2^{p}}} \\
& = &
2^{p}.
\eeqn
We thus have a simple way of calculating the convergence rate, and
thus a way to find $p$.

Homework: 7.1) 2(a), (b), (f) for midpoint, trapezoidal, and
simpson.  Compare with the error for trapezoidal (7.32) and
Richardson's extrapolation.

7.2) 8

\section{Gaussian Quadrature}

And now for something completely different\ldots

Last time we considered the standard way of thinking about integration, namely summing up a bunch of small areas.  The technique we will discuss today was introduced in 1814 by Gauss, hence the name.  We will now consider the integral
\beqn
I(f) = \int_{-1}^{1}f(t)dt.
\eeqn
Note that this is a perfectly general statement, as all we need to do to convert this to an integral of the form $\int_a^bf(x)dx$ is to find a linear mapping between the two intervals of integration:
\begin{eqnarray}
t&=& [-1,1]\\
x&=& [a,b]
\end{eqnarray}
We thus want to find a function $x=g(t)$ such that $a=g(-1)$, $b=g(1)$, and $g(\cdot)$ is linear.  We have two points which determine a line, so we can use the two-point formula for a line:
\begin{eqnarray}
m_g
&=& \frac{x_1-x_0}{t_1-t_0}\\
&=& \frac{b-a}{1--1}\\
&=& \frac{b-a}{2}\\
x
&=& x_1+m_g(t-t_1)\\
&=& b+\frac{b-a}{2}(t-1)\\
&=& \frac{1}{2}(2b+t(b-a)-(b-a))\\
&=& \frac{1}{2}(t(b-a)+b+a)\\
&=& \frac{t(b-a)+b+a}{2}
\end{eqnarray}
So, if we define $x=0.5(t(b-a)+b+a)$ and thus $dx=0.5(b-a)dx$ then we have\footnote{You will see some books define $t=(2x-a-b)/(b-a)$, and $dt=2dx/(b-a)$, which is the same formula but not as convenient for what we want.}
\beqn
\int_{a}^{b}{f(x)dx}
 & = &
\int_{g(-1)}^{g(1)}{f(x)g'(t)dt}\\
 & = &
\int_{-1}^{1}{f\left(\frac{(b-a)t+b+a}{2}\right)\left(\frac{b-a}{2}\right)dt}
\eeqn
We have that the integral is general, if not all that obvious as to why we chose this in the first place (it will become more apparent in a few moments).  We still need to show how to estimate the integral.  The basic idea is to approximate the integral by weighted evaluations of the function at a series of node points.  To get a good estimate we will require that our estimate at $n$ node points will be good for every polynomial up to order $2n-1$.  How?  Pick $w_{i}$ and $x_{i}$ such that
\beqn
I(f)=\sum_{i=1}^{n}w_{i}f(x_{i})
\eeqn
holds for all $f\in\{1,x,\ldots,x^{2n-1}\}$.  Let's do some examples.

\vspace{.1in}\noindent
\textbf{Example:}

Consider $n=1$.
\beqn
\int_{-1}^{1}dx
 & = &
w_{1}f(x_{1}) \\
2
 & = &
w_{1} \\
\int_{-1}^{1}xdx
 & = &
w_{1}x_{1} \\
0
 & = &
2x_{1} \\
0
 & = &
x_{1}
\eeqn

Also, consider $n=2$.
\beqn
\int_{-1}^{1}dx
 & = &
w_{1}f(x_{1})+w_{2}f(x_{2}) \\
2
 & = &
w_{1}+w_{2} \\
\int_{-1}^{1}xdx
 & = &
w_{1}f(x_{1})+w_{2}f(x_{2}) \\
0
 & = &
w_{1}x_{1}+w_{2}x_{2} \\
\int_{-1}^{1}x^{2}dx
 & = &
w_{1}f(x_{1})+w_{2}f(x_{2}) \\
\frac{2}{3}
 & = &
w_{1}x_{1}^{2}+w_{2}x_{2}^{2} \\
\int_{-1}^{1}x^{3}dx
 & = &
w_{1}f(x_{1})+w_{2}f(x_{2}) \\
0
 & = &
w_{1}x_{1}^{3}+w_{2}x_{2}^{3}
\eeqn
Solve these four equations for four unknowns and we find
\beqn
w_{1} & = & 1 \\
w_{2} & = & 1 \\
x_{1} & = & -\frac{1}{\sqrt{3}} \\
x_{1} & = & \frac{1}{\sqrt{3}}
%w_{2} & = & 2-w_{1} \\
%x_{2} & = & -\frac{w_{1}}{2-w_{1}}x_{1} \\
%w_{1} & = & -(2-w_{1})\left(\frac{-\frac{w_{1}}{2-w_{1}}x_{1}}{x_{1}}\right)^{2} \\
%x_{1} & = & \left(
%  \frac{2-w_{1}}{-(2-w_{1})\left(\frac{-\frac{w_{1}}{2-w_{1}}x_{1}}{x_{1}}\right)^{2}}
%  \right)^{\frac{1}{3}}
\eeqn
The node points turn out to be the roots of the Legendre
polynomials.  The Legendre polynomials are defined on $[-1,1]$ hence
our choice for the limits of integration.  You can find the Legendre
polynomials by the following properties.
\begin{itemize}
\item
$P_{n}$ is a polynomial of order $n$.
\item
They are orthogonal
\beqn
\int_{-1}^{1}P_{i}(x)P_{j}(x)dx=0
\eeqn
when $i\ne j$.
\item
The normalization is
\beqn
\int_{-1}^{1}P_{n}(x)^{2}dx=\frac{2}{2n+1}
\eeqn
\end{itemize}
The first several Legendre polynomials are
$\{1,x,x^{2}-\frac{1}{3},x^{3}-\frac{3}{5}x,x^{4}-\frac{6}{7}x^{2}+\frac{3}{35}\}$.

The constants can be found by
\beqn
w_{i}=\int_{-1}^{1}\prod_{j=1, j\ne i}^{n}\frac{x-x_{j}}{x_{i}-x_{j}}dx
\eeqn

In general we do not need to use this as the values are well tabulated, see for instance Table~\ref{t-gauss-quad}.


\begin{table}

  \centering
  \caption{Gauss-Legendre Abscissae and Weights to 8 Decimal Places}\label{t-gauss-quad}

\begin{tabular}{rr@{}llc}
n  & \multicolumn{2}{c}{Evaluation Points, $x_i$} & Weights, $w_i$ & Degree (2n-1) \\ \hline
1  & 0&.0                                         & 2.0            & 1 \\ \hline
2  & $\pm$&$\frac{1}{\sqrt{3}}$                   & 1.0            & 3 \\ \hline
3  & 0&.0                                         & 0.88888889     & 5 \\
   & $\pm$0&.77459667                             & 0.55555555     &   \\ \hline
4  & $\pm$0&.33998104                             & 0.65214515     & 7 \\
   & $\pm$0&.86113631                             & 0.34785485     &   \\ \hline
5  & 0&.0                                         & 0.56888889     & 9 \\
   & $\pm$0&.53846931                             & 0.47862867     &   \\
   & $\pm$0&.90617985                             & 0.23692689     &   \\ \hline
6  & $\pm$0&.23861918                             & 0.46791393     & 11 \\
   & $\pm$0&.66120939                             & 0.36076157     &   \\
   & $\pm$0&.93246951                             & 0.17132449     &   \\ \hline
7  & 0&.0                                         & 0.41795918     & 13 \\
   & $\pm$0&.40584515                             & 0.38183005     & \\
   & $\pm$0&.74153119                             & 0.27970539     & \\
   & $\pm$0&.94910791                             & 0.12948497     & \\ \hline
8  & $\pm$0&.18343464                             & 0.36268378     & 15 \\
   & $\pm$0&.52553241                             & 0.31370665     & \\
   & $\pm$0&.79666648                             & 0.22238103     & \\
   & $\pm$0&.96028986                             & 0.10122854     & \\ \hline
10 & $\pm$0&.14887434                             & 0.29552422     & 19 \\
   & $\pm$0&.43339539                             & 0.26926672     & \\
   & $\pm$0&.67940957                             & 0.21908636     & \\
   & $\pm$0&.86506337                             & 0.14945135     & \\
   & $\pm$0&.97390653                             & 0.06667134     & \\ \hline
\end{tabular}
\end{table}



Homework Redo 7.1) 2(a), (b), (f) for gaussian quadrature up to
$n=8$.  How does the convergence compare?

