\chapter{Systolic Array}

The preceding algorithms are $O(n^2)$ if implemented with ripple adders, $O(n\log(n))$ if implemented with conditional sum adders, or $O(n)$ if implemented with look-ahead adders.  The look-ahead adders have a large constant, so the $O(n)$ is not a perfect indicator of performance, and they are currently not practical beyond about 8 bits.  It would be nice to find a way to multiply that has $O(n)$ and a small constant multiplier.  Systolic arrays are $O(n)$, and have a constant multiplier of about 6 depending on your hardware, which is about half what it takes with even block (group) carry look-ahead adders using serial routines.

\begin{figure}[p]
  \SystolicCell
  \caption{Individual Cell of Systolic Array}\label{fig-systolic-cell}
\end{figure}

\begin{figure}[p]
  \SystolicArray
  \caption{Systolic Array For 4 Bit Numbers}\label{fig-systolic-array}
\end{figure}


\section{Integrated Examples}







\begin{example}
Calculate the following expression in binary using 2's
complement and 8 bits total.  Show all work.
\beqn
(9*9-24)/3
\eeqn
{\color{ans}
Sol:

$9_{10}=00001001_{2}$ and $3_{10}=00000011_{2}$

\begin{tabular}{l|l}
24 & \\
\hline
12 & 0 \\
6 & 0 \\
3 & 0 \\
1 & 1 \\
0 & 1 \\
\end{tabular}

$24_{10}=00011000_{2}$ thus $-24_{10}=11101000_{2}$.
Thus $9*9$,

\begin{tabular}{cccccccccccc}
&&&&0&0&0&0&1&0&0&1 \\
&&&&0&0&0&0&1&0&0&1 \\
\hline
&&&&0&0&0&0&1&0&0&1 \\
&0&0&0&0&1&0&0&1&&& \\
\hline
&&&&0&1&0&1&0&0&0&1 \\
\end{tabular}

Then (subtracting 24),

\begin{tabular}{ccccccccc}
&0&1&0&1&0&0&0&1 \\
&1&1&1&0&1&0&0&0 \\
\hline
1&0&0&1&1&1&0&0&1 \\
&0&0&1&1&1&0&0&1 \\
\end{tabular}

Now perform the division:

\begin{tabular}{cccccccc}
&&&1&0&0&1&1 \\ \cline{3-8}
1 & 1 & \multicolumn{1}{|c}{1} & 1 & 1 & 0 & 0 & 1 \\
  &   & 1 & 1 &   &   &   & \\ \cline{3-4}
  &   &   & 0 & 1 & 0 & 0 & \\
  &   &   &   &   & 1 & 1 & \\ \cline{6-7}
  &   &   &   &   &   & 1 & 1 \\
  &   &   &   &   &   & 1 & 1 \\ \cline{7-8}
  &   &   &   &   &   &   & 0 \\
\end{tabular}

The answer is thus $00010011_{2} = 19_{10}$.
}
\end{example}



