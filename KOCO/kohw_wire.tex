\documentclass{article}

\begin{document}
\begin{eqnarray}
1 &=& \frac{heat generated}{heat lost} \\
 &=& \frac{i^2R}{therm  (area wire)} \\
 &=& \frac{i^2 \frac{\rho L}{A}}{therm (L2\pi r)} \\
 &=& \frac{i^2 \frac{\rho L}{\pi r^2}}{therm L2\pi r} \\
 &=& \frac{i^2\rho}{therm 2\pi^2 r^3} \\
r &=& \sqrt[3]{\frac{i^2\rho}{therm 2\pi^2}} \\
 &=& \sqrt[3]{\frac{i^2\rho}{therm 2\pi^2}}
\end{eqnarray}


 The amount of heat energy (q) in Joules gained or lost by a substance is equal to the mass of the substance (m) in grams multiplied by its specific heat capacity (Cg) multiplied by the change in temperature (final temperature - initial temperature) and is stated as: q = m x Cg x (Tf - Ti).


\end{document}