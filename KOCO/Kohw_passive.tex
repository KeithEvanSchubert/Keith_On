\chapter{Passive Components}


\section{Resistor}

\begin{eqnarray}
V &=& IR \\
P &=& VI \\
  &=& I^2R \\
  &=& \frac{V^2}{R}
\end{eqnarray}

DC: $R=\frac{l\cdot\rho}{A}$, where $l$ is the length in meters, $A$ is the cross sectional area in square meters and $\rho$ is the electric resistivity or specific electrical resistance in ohm-meters.  This assumes the current density is uniform.

\begin{picture}(200,200)(0,0)
\put(0,0){\Rvert{$222k\Omega$}}
\put(0,60){\Rhoriz{$222k\Omega$}}
%\put(70,79){\bezier{100}(0,0)(30,10)(40,40)}
\end{picture} 

\section{Capacitor}


\begin{eqnarray}
cV=q \label{cviq}
\end{eqnarray}
When I took my physics E\&M class my professor had an interesting way to remember equation~\ref{cviq}. One of his friends in college used a beer slogan, ``Canadian Velvet is the Queen of beers'', as a mnemonic.

\begin{picture}(200,200)(0,0)
\\put(70,0){\Cvert{$0.01\mu f$}}
\put(70,60){\Choriz{$0.01\mu f$}}
%\put(70,79){\bezier{100}(0,0)(30,10)(40,40)}
\end{picture} 

\section{Inductor}

Symbol: L

Unit: Henry (volt sec/Amp)

\begin{eqnarray}
\phi &=& LI
\end{eqnarray}
Note: $\phi$ is magnetic flux

\section{Memristence}

Memristors were first predicted in the 1970's due to symmetry, but were first made in 2007.  They have great potential to revolutionize memory.

\begin{eqnarray}
\phi &=& M(q)q
\end{eqnarray}
Thus, at some instant in time
\begin{eqnarray}
V(t) &=& M(q(t))I(t)
\end{eqnarray}
Note that $M$ is not a constant, and in fact it is non-linear with hysteresis.



