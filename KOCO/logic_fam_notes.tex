Diode Logic
  In DL (diode logic), all the logic is implemented using diodes and resistors. One basic thing about the diode, is that diode needs to be forward biased to conduct. Below is the example of a few DL logic circuits.






  When no input is connected or driven, output Z is low, due to resistor R1. When high is applied to either X or Y, or both X and Y are driven high, the corresponding diode get forward biased and thus conducts. When any diode conducts, output Z goes high.



  Points to Ponder

  Diode Logic suffers from voltage degradation from one stage to the next.
Diode Logic only permits OR and AND functions.
Diode Logic is used extensively but not in integrated circuits.



   Resistor Transistor Logic
  In RTL (resistor transistor logic), all the logic are implemented using resistors and transistors. One basic thing about the transistor (NPN), is that HIGH at input causes output to be LOW (i.e. like a inverter). Below is the example of a few RTL logic circuits.






  A basic circuit of an RTL NOR gate consists of two transistors Q1 and Q2, connected as shown in the figure above. When either input X or Y is driven HIGH, the corresponding transistor goes to saturation and output Z is pulled to LOW.






   Diode Transistor Logic


  In DTL (Diode transistor logic), all the logic is implemented using diodes and transistors. A basic circuit in the DTL logic family is as shown in the figure below. Each input is associated with one diode. The diodes and the 4.7K resistor form an AND gate. If input X, Y or Z is low, the corresponding diode conducts current, through the 4.7K resistor. Thus there is no current through the diodes connected in series to transistor base . Hence the transistor does not conduct, thus remains in cut-off, and output out is High.



  If all the inputs X, Y, Z are driven high, the diodes in series conduct, driving the transistor into saturation. Thus output out is Low.






   Transistor Transistor Logic
  In Transistor Transistor logic or just TTL, logic gates are built only around transistors. TTL was developed in 1965. Through the years basic TTL has been improved to meet performance requirements. There are many versions or families of TTL.



  Standard TTL.
High Speed TTL
Low Power TTL.
Schhottky TTL.



  Here we will discuss only basic TTL as of now; maybe in the future I will add more details about other TTL versions. As such all TTL families have three configurations for outputs.



  Totem - Pole output.
Open Collector Output.
Tristate Output.



  Before we discuss the output stage let's look at the input stage, which is used with almost all versions of TTL. This consists of an input transistor and a phase splitter transistor. Input stage consists of a multi emitter transistor as shown in the figure below. When any input is driven low, the emitter base junction is forward biased and input transistor conducts. This in turn drives the phase splitter transistor into cut-off.






   Totem - Pole Output
  Below is the circuit of a totem-pole NAND gate, which has got three stages.



  Input Stage
Phase Splitter Stage
Output Stage



  Input stage and Phase splitter stage have already been discussed. Output stage is called Totem-Pole because transistor Q3 sits upon Q4.



  Q2 provides complementary voltages for the output transistors Q3 and Q4, which stack one above the other in such a way that while one of these conducts, the other is in cut-off.



  Q4 is called pull-down transistor, as it pulls the output voltage down, when it saturates and the other is in cut-off (i.e. Q3 is in cut-off). Q3 is called pull-up transistor, as it pulls the output voltage up, when it saturates and the other is in cut-off (i.e. Q4 is in cut-off).



  Diodes in input are protection diodes which conduct when there is large negative voltage at input, shorting it to the ground.






   Tristate Output.
  Normally when we have to implement shared bus systems inside an ASIC or externally to the chip, we have two options: either to use a MUX/DEMUX based system or to use a tri-state base bus system.



  In the latter, when logic is not driving its output, it does not drive LOW neither HIGH, which means that logic output is floating. Well, one may ask, why not just use an open collector for shared bus systems? The problem is that open collectors are not so good for implementing wire-ANDs.



  The circuit below is a tri-state NAND gate; when Enable En is HIGH, it works like any other NAND gate. But when Enable En is driven LOW, Q1 Conducts, and the diode connecting Q1 emitter and Q2 collector, conducts driving Q3 into cut-off. Since Q2 is not conducting, Q4 is also at cut-off. When both pull-up and pull-down transistors are not conducting, output Z is in high-impedance state.






  Note : I will try to add more details when I find time.











































I have received a number of requests, asking just what goes on inside logic gates to actually perform logic functions. So, by popular demand, here are the internal schematics of various gates, as implemented by several different logic families.

I won't cover the internal operation of individual semiconductor devices in these pages, except to state the basic behavior of a given device under specific conditions. More detailed coverage of semiconductor physics and internal behavior is a job for another set of pages, which will come later.


--------------------------------------------------------------------------------

There are several different families of logic gates. Each family has its capabilities and limitations, its advantages and disadvantages. The following list describes the main logic families and their characteristics. You can follow the links to see the circuit construction of gates of each family.

Diode Logic (DL)

Diode logic gates use diodes to perform AND and OR logic functions. Diodes have the property of easily passing an electrical current in one direction, but not the other. Thus, diodes can act as a logical switch.

Diode logic gates are very simple and inexpensive, and can be used effectively in specific situations. However, they cannot be used extensively, as they tend to degrade digital signals rapidly. In addition, they cannot perform a NOT function, so their usefulness is quite limited.



Resistor-Transistor Logic (RTL)

Resistor-transistor logic gates use Transistors to combine multiple input signals, which also amplify and invert the resulting combined signal. Often an additional transistor is included to re-invert the output signal. This combination provides clean output signals and either inversion or non-inversion as needed.

RTL gates are almost as simple as DL gates, and remain inexpensive. They also are handy because both normal and inverted signals are often available. However, they do draw a significant amount of current from the power supply for each gate. Another limitation is that RTL gates cannot switch at the high speeds used by today's computers, although they are still useful in slower applications.

Although they are not designed for linear operation, RTL integrated circuits are sometimes used as inexpensive small-signal amplifiers, or as interface devices between linear and digital circuits.



Diode-Transistor Logic (DTL)

By letting diodes perform the logical AND or OR function and then amplifying the result with a transistor, we can avoid some of the limitations of RTL. DTL takes diode logic gates and adds a transistor to the output, in order to provide logic inversion and to restore the signal to full logic levels.



Transistor-Transistor Logic (TTL)


The physical construction of integrated circuits made it more effective to replace all the input diodes in a DTL gate with a transistor, built with multiple emitters. The result is transistor-transistor logic, which became the standard logic circuit in most applications for a number of years.

As the state of the art improved, TTL integrated circuits were adapted slightly to handle a wider range of requirements, but their basic functions remained the same. These devices comprise the 7400 family of digital ICs.



Emitter-Coupled Logic (ECL)

Also known as Current Mode Logic (CML), ECL gates are specifically designed to operate at extremely high speeds, by avoiding the "lag" inherent when transistors are allowed to become saturated. Because of this, however, these gates demand substantial amounts of electrical current to operate correctly.



CMOS Logic

One factor is common to all of the logic families we have listed above: they use significant amounts of electrical power. Many applications, especially portable, battery-powered ones, require that the use of power be absolutely minimized. To accomplish this, the CMOS (Complementary Metal-Oxide-Semiconductor) logic family was developed. This family uses enhancement-mode MOSFETs as its transistors, and is so designed that it requires almost no current to operate.

CMOS gates are, however, severely limited in their speed of operation. Nevertheless, they are highly useful and effective in a wide range of battery-powered applications.



--------------------------------------------------------------------------------

Most logic families share a common characteristic: their inputs require a certain amount of current in order to operate correctly. CMOS gates work a bit differently, but still represent a capacitance that must be charged or discharged when the input changes state. The current required to drive any input must come from the output supplying the logic signal. Therefore, we need to know how much current an input requires, and how much current an output can reliably supply, in order to determine how many inputs may be connected to a single output.

However, making such calculations can be tedious, and can bog down logic circuit design. Therefore, we use a different technique. Rather than working constantly with actual currents, we determine the amount of current required to drive one standard input, and designate that as a standard load on any output. Now we can define the number of standard loads a given output can drive, and identify it that way. Unfortunately, some inputs for specialized circuits require more than the usual input current, and some gates, known as buffers, are deliberately designed to be able to drive more inputs than usual. For an easy way to define input current requirements and output drive capabilities, we define two new terms:

Fan-in

The number of standard loads drawn by an input to ensure reliable operation. Most inputs have a fan-in of 1.

Fan-out

The number of standard loads that can be reliably driven by an output, without causing the output voltage to shift out of its legal range of values.

Remember, fan-in and fan-out apply directly only within a given logic family. If for any reason you need to interface between two different logic families, be careful to note and meet the drive requirements and limitations of both families, within the interface circuitry.
