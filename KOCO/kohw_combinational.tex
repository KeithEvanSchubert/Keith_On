\chapter{Combinational Circuits}
\label{c-comb}


Combinational circuits are the most basic type of circuit that can be designed in that they have no memory input.  In a combinational circuit the outputs are completely determined by the inputs.

Consider a simple example.  Say I have two hard disks on my computer and I want to hook up a light that shows when either is accessed.  Hard disks have an output line that signals when they are accessed so we have two variables $d_1$ and $d_2$ which are the disk access output signals from the drives.  Since I want the light to come on when either drive is accessed the truth table describing this is

\begin{tabular}{c|c||cl}
$d_1$ & $d_2$ & light & comments\\ \hline
0     & 0     & 0     & neither disk accessed, no light \\
0     & 1     & 1     & second disk being accessed, light on \\
1     & 0     & 1     & first disk being accessed, light on \\
1     & 1     & 1     & both disks being accessed, light on \\
\end{tabular}

This table is identical to the definition of ``or'' so we have that $\hbox{light}=d_1 + d_2$.  Thus by connecting the signals from the disks to an ``or'' gate and using the output of the gate to drive the light.  This is a combinational circuit because it does not matter what happened in the past or what some variable's value is (the value of variables in a circuit with memory is known as state).

Combinational circuits are the foundation of digital design, as sequential circuits (the circuits with memory) can be handled as a combinational circuits driven by inputs and memory and the outputs not only drive other circuits, they modify the memory that drives the input.  You can thus consider all sequential circuits as combinational ones with feedback.



\section{Designing: Tables}

If there are only a few trues (or falses) that need to be generated and a small number of input variables, then it is easy to do the design off a truth table by reading the canonical terms.  Even complex problems can be designed with the use of decoders or multiplexors (mux).

\subsection{Implementing With Sum of Products}

Sum of Products design rules:
\begin{itemize}
    \item For each row in the table where the output is a ``1'', connect the inputs to a \textbf{nand} gate (or an \textbf{and} gate) being sure to invert any input line that has a ``0'' in that row.
    \item Connect the outputs of the previous gates into another \textbf{nand} gate (or an \textbf{or} gate if you used \textbf{and} gates in the previous step).
    \item The output of the last gate is the desired output.
\end{itemize}

\subsection{Implementing With Product of Sums}

Sum of Products design rules:
\begin{itemize}
    \item For each row in the table where the output is a ``0'', connect the inputs to a \textbf{nor} gate (or an \textbf{or} gate) being sure to invert any input line that has a ``1'' in that row.
    \item Connect the outputs of the previous gates into another \textbf{nor} gate (or an \textbf{and} gate if you used \textbf{or} gates in the previous step).
    \item The output of the last gate is the desired output.
\end{itemize}

\subsection{Implementing With Decoders}

Decoders have an enable input, $n$ address lines, and $2^n$ output lines that are true if the decoder is enabled and the address on the address line is their line on the decoder.  Decoder designs have a few simple rules:
\begin{itemize}
    \item Enable the decoder.
    \item Connect the inputs to the address lines in the sequence of the table.
    \item Connect the decoder outputs that correspond to 1's in the table to an \textbf{or} gate.
    \item The output of the \textbf{or} gate is the desired output.
\end{itemize}

\vspace{.1in}
It is easiest to see this by an example.  Consider the following table from our canonical term section.

\begin{tabular}{cc}
\begin{tabular}{c|c|c|c}
x & y & z & $a_4$ \\ \hline
0 & 0 & 0 & 0 \\
0 & 0 & 1 & 0 \\
0 & 1 & 0 & 1 \\
0 & 1 & 1 & 0 \\
1 & 0 & 0 & 1 \\
1 & 0 & 1 & 1 \\
1 & 1 & 0 & 0 \\
1 & 1 & 1 & 1 \\
\end{tabular}
&
{\tt    \setlength{\unitlength}{0.92pt}
\begin{picture}(180,150)(0,55)
\put(32,106){1}
\put(40,20){\DecoderEight}
\put(40,10){\{x,y,z\}}
\put(128,100){\OrFour}
\put(80,40){\line(3,4){48}}
\put(80,80){\line(3,2){42}}
\put(122,108){\line(1,0){6}}
\put(80,100){\line(4,1){48}}
\put(80,140){\line(2,-1){48}}
\put(170,107){$a_4$}
\end{picture}} \\
\end{tabular}

The technique can be seen to be fairly straightforward.  It can require a large decoder if the 1's and 0's are mixed, but it can be done with a small decoder if there are few tightly grouped 1's or 0's.

\subsection{Implementing With Multiplexors}

A mux has $2^n$ input lines, $n$ address select lines, and 1 output.  The input line that corresponds to the address is passed to the output.  The design technique is a little tricky.
\begin{itemize}
    \item All but one of the inputs are connected to the address select lines.
    \item The remaining input is used to divide the table into pairs, each pair corresponding to one of the $2^n$ input lines.
    \begin{itemize}
        \item If both outputs in a pair in the table are both zero then the corresponding input line is grounded.
        \item If both outputs in a pair in the table are both one then the corresponding input line is set high.
        \item If both outputs in a pair in the table are the same as the one unconnected input, then that input is connected to the corresponding input to the mux.
        \item If both outputs in a pair in the table are the opposite of the one unconnected input, then the inverse of that input is connected to the corresponding input to the mux.
    \end{itemize}
    \item The desired function is on the output of the mux.
\end{itemize}

\vspace{.1in}
Let us one more time examine our example from our canonical term section.

\begin{tabular}{cc}
\begin{tabular}{c|c|c|cl}
x & y & z & $a_4$ & Mux Input \\ \hline
0 & 0 & 0 & 0     & 0         \\
0 & 0 & 1 & 0     &           \\ \hline
0 & 1 & 0 & 1     & z'        \\
0 & 1 & 1 & 0     &           \\ \hline
1 & 0 & 0 & 1     & 1         \\
1 & 0 & 1 & 1     &           \\ \hline
1 & 1 & 0 & 0     & z         \\
1 & 1 & 1 & 1     &           \\
\end{tabular}
&
{\tt    \setlength{\unitlength}{0.92pt}
\begin{picture}(100,70)(0,60)
\put(20,20){\MuxFour}
\put(25,10){\{x,y\}}
\put(10,97){0}
\put(10,77){z'}
\put(10,57){1}
\put(10,37){z}
\put(62,67){$a_4$}
\end{picture}} \\
\end{tabular}

\vspace{.1in}
This one is hard to see at first but it is easy and compact once understood.

\section{Designing: Karnaugh Maps}

Karnaugh maps are a nice visual way to handle small design problems, i.e. those with less than 6-8 inputs.  Karnaugh maps are formed by making a table indexed in both rows and columns by the inputs which are arranged in Grey code order (00,01,11,10)\footnote{Veitch diagrams are just like Karnaugh maps, but they are in normal binary order.  This makes them a pain to recognize patterns and so they are rarely used.}.

Basic rules for all encirclements:
\begin{enumerate}
\item Always encircle only a number of items equal to a power of 2 (1, 2, 4, 8, 16, etc.).
\item Only encircle either 0's or 1's, but not a mixture.  Since don't cares, x, could be either a 0 or 1, you can mix and match them.
\item Make only the largest encirclements possible.
\item Overlap encirclements (partial due to above rule) whenever possible to remove errors of type 1.  Diagonal overlaps will take care of errors of type 2.
\end{enumerate}

Rules for encircling with \textbf{and} gates for SOP:
\begin{enumerate}
\item Only encircle 1's.
\item Encirclements must be horizontally or vertically aligned rectangles.
\end{enumerate}

Rules for encircling with \textbf{or} gates for POS:
\begin{enumerate}
\item Only encircle 0's.
\item Encirclements must be horizontally or vertically aligned rectangles.
\end{enumerate}

Rules for encircling with \textbf{xor} or \textbf{equiv} gates:
\begin{enumerate}
\item Only encircle 1's.
\item Encirclements must be checkerboard patterned (diagonal).
\end{enumerate}


\begin{example}
Make a Karnaugh map for $\Pi(1,2,3,6,7,9,11,15)_{A,B,C,D}$ and use it to simplify the expression.  Implement your result using And, Or, and inverter gates in a HDL module to describe the circuit.

Sol:

\begin{tabular}{r|c|c|c|c|l}
\multicolumn{3}{c}{}&\multicolumn{2}{c}{\underline{$\;\;\;\text{A}\;\;\;$}}& \\\cline{2-5}
                           & 1 & 1 & 1 & 1 & \\\cline{2-5}
                           & 0 & 1 & 1 & 0 & \multirow{2}{*}{\Big| D} \\\cline{2-5}
\multirow{2}{*}{C$\Big|$}  & 0 & 0 & 0 & 0 &  \\\cline{2-5}
                           & 0 & 0 & 1 & 1 & \\\cline{2-5}
\multicolumn{2}{c}{}&\multicolumn{2}{c}{\vrule height 11pt width 0pt$\overline{\;\;\;\text{B}\;\;\;}$}&\multicolumn{2}{c}{} \\
\end{tabular}

Three 4 entry encirclements of zeros (two squares and a row).  This gives the simplification as:
\beqn
(C'+D')\cdot(A + C')\cdot(B+D')
\eeqn
Alternately you could make three 4 entry encirclements of ones (two squares and a row).  The simplification would then be:
\beqn
C'\cdot D' + B\cdot C' + A\cdot D'
\eeqn

A HDL implementation of the simplified sum of products form is:
\begin{verbatim}
module my_circ(F,A,B,C,D);
  input A,B,C,D;
  output F;

  wire c,d,e,x,y,z;

  not g1(c,C);
  not g2(d,D);
  and g3(x,c,d);
  and g4(y,B,c);
  and g5(z,A,d);
  or g6(e,x,y);
  or g7(F,e,z);
endmodule
\end{verbatim}


  {\tt    \setlength{\unitlength}{0.6pt}
  \begin{picture}(40,100)(0,0)
  \put(0,0){\PALIn{x}}
  \end{picture}}
\end{example}



\begin{example}
We wanted to design a system to check three lines, say A, B, C.  If only one line is active we want to receive a signal.  We are also interested in knowing if lines A and C are active, and we want no errors of type-1.  The design is small, so we start with a Karnaugh map.

\begin{tabular}{r|c|c|c|c|}
\multicolumn{3}{c}{}&\multicolumn{2}{c}{\underline{$\;\;\;\text{A}\;\;\;$}}\\\cline{2-5}
     & 0 & 1 & 0 & 1 \\\cline{2-5}
C$|$ & 1 & 0 & 1 & 1 \\\cline{2-5}
\multicolumn{2}{c}{}&\multicolumn{2}{c}{\vrule height 11pt width 0pt$\overline{\;\;\;\text{B}\;\;\;}$}&\multicolumn{1}{c}{} \\
\end{tabular}
\end{example}

\section{Quine-McCluskey}

Originally proposed by Quine and then modified by McCluskey, this method provides an symbolic tabular way to minimize a boolean algebraic function. Graphical methods like Karnaugh maps are great for up to about 6 variables, but then they bog down really badly.

The idea of this method is to combine terms using the rule $xy+xy' = x$, where $x$ represents multiple variables, but $y$ is only one variable.


\begin{eqnarray}
\sum_{abcd}(1,3,4,5,6,7,10,14,15)
\end{eqnarray}
We begin by writing the minterms in binary.  We then sort them so they will be in order of increasing index. Index is defined to be the number of 1's in the expression.  In order to combine terms they may differ by only 1 value, so we only need compare each index group with the group above it.  Here is the term by term combination to generate the 2-term implicants from the minterms.  When a minterm is used it is checked to note it cannot be a prime implicant, though we continue to use it to generate other terms as a minterm can be in multiple groupings.

\begin{tabular}{cp{.2in}cp{.2in}c}
\begin{tabular}{cp{.1in}cp{.1in}}
\multicolumn{2}{c}{Minterms} & \multicolumn{2}{c}{2-terms} \\\hline
0001 & & & \\
0100 & & & \\\cline{1-2}
0011 & & & \\
0101 & & & \\
0110 & & & \\
1010 & & & \\\cline{1-2}
0111 & & & \\
1110 & & & \\\cline{1-2}
1111 & & & \\
\end{tabular}
&&
\begin{tabular}{cp{.1in}cp{.1in}}
\multicolumn{2}{c}{Minterms} & \multicolumn{2}{c}{2-terms} \\\hline
\textcolor[rgb]{1.00,0.00,0.00}{0001} & \checkmark & \textcolor[rgb]{1.00,0.00,0.00}{00-1} & \\
0100 & &  & \\\cline{1-2}
\textcolor[rgb]{1.00,0.00,0.00}{0011} & \checkmark & & \\
0101 & & & \\
0110 & & & \\
1010 & & & \\\cline{1-2}
0111 & & & \\
1110 & & & \\\cline{1-2}
1111 & & & \\
\end{tabular}
&&
\begin{tabular}{cp{.1in}cp{.1in}}
\multicolumn{2}{c}{Minterms} & \multicolumn{2}{c}{2-terms} \\\hline
\textcolor[rgb]{1.00,0.00,0.00}{0001} & \checkmark & 00-1 & \\
0100 &            & \textcolor[rgb]{1.00,0.00,0.00}{0-01} & \\\cline{1-2}
0011 & \checkmark & & \\
\textcolor[rgb]{1.00,0.00,0.00}{0101} & \checkmark & & \\
0110 & & & \\
1010 & & & \\\cline{1-2}
0111 & & & \\
1110 & & & \\\cline{1-2}
1111 & & & \\
\end{tabular}\\
\end{tabular}

\vspace{.1in}

\begin{tabular}{cp{.2in}cp{.2in}c}
\begin{tabular}{cp{.1in}cp{.1in}}
\multicolumn{2}{c}{Minterms} & \multicolumn{2}{c}{2-terms} \\\hline
0001 & \checkmark & 00-1 & \\
\textcolor[rgb]{1.00,0.00,0.00}{0100} & \checkmark & 0-01 & \\\cline{1-2}
0011 & \checkmark & \textcolor[rgb]{1.00,0.00,0.00}{010-} & \\
\textcolor[rgb]{1.00,0.00,0.00}{0101} & \checkmark & & \\
0110 & & & \\
1010 & & & \\\cline{1-2}
0111 & & & \\
1110 & & & \\\cline{1-2}
1111 & & & \\
\end{tabular}
&&
\begin{tabular}{cp{.1in}cp{.1in}}
\multicolumn{2}{c}{Minterms} & \multicolumn{2}{c}{2-terms} \\\hline
0001 & \checkmark & 00-1 & \\
\textcolor[rgb]{1.00,0.00,0.00}{0100} & \checkmark & 0-01 & \\\cline{1-2}
0011 & \checkmark & 010- & \\
0101 & \checkmark & \textcolor[rgb]{1.00,0.00,0.00}{01-0} & \\
\textcolor[rgb]{1.00,0.00,0.00}{0110} & \checkmark & & \\
1010 & & & \\\cline{1-2}
0111 & & & \\
1110 & & & \\\cline{1-2}
1111 & & & \\
\end{tabular}
&&
\begin{tabular}{cp{.1in}cp{.1in}}
\multicolumn{2}{c}{Minterms} & \multicolumn{2}{c}{2-terms} \\\hline
0001 & \checkmark & 00-1 & \\
0100 & \checkmark & 0-01 & \\\cline{1-2}
\textcolor[rgb]{1.00,0.00,0.00}{0011} & \checkmark & 010- & \\
0101 & \checkmark & 01-0 & \\
0110 & \checkmark & \textcolor[rgb]{1.00,0.00,0.00}{0-11} & \\
1010 & & & \\\cline{1-2}
\textcolor[rgb]{1.00,0.00,0.00}{0111} & \checkmark & & \\
1110 & & & \\\cline{1-2}
1111 & & & \\
\end{tabular}\\
\end{tabular}

\vspace{.1in}

\begin{tabular}{cp{.2in}cp{.2in}c}
\begin{tabular}{cp{.1in}cp{.1in}}
\multicolumn{2}{c}{Minterms} & \multicolumn{2}{c}{2-terms} \\\hline
0001 & \checkmark & 00-1 & \\
0100 & \checkmark & 0-01 & \\\cline{1-2}
0011 & \checkmark & 010- & \\
\textcolor[rgb]{1.00,0.00,0.00}{0101} & \checkmark & 01-0 & \\
0110 & \checkmark & 0-11 & \\
1010 &            & \textcolor[rgb]{1.00,0.00,0.00}{01-1} & \\\cline{1-2}
\textcolor[rgb]{1.00,0.00,0.00}{0111} & \checkmark & & \\
1110 & & & \\\cline{1-2}
1111 & & & \\
\end{tabular}
&&
\begin{tabular}{cp{.1in}cp{.1in}}
\multicolumn{2}{c}{Minterms} & \multicolumn{2}{c}{2-terms} \\\hline
0001 & \checkmark & 00-1 & \\
0100 & \checkmark & 0-01 & \\\cline{1-2}
0011 & \checkmark & 010- & \\
0101 & \checkmark & 01-0 & \\
\textcolor[rgb]{1.00,0.00,0.00}{0110} & \checkmark & 0-11 & \\
1010 &            & 01-1 & \\\cline{1-2}
\textcolor[rgb]{1.00,0.00,0.00}{0111} & \checkmark & \textcolor[rgb]{1.00,0.00,0.00}{011-} & \\
1110 & & & \\\cline{1-2}
1111 & & & \\
\end{tabular}
&&
\begin{tabular}{cp{.1in}cp{.1in}}
\multicolumn{2}{c}{Minterms} & \multicolumn{2}{c}{2-terms} \\\hline
0001 & \checkmark & 00-1 & \\
0100 & \checkmark & 0-01 & \\\cline{1-2}
0011 & \checkmark & 010- & \\
0101 & \checkmark & 01-0 & \\
\textcolor[rgb]{1.00,0.00,0.00}{0110} & \checkmark & 0-11 & \\
1010 &            & 01-1 & \\\cline{1-2}
0111 & \checkmark & 011- & \\
\textcolor[rgb]{1.00,0.00,0.00}{1110} & \checkmark & \textcolor[rgb]{1.00,0.00,0.00}{-110} & \\\cline{1-2}
1111 & & & \\
\end{tabular}\\
\end{tabular}

\vspace{.1in}

\begin{tabular}{cp{.2in}cp{.2in}c}
\begin{tabular}{cp{.1in}cp{.1in}}
\multicolumn{2}{c}{Minterms} & \multicolumn{2}{c}{2-terms} \\\hline
0001 & \checkmark & 00-1 & \\
0100 & \checkmark & 0-01 & \\\cline{1-2}
0011 & \checkmark & 010- & \\
0101 & \checkmark & 01-0 & \\
0110 & \checkmark & 0-11 & \\
\textcolor[rgb]{1.00,0.00,0.00}{1010} & \checkmark & 01-1 & \\\cline{1-2}
0111 & \checkmark & 011- & \\
\textcolor[rgb]{1.00,0.00,0.00}{1110} & \checkmark & -110 & \\\cline{1-2}
1111 &            & \textcolor[rgb]{1.00,0.00,0.00}{1-10} & \\
&&&\\
\end{tabular}
&&
\begin{tabular}{cp{.1in}cp{.1in}}
\multicolumn{2}{c}{Minterms} & \multicolumn{2}{c}{2-terms} \\\hline
0001 & \checkmark & 00-1 & \\
0100 & \checkmark & 0-01 & \\\cline{1-2}
0011 & \checkmark & 010- & \\
0101 & \checkmark & 01-0 & \\
0110 & \checkmark & 0-11 & \\
1010 & \checkmark & 01-1 & \\\cline{1-2}
\textcolor[rgb]{1.00,0.00,0.00}{0111} & \checkmark & 011- & \\
1110 & \checkmark & -110 & \\\cline{1-2}
\textcolor[rgb]{1.00,0.00,0.00}{1111} & \checkmark & 1-10 & \\
     &            & \textcolor[rgb]{1.00,0.00,0.00}{-111} &\\
     &&&\\
\end{tabular}
&&
\begin{tabular}{cp{.1in}cp{.1in}}
\multicolumn{2}{c}{Minterms} & \multicolumn{2}{c}{2-terms} \\\hline
0001 & \checkmark & 00-1 & \\
0100 & \checkmark & 0-01 & \\\cline{1-2}
0011 & \checkmark & 010- & \\
0101 & \checkmark & 01-0 & \\
0110 & \checkmark & 0-11 & \\
1010 & \checkmark & 01-1 & \\\cline{1-2}
0111 & \checkmark & 011- & \\
\textcolor[rgb]{1.00,0.00,0.00}{1110} & \checkmark & -110 & \\\cline{1-2}
\textcolor[rgb]{1.00,0.00,0.00}{1111} & \checkmark & 1-10 & \\
     &            & -111 & \\
     &            & \textcolor[rgb]{1.00,0.00,0.00}{111-} &\\
\end{tabular}\\
\end{tabular}

\vspace{.1in}



We now move on to generating the 4-term implicants from the 2-term implicants.  We do it in the exact same way as the preceding development.

\begin{tabular}{cp{.2in}c}
\begin{tabular}{cp{.1in}cp{.1in}cp{.1in}}
\multicolumn{2}{c}{Minterms} & \multicolumn{2}{c}{2-terms}  & \multicolumn{2}{c}{4-terms} \\\hline
0001 & \checkmark & \textcolor[rgb]{1.00,0.00,0.00}{00-1} & \checkmark & \textcolor[rgb]{1.00,0.00,0.00}{0-$\,$-1} &  \\
0100 & \checkmark & 0-01 &            &  &  \\\cline{1-2}
0011 & \checkmark & 010- &            &  &  \\
0101 & \checkmark & 01-0 &            &  &  \\\cline{3-4}
0110 & \checkmark & 0-11 &            &  &  \\
1010 & \checkmark & \textcolor[rgb]{1.00,0.00,0.00}{01-1} & \checkmark &  &  \\\cline{1-2}
0111 & \checkmark & 011- &            &  &  \\
1110 & \checkmark & -110 &            &  &  \\\cline{1-2}
1111 & \checkmark & 1-10 &            &  &  \\\cline{3-4}
     &            & -111 &            &  &  \\
     &            & 111- &            &  &  \\
\end{tabular}
&&
\begin{tabular}{cp{.1in}cp{.1in}cp{.1in}}
\multicolumn{2}{c}{Minterms} & \multicolumn{2}{c}{2-terms}  & \multicolumn{2}{c}{4-terms} \\\hline
0001 & \checkmark & 00-1 & \checkmark & 0-$\,$-1 &  \\
0100 & \checkmark & \textcolor[rgb]{1.00,0.00,0.00}{0-01} & \checkmark & \textcolor[rgb]{1.00,0.00,0.00}{0-$\,$-1} &  \\\cline{1-2}
0011 & \checkmark & 010- &            &  &  \\
0101 & \checkmark & 01-0 &            &  &  \\\cline{3-4}
0110 & \checkmark & \textcolor[rgb]{1.00,0.00,0.00}{0-11} & \checkmark &  &  \\
1010 & \checkmark & 01-1 & \checkmark &  &  \\\cline{1-2}
0111 & \checkmark & 011- &            &  &  \\
1110 & \checkmark & -110 &            &  &  \\\cline{1-2}
1111 & \checkmark & 1-10 &            &  &  \\\cline{3-4}
     &            & -111 &            &  &  \\
     &            & 111- &            &  &  \\
\end{tabular}\\
\end{tabular}

\vspace{.1in}

\begin{tabular}{cp{.2in}c}
\begin{tabular}{cp{.1in}cp{.1in}cp{.1in}}
\multicolumn{2}{c}{Minterms} & \multicolumn{2}{c}{2-terms}  & \multicolumn{2}{c}{4-terms} \\\hline
0001 & \checkmark & 00-1 & \checkmark & 0-$\,$-1 &  \\
0100 & \checkmark & 0-01 & \checkmark & 0-$\,$-1 &  \\\cline{1-2}
0011 & \checkmark & \textcolor[rgb]{1.00,0.00,0.00}{010-} & \checkmark & \textcolor[rgb]{1.00,0.00,0.00}{01-$\,$-} &  \\
0101 & \checkmark & 01-0 &            &  &  \\\cline{3-4}
0110 & \checkmark & 0-11 & \checkmark &  &  \\
1010 & \checkmark & 01-1 & \checkmark &  &  \\\cline{1-2}
0111 & \checkmark & \textcolor[rgb]{1.00,0.00,0.00}{011-} & \checkmark &  &  \\
1110 & \checkmark & -110 &            &  &  \\\cline{1-2}
1111 & \checkmark & 1-10 &            &  &  \\\cline{3-4}
     &            & -111 &            &  &  \\
     &            & 111- &            &  &  \\
\end{tabular}
&&
\begin{tabular}{cp{.1in}cp{.1in}cp{.1in}}
\multicolumn{2}{c}{Minterms} & \multicolumn{2}{c}{2-terms}  & \multicolumn{2}{c}{4-terms} \\\hline
0001 & \checkmark & 00-1 & \checkmark & 0-$\,$-1 &  \\
0100 & \checkmark & 0-01 & \checkmark & 0-$\,$-1 &  \\\cline{1-2}
0011 & \checkmark & 010- & \checkmark & 01-$\,$- &  \\
0101 & \checkmark & \textcolor[rgb]{1.00,0.00,0.00}{01-0} & \checkmark & \textcolor[rgb]{1.00,0.00,0.00}{01-$\,$-} &  \\\cline{3-4}
0110 & \checkmark & 0-11 & \checkmark &  &  \\
1010 & \checkmark & \textcolor[rgb]{1.00,0.00,0.00}{01-1} & \checkmark &  &  \\\cline{1-2}
0111 & \checkmark & 011- & \checkmark &  &  \\
1110 & \checkmark & -110 &            &  &  \\\cline{1-2}
1111 & \checkmark & 1-10 &            &  &  \\\cline{3-4}
     &            & -111 &            &  &  \\
     &            & 111- &            &  &  \\
\end{tabular}\\
\end{tabular}

\vspace{.1in}

\begin{tabular}{cp{.2in}c}
\begin{tabular}{cp{.1in}cp{.1in}cp{.1in}}
\multicolumn{2}{c}{Minterms} & \multicolumn{2}{c}{2-terms}  & \multicolumn{2}{c}{4-terms} \\\hline
0001 & \checkmark & 00-1 & \checkmark & 0-$\,$-1 &  \\
0100 & \checkmark & 0-01 & \checkmark & 0-$\,$-1 &  \\\cline{1-2}
0011 & \checkmark & 010- & \checkmark & 01-$\,$- &  \\
0101 & \checkmark & 01-0 & \checkmark & 01-$\,$- &  \\\cline{3-4}
0110 & \checkmark & 0-11 & \checkmark & \textcolor[rgb]{1.00,0.00,0.00}{-11-}     &  \\
1010 & \checkmark & 01-1 & \checkmark &  &  \\\cline{1-2}
0111 & \checkmark & \textcolor[rgb]{1.00,0.00,0.00}{011-} & \checkmark &  &  \\
1110 & \checkmark & -110 &            &  &  \\\cline{1-2}
1111 & \checkmark & 1-10 &            &  &  \\\cline{3-4}
     &            & -111 &            &  &  \\
     &            & \textcolor[rgb]{1.00,0.00,0.00}{111-} & \checkmark &  &  \\
\end{tabular}
&&
\begin{tabular}{cp{.1in}cp{.1in}cp{.1in}}
\multicolumn{2}{c}{Minterms} & \multicolumn{2}{c}{2-terms}  & \multicolumn{2}{c}{4-terms} \\\hline
0001 & \checkmark & 00-1 & \checkmark & 0-$\,$-1 &  \\
0100 & \checkmark & 0-01 & \checkmark & 0-$\,$-1 &  \\\cline{1-2}
0011 & \checkmark & 010- & \checkmark & 01-$\,$- &  \\
0101 & \checkmark & 01-0 & \checkmark & 01-$\,$- &  \\\cline{3-4}
0110 & \checkmark & 0-11 & \checkmark & -11-     &  \\
1010 & \checkmark & 01-1 & \checkmark & \textcolor[rgb]{1.00,0.00,0.00}{-11-}     &  \\\cline{1-2}
0111 & \checkmark & 011- & \checkmark &  &  \\
1110 & \checkmark & \textcolor[rgb]{1.00,0.00,0.00}{-110} & \checkmark &  &  \\\cline{1-2}
1111 & \checkmark & 1-10 &            &  &  \\\cline{3-4}
     &            & \textcolor[rgb]{1.00,0.00,0.00}{-111} & \checkmark &  &  \\
     &            & 111- & \checkmark &  &  \\
\end{tabular}\\
\end{tabular}

The prime implicants are thus:

\begin{tabular}{cccc}
0-$\,$-1 & 01-$\,$- &  -11- & 1-10 \\
A'D      & A'B      & BC    & ACD' \\
\end{tabular}

Now let's add some don't care conditions
\begin{eqnarray}
\sum_{abcd}(1,3,4,5,6,7,10,14,15)+DC(2,9,11)
\end{eqnarray}
I will keep the old chart and just add in the new terms to save space.

\vspace{.1in}

\begin{tabular}{cp{.2in}c}
\begin{tabular}{cp{.1in}cp{.1in}cp{.1in}}
\multicolumn{2}{c}{Minterms} & \multicolumn{2}{c}{2-terms}  & \multicolumn{2}{c}{4-terms} \\\hline
0001 & \checkmark & 00-1 & \checkmark & 0-$\,$-1 &  \\
\textcolor[rgb]{0.00,0.00,1.00}{0010} &  & 0-01 & \checkmark & 01-$\,$- &  \\
0100 & \checkmark & 010- & \checkmark & -11- &  \\\cline{1-2}
0011 & \checkmark & 01-0 & \checkmark &  &  \\\cline{3-4}
0101 & \checkmark & 0-11 & \checkmark &      &  \\
0110 & \checkmark & 01-1 & \checkmark &      &  \\
\textcolor[rgb]{0.00,0.00,1.00}{1001} &  & 011- & \checkmark &  &  \\
1010 & \checkmark & -110 & \checkmark &  &  \\\cline{1-2}
0111 & \checkmark & 1-10 &            &  &  \\\cline{3-4}
\textcolor[rgb]{0.00,0.00,1.00}{1011} &  & -111 & \checkmark &  &  \\
1110     & \checkmark & 111- & \checkmark &  &  \\\cline{1-2}
1111     & \checkmark &      &            &  &  \\
\end{tabular}
&&
\begin{tabular}{cp{.1in}cp{.1in}cp{.1in}}
\multicolumn{2}{c}{Minterms} & \multicolumn{2}{c}{2-terms}  & \multicolumn{2}{c}{4-terms} \\\hline
0001 & \checkmark & 00-1 & \checkmark & 0-$\,$-1 &  \\
\textcolor[rgb]{0.00,0.00,1.00}{0010} &  & 0-01 & \checkmark & 01-$\,$- &  \\
0100 & \checkmark & 010- & \checkmark & -11- &  \\\cline{1-2}
0011 & \checkmark & 01-0 & \checkmark &  &  \\\cline{3-4}
0101 & \checkmark & 0-11 & \checkmark &      &  \\
0110 & \checkmark & 01-1 & \checkmark &      &  \\
\textcolor[rgb]{0.00,0.00,1.00}{1001} &  & 011- & \checkmark &  &  \\
1010 & \checkmark & -110 & \checkmark &  &  \\\cline{1-2}
0111 & \checkmark & 1-10 &            &  &  \\\cline{3-4}
\textcolor[rgb]{0.00,0.00,1.00}{1011} &  & -111 & \checkmark &  &  \\
1110     & \checkmark & 111- & \checkmark &  &  \\\cline{1-2}
1111     & \checkmark &      &            &  &  \\
\end{tabular}\\
\end{tabular}

\section{Espresso}

Quine-McClusky is thorough and equivalent to Karnaugh maps - thus you will get the same benefits.  The disadvantage is it is very slow, since circuit minimization is NP-Hard (technically $\Sigma_2^P-$complete).  For synchronous circuits, which are typical in programable logic, there is no need to worry about errors of type-1 if the timing requirements are met.  While an PLA, FPGA, etc., could use an non-minimized logic circuit, redundancy wastes space and time so we would like to get a good solution in reasonable time.  Enter heuristic algorithms.  Espresso is one of the best known heuristic algorithms, and in practice gets good results in reasonable time.  In theory it is not guaranteed.

\subsection{Algorithm}
Espresso produces a list of products that completely covers the on-set but does not cover any of the off-set.  None, part, or all of the don't care set may be covered.  Let's first define these sets:

\begin{description}
\item[On-Set($\mathcal{F}$)] The minterms whose output is 1
\item[Don't-Care-Set($\mathcal{D}$)] The minterms whose output value does not matter
\item[Off-Set($\mathcal{R}$)] The minterms whose output is 0
\end{description}

The basic Espresso algorithm is three steps:
\begin{enumerate}
\item Expand
\item Irredundant Cover
\item Reduce
\end{enumerate}

Start with an initial cover (for instance generate randomly or specify minterms).

\subsubsection{Expand}
The basic idea/flow of the algorithm is to expand the implicants, $i$, in the current cover until the cover only has prime implicants.  One important idea is how to order the expansion.  This is done by finding the weight, $w$, of each literal, $l$.  The literals are the variables and their primes, that compose the implicants.  For instance, the implicant $xy'$ is composed of the literals $x$ and $y'$, which implies the existence of literals $x'$ and $y$.  The list of all literals in a problem is in the set $L$.  First, we sort the implicants by how much its literals are used.  This is kind of a sparsity measure, and by selecting the implicants with the lowest weights we start with the implicants that are in less covered regions generally.

\begin{algorithm}[H]
 \KwData{$\mathcal{F}$, $L$}
 \KwResult{$\mathcal{F}$ with cover weights}
 \ForEach{$i \in \mathcal{F}$}{
   \ForEach{$l\in L$}{
     \If{$i(l)=1$}{
       $w(l)++$
     }
     \ElseIf{$i(l)=0$}{
       $w(l')++$
     }
     \Else{
       $w(l)++$
       $w(l')++$
     }
   }
 }
 \caption{Calculate Cover Weights}
\end{algorithm}

Once we have selected an implicant we will look at all the directions we can expand (reducing the literals in the term) and see if they result in a valid implicant (i.e. only contains members of $\mathcal{F}$ and $\mathcal{D}$).  Select the resulting implicant that contains the largest number of other implicants in the cover.  Next try to maximize the overlaps in terms by expansion - this induces redundancies that we can try to eliminate later.  Finally expand to be as large as possible.  Try all possible expansions.


\begin{algorithm}[H]
 \KwData{$\mathcal{F}$}
 \KwResult{$\mathcal{F}$ with only prime implicants}
 
 Calculate cover weights
 \Repeat{all implicants are prime}{
  select lowest weight implicant that is not prime from cover
  generate all prime implicants that contain the selected implicant
  select the prime implicant that contains the largest number of implicants from the current cover
  delete all implicants in the current cover contained in the new prime implicant
 }
 \caption{Expand}
\end{algorithm}


% Define supercube(A,X) =  smallest cube which contains both A and X


\subsubsection{Irredundant Cover}

The goal of this stage is to make F irredundant, i.e. to delete the maximum number of implicants, without invalidating the cover.  Given the expense of the prime implicant table used in Quine-McCluskey, one could be tempted to use a greedy algorithm to remove one redundant implicant at a time till no redundant implicants remain.  Unfortunately, this is suboptimal.  Interestingly the optimal solution of solving the PI table is not generally costly in this case since you don't have all prime implicants due to the construction of the expansion step, so this is the approach used in Expresso.  Quine-McCluskey has has a prime generation step, so it has a larger table, and usually it is much larger.

The irredundant cover step thus sets up a simplified PI table (as a graph or array) and solves it exactly to determine the minimum number of prime implicants to keep.  Redundant implicants are deleted.





\subsubsection{Reduce}

Now for the step that will seem the strangest, we will reduce the size of the implicants as much as possible without compromising the cover.  This is done to provide a good starting point for the next iteration.  Without it, there would only be one iteration in the algorithm.  Note that reduce is highly dependent on the order the implicants are reduced.


\subsubsection{Post Processing}

Espresso attempts to expand to primes that are used in several of the output terms.  Each of the primes can be essential to one output, but not to others.  Some can even be redundant to some of the outputs.  The extra outputs the terms are driving is a fanout problem for the AND-gates.  This is a like reduce, but in this case only unnecessary output connections are deleted.  This is sometimes called ``make-sparse'' or ``reduce output parts''.


\subsection{Software}

Espresso is open source from UC Berkeley.  There are a number of free programs that use it as a back end to solve circuit minimization.  The basic problem description(input) and problem solution(output) formats are pretty easy in that they follow PLA programming standards.  .  The header section has three possible statements, the first two are in both the description and solution format, and the final is in the solution format only:
\begin{description}
\item[.i n] there are $n$ input variables.  Used in both.
\item[.o m] there are $m$ output variables.  Used in both.
\item[.v k] there are $k$ polynomial terms in the solution. Used only in the solution file.
\end{description}
The next section of the files has an input section on the left and an output section on the right.  This is multiple column.  For the problem description (input file) this is the table to be minimized.  For the solution it is the input logic and output logic.

Finally all files are terminated by a .e 