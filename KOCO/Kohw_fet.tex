\chapter{Field Effect Transistors}

\section{Ideal Behavior}

A Field Effect Transistor (FET) is in one sense essentially a capacitor, and thus it is governed by
\begin{eqnarray}
CV &=& Q\\
C_{gb}(V_{gc}-V_t) &=& Q_{channel}
\end{eqnarray}
where,
\begin{itemize}
\item $C_{gb}$ is the capacitance between the gate and the body, this is often just called the gate capacitance.
\item $V_{gc}$ is the Voltage between the gate and the channel.  Note that $V_c=V_{ds}/2$, so $V_{gc}=V_gs-V_{ds}/2$.
\item $V_t$ is the threshold voltage, i.e. the minimum voltage to cause an inversion layer to form.
\item $Q_{channel}$ is the charge carries available in the channel to conduct.
\end{itemize}
The more charge carriers, $Q_{channel}$, the easier the current will flow, so calculating this is an essential step to quantitatively analyzing a FET.  First we need to find out what our capacitance, $C_{gb}$ is, this is done by
\begin{eqnarray}
C_{gb} &=& \varepsilon_{ox}\frac{WL}{t_{ox}} \\
&=& 3.9\varepsilon_0\frac{WL}{t_{ox}}
\end{eqnarray}
\begin{itemize}
\item $\varepsilon_{ox}$ is the permittivity\footnote{Permittivity is the resistance to forming an electric field.} of the insulating oxide layer.  Note: the 3.9 is the relative permittivity of silicon dioxide compared the the permittivity of free space.  Relative permittivity is denoted $\varepsilon_r$, and varies by material, frequency, temperature, and sometimes even direction.  We will treat it as a constant, which is ok for our operating situation.
\item $\varepsilon_0$ is the permittivity of free space ($8.85 \times 10^{-14}$ [F/cm]).
\item $W$ is the width of the gate (along source and drain).
\item $L$ is the length under the gate (between source and drain).
\end{itemize}
Now we want to get the current flowing in the channel, but that means we need to know how fast they are moving.  In a semiconductor the velocity of the charge carrier, $v_c$, is given by
\begin{eqnarray}
v_c &=& \mu_cE\\
&=&\mu_c\frac{V_{ds}}{L}
\end{eqnarray}
where $\mu_c$ is the mobility of the charge carrier.  The length of the channel divided by the velocity of the carriers, gives us the time for a charge to cross the channel, $T_{channel}$.  The total charge in the channel divided by this time is then the current.
\begin{eqnarray}
i_{ds} &=& \frac{Q_{channel}}{T_{channel}}\\
 &=& \frac{C_{gb}(V_{gc}-V_t)}{\frac{L}{v_c}}\\
 &=& \frac{3.9\varepsilon_0\frac{WL}{t_{ox}}(V_{gc}-V_t)}{\frac{L}{\mu_c\frac{V_{ds}}{L}}}\\
 &=& 3.9\varepsilon_0\frac{W}{t_{ox}}(V_{gs}-V_t)\mu_c\frac{V_{ds}}{L}\\
 &=& \frac{3.9\varepsilon_0}{t_{ox}}\mu_c\frac{W}{L}(V_{gc}-V_t)V_{ds}\\
 &=& \frac{3.9\varepsilon_0}{t_{ox}}\mu_c\frac{W}{L}(V_{gs}-V_t-V_{ds}/2)V_{ds}
\end{eqnarray}


\section{Amplification}

\begin{eqnarray}
i_{DS} &=& \frac{k}{2}(V_{GS}-V_T)^2
\end{eqnarray}
if $V_{DS} \geq V_{GS}-V_T\geq 0$. 