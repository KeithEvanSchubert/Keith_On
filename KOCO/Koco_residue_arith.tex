\chapter{Residue Arithmetic}

We have shown different ways of calculating the sum and product of binary numbers.  In this section we will examine a different way to represent numbers and thus to calculate.  In residue arithmetic numbers are represented by their remainders of a group of numbers that constitute the basis of the representation.   Let's consider a simple example of how numbers can be represented in this method.

\begin{tabular}{|l|cc|} \hline
  Number & \%2 & \%3 \\ \hline
  0      & 0   & 0   \\
  1      & 1   & 1   \\ \hline
  2      & 0   & 2   \\
  3      & 1   & 0   \\ \hline
  4      & 0   & 1   \\
  5      & 1   & 2   \\ \hline
\end{tabular}

Note that each of the numbers from 0 through 5 can be represented uniquely by their remainders.  Note that the number 6 would be 0,0 and thus not distinguishable from 0.  You can represent six numbers (1-5) because the product of the basis numbers is $2\times 3=6$.  That we can represent the numbers is one thing, being able to calculate easily is another.  Lets consider addition first:

\begin{tabular}{r@{=}lcr@{=}l}
1 & 1,1               && 2 & 0,2 \\
2 & 0,2               && 3 & 1,0 \\ \cline{1-2}\cline{4-5}
3 & (0+1)\%2,(1+2)\%3 && 5 & (0+1)\%2,(2+0)\%3 \\
  & 1,0               &&   & 1,2 \\
\end{tabular}

If you look up (1,0) in our table you will find it corresponds to 3, similarly (1,2) corresponds to 5.  Now lets try some multiplication problems:

\begin{tabular}{r@{=}lcr@{=}l}
2 & 0,2                               && 1 & 1,1 \\
2 & 0,2                               && 3 & 1,0 \\ \cline{1-2}\cline{4-5}
4 & ($0\times 0$)\%2,($2\times 2$)\%3 && 3 & ($1\times 1$)\%2,($1\times 0$)\%3 \\
  & 0,1                               &&   & 1,0 \\
\end{tabular}

If you look up (0,1) in our table you will find it corresponds to 4, similarly (1,0) corresponds to 3.  Subtraction is slightly more complex, similar to the 2's complement\footnote{In fact it is a radix complement, in particular since for our example their are 6 numbers in our example, we will be calculating the 6's complement and then finding its residue.} an inverse of each remainder (the representation) must be found.  This is done by subtracting each remainder from the number it was modulused from.  This is easiest to see in an example.

\begin{example}
First, let's get a table of the numbers and their negatives (additive inverses):

\begin{tabular}{|l|r@{,}l|r@{,}l|l|} \hline
  Number &\multicolumn{2}{|c|}{Residue}&\multicolumn{2}{|c|}{Negative} & Negative \\
  Decimal& \%2 & \%3 & \%2        & \%3        & Decimal\\ \hline
  0      & 0   & 0   & (2-0)\%2=0 & (3-0)\%3=0 & 0 \\
  1      & 1   & 1   & (2-1)\%2=1 & (3-1)\%3=2 & 5 \\ \hline
  2      & 0   & 2   & (2-0)\%2=0 & (3-2)\%3=1 & 4 \\
  3      & 1   & 0   & (2-1)\%2=1 & (3-0)\%3=0 & 3 \\ \hline
  4      & 0   & 1   & (2-0)\%2=0 & (3-1)\%3=2 & 2 \\
  5      & 1   & 2   & (2-1)\%2=1 & (3-2)\%3=1 & 1 \\ \hline
\end{tabular}

Now let's do some calculations.

\beqn
5-2 & = & (1,2)-(0,2) \\
    & = & (1,2)+(0,1) \\
    & = & (1+0,2+1) \\
    & = & (1,0) \\
    & = & 3
\eeqn

\beqn
4-4 & = & (0,1)-(0,1) \\
    & = & (0,1)+(0,2) \\
    & = & (0+0,1+2) \\
    & = & (0,0) \\
    & = & 0
\eeqn

\beqn
2-1 & = & (0,2)-(1,1) \\
    & = & (0,2)+(1,2) \\
    & = & (0+1,2+2) \\
    & = & (1,1) \\
    & = & 1
\eeqn
\end{example}

The basis of the representation must be relatively prime, that is they must have unique prime factors (they cannot share prime factors with other basis numbers).  This means that you can have a number like 4 ($2\times 2$) as long as no other basis had 2 as a factor, but you could not have 9 ($3\times 3$) and 12 ($2\times 2\times 3$), or 6 ($2\times 3$) and 10 ($2\times 5$) in the same basis.  To see why consider the basis (4,6), it should give unique representations for $4\times 6=24$ numbers (0-23).

\begin{tabular}{|r|cc||r|cc|} \hline
  Number & \%4 & \%6 & Number & \%4 & \%6 \\ \hline
  0      & 0   & 0   & 12     & 0   & 0   \\
  1      & 1   & 1   & 13     & 1   & 1   \\ \hline
  2      & 2   & 2   & 14     & 2   & 2   \\
  3      & 3   & 3   & 15     & 3   & 3   \\ \hline
  4      & 0   & 4   & 16     & 0   & 4   \\
  5      & 1   & 5   & 17     & 1   & 5   \\ \hline
  6      & 2   & 0   & 18     & 2   & 0   \\
  7      & 3   & 1   & 19     & 3   & 1   \\ \hline
  8      & 0   & 2   & 20     & 0   & 2   \\
  9      & 1   & 3   & 21     & 1   & 3   \\ \hline
  10     & 2   & 4   & 22     & 2   & 4   \\
  11     & 3   & 5   & 23     & 3   & 5   \\ \hline
\end{tabular}

Notice the first and second column are the same, and thus do not give us the full range we wanted.
