\chapter{Addressing}
\label{c-address}


\begin{itemize}
  \item .bss
  \item .data
  \item .text
\end{itemize}

\begin{description}
  \item[.bss]  (block started by symbol) memory, reserved only
  \item[.data] memory, predefined values
  \item[.text] instructions
  \item[.reserve val] (alternately ".skip val") sets aside val bytes of memory
  \item[.equate name, val] (alternately ".set name, val") makes name a constant with value val
  \item[.byte val] (alternately .b, ub, sb) specifies the operation to be on a byte
  \item[.half val] (alternately .h, uh, sh) specifies the operation to be on a half word (2 bytes)
  \item[.word val] (alternately .w) specifies the operation to be on a word (4 bytes)
  \item[.align val] aligns the memory location counter
\end{description}

Note that val may be a constant expression for readability.

\noindent
\begin{tabular}{llll}
Name & Generic & Sparc & Uses \\ \hline
memory direct & mX & [\%r0+X] & \\
register direct & rX & \%rX & \\
immediate & \#X & X & \\
memory indirect & @mX & - & pointers \\
register indirect & @rX & [\%rX] & pointers \\
memory indexed & label[mX] & - & arrays \\
register indexed & label[rX] & [\%rY + \%rX] & arrays \\
 & & & (note \%rY is loaded with label) \\
pre-increment & +[rX] & - & increments by size (stride) each time \\
post-increment & [rX]+ & - & increments by size (stride) each time \\
pre-decrement & -[rX] & - & decrements by size (stride) each time \\
post-decrement & [rX]- & - & decrements by size (stride) each time \\
memory displaced & mX $\rightarrow$ label & - & struct \\
register displaced & rX $\rightarrow$ label & [\%rX + label] & struct \\
\end{tabular}


\begin{tabular}{cc}
\begin{tabular}{r|l|l|l|l|} \cline{2-5}
  m0  & 0x00 & 0x00 & 0x00 & 0x12 \\ \cline{2-5}
  m4  & 0x00 & 0x00 & 0x00 & 0x08 \\ \cline{2-5}
  m8  & 0x01 & 0x23 & 0x45 & 0x67 \\ \cline{2-5}
  m12 & 0x89 & 0xAB & 0xCD & 0xEF \\ \cline{2-5}
  m16 & 0x12 & 0x34 & 0x56 & 0x78 \\ \cline{2-5}
  m20 & 0x9A & 0xBC & 0xDE & 0xF0 \\ \cline{2-5}
  m24 & 0x11 & 0x11 & 0x11 & 0x11 \\ \cline{2-5}
\end{tabular}

&

\begin{tabular}{r|l|l|l|l|} \cline{2-5}
  r0  & 0x00 & 0x00 & 0x00 & 0x00 \\ \cline{2-5}
  r1  & 0x00 & 0x00 & 0x00 & 0x08 \\ \cline{2-5}
  r2  & 0x00 & 0x00 & 0x00 & 0x0C \\ \cline{2-5}
  r3  & 0x00 & 0x00 & 0x00 & 0x04 \\ \cline{2-5}
  r4  & 0x00 & 0x00 & 0x00 & 0x10 \\ \cline{2-5}
\end{tabular}
\end{tabular}
\vspace{.1in}

Let var1 be a label for the value 8.

\vspace{.1in}
\begin{tabular}{llll}
Representation & X=4 & Effective Address & Expression \\ \hline
mX & m4 & 0x00000004 & 0x00000008 \\
rX & r4 & - & 0x00000010 \\
\#X & \#4 & - & 0x00000004 \\
@mX & @m4 & 0x00000008 & 0x01234567 \\
@rX & @r4 & 0x00000010 & 0x12345678 \\
var1[mX] & 8[m4] & 0x00000010 (i.e.: 8+8) & 0x12345678 \\
var1[rX] & 8[r4] & 0x00000018 (i.e.: 8+16) & 0x11111111 \\
+[rX] & +[r4] & 0x00000014 & 0x9ABCDEF0 \\
      &       &  & r4 $\leftarrow$ 0x00000014 before \\
$[$rX]+ & [r4]+ & 0x00000010 & 0x12345678 \\
      &       &  & r4 $\leftarrow$ 0x00000014 after \\
-[rX] & -[r4] & 0x0000000C & 0x89ABCDEF \\
      &       &  & r4 $\leftarrow$ 0x0000000C before \\
$[$rX]- & [r4]- & 0x00000010 & 0x12345678 \\
      &       &  & r4 $\leftarrow$ 0x0000000C after \\
mX $\rightarrow$ var1 & m4 $\rightarrow$ 8 & 0x00000010 (i.e.: 8+8) & 0x12345678 \\
rX $\rightarrow$ var1 & r4 $\rightarrow$ 8 & 0x00000018 (i.e.: 8+16) & 0x11111111 \\
\end{tabular}


\subsection{Arrays}

For instance consider an array of 10 integers.

\begin{verbatim}
   int my_int[10];
\end{verbatim}

This creates both the array of integers and a pointer to the first element.  The elements are numbered 0 to 9 and are accessed by $my_int[i]$ for $i\in\{0,1,\ldots,9\}$.  They can also be accessed by $*(my_int+i)$.  In assembly we would have:

\begin{verbatim}
  my_int:  .skip 10*4    ; each int is 4 bytes
\end{verbatim}

The contents can be accessed by:

\begin{verbatim}
  set i, %r2
  ld  [%r2], %r2
  umul %r2, 4, %r3
  set my_int, %r4

  ld [%r4 + %r3], %r5
\end{verbatim}

or if my\_int (the address) fits in a 13 bit signed constant:

\begin{verbatim}
  set i, %r2
  ld  [%r2], %r2
  umul %r2, 4, %r3

  ld [%r3+my_int], %r5
\end{verbatim}

Essentially the address is my\_int + i*4, but this assumes that start of my array is zero.  How about a language like Pascal or VB which allows other starting values?  Consider defining an array (-m,-m+1,\ldots, -1, 0, 1, \ldots, n).  To use the address my\_int + i*size we have

\begin{verbatim}
          .skip m*size       ! negatives
          .skip (n+1)*size   ! zero and positives
\end{verbatim}

Alternately,

\begin{verbatim}
         .skip (m+n+1)*size  ! whole thing
\end{verbatim}

This causes the address to be my\_int + (i+m)*size.  Now you might think this will be longer, but note that it can be rewritten as

\vspace{.1in}
\begin{tabular}{l}
    my\_int + (i+m)*size \\
    my\_int + i*size + m*size \\
    (my\_int + m*size) + i*size \\
\end{tabular}
\vspace{.1in}

That is, rather than constantly biasing the index, it makes more sense to bias the base.  Essentially it makes the second method look like the first, but it works for a positive starting number (by making m a negative).  Since it is more general the later form is what is used in practice.

\subsection{String Storage}

\begin{description}
  \item[string256] (aka length plus value) length of string in first byte, string following
  \item[NULL terminated] string followed by 0
\end{description}


\subsection{Structs}

\begin{verbatim}
struct book{
  int pages;
  float price;
  char title[20];
}library[100];
\end{verbatim}

Would be implemented:

\begin{verbatim}
         .set pages, 0
         .set price, 4
         .set title, 8
         .set book_size, 28

.bss

library: .skip 100*book_size
\end{verbatim}

.bss is done in .data on some assemblers or machines 